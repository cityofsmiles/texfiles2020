\documentclass[14pt]{beamer}
\usepackage[utf8]{inputenc}
\usepackage{bookman} %font style
\usepackage{multicol}
\usepackage{cancel}
\usepackage{tikz}
\usepackage{xcolor} % for font color
\usetikzlibrary{calc}
\usetikzlibrary{positioning}
\usetikzlibrary{arrows.meta}
\usetikzlibrary{angles,quotes}
%\usetikzlibrary{decorations.pathreplacing}
\usepackage{wasysym} %for checked symbol 
\usepackage{tabularx} 
\usepackage{color, colortbl} %coloring table cells
\usepackage{gensymb} %degree symbol
\usepackage{amsfonts} %integer symbol
\usepackage{pgfplots} %graphs
\usepackage{easytable} % TAB
\usepackage{siunitx} % celsius symbol
\usepackage{stackengine} %to define \pesos 
\usepackage{systeme} % system of equations
\usepackage{pifont} % cross mark

\newcommand\pesos{\stackengine{-1.1ex}{P}{\stackengine{-1ex}{$-$}{$-$}{O}{c}{F}{F}{S}}{O}{c}{F}{T}{S}} 

\definecolor{Lgray}{gray}{0.9}
\definecolor{Dgray}{gray}{0.6}

\newcolumntype{Y}{>{\centering\arraybackslash}X} %for tabularx

\makeatletter
\newlength\beamerleftmargin
\setlength\beamerleftmargin{\Gm@lmargin}
\makeatother

\newcommand{\void}{\text{\hspace{2em}}}

\newcommand{\minivoid}{\text{\hspace{1em}}}

\newcommand{\sol}[3]{\begin{tikzpicture}[remember picture, overlay]
		\color{red}
        \node at (current page.south west) (sw) {};
		\node[anchor=south west, inner sep=0pt] at ($(sw) + (#1, #2) $)   (a) {
			\begin{minipage}[t]{0.75\textwidth}
				#3
			\end{minipage}};
    \end{tikzpicture}}


\newcommand \redcheck {{\color{red}\checkmark}}

\newcommand \redcross {{\color{red}\ding{56}}}

\newcommand{\vone}{\vspace{1em}}

\newcommand{\vhalf}{\vspace{0.5em}}

\newcommand{\mygrid}[2]{\begin{center}
		\begin{tikzpicture}[scale=#1]		
			\draw [help lines] (-10, -10) grid (10, 10);
			\draw[line width=0.5mm, <->, >={Latex[round]}] (-10, 0) -- (10, 0);
			\draw[line width=0.5mm, <->, >={Latex[round]}] (0, -10) -- (0, 10);	
			\input{#2}
		\end{tikzpicture}  
	\end{center}} 

\newcommand{\arrowcomment}[9]{\begin{tikzpicture}[remember picture, overlay]
		\node at (current page.south west) (sw) {};
		\node[anchor=south west, inner sep=0pt] at ($(sw) + (0, 0) $)   (a) {
			\begin{tikzpicture}[->,>=stealth, thick, main node/.style={rectangle,font=\sffamily\bfseries}, remember picture, overlay]

				\node (1)  at (#1, #2) {};

				\node[main node] (2)  at (#3, #4) {#5};

				\draw [->, red] (1.#6) to [out=#8,in=#9] (2.#7);

			\end{tikzpicture}
		};
\end{tikzpicture}} 

\newcommand{\plotit}[3]{\begin{center}
		\begin{tikzpicture}[scale=#2, main node/.style={rectangle,font=\sffamily\bfseries}]		
			\draw [help lines] (-#3, -#3) grid (#3, #3);
			\draw[line width=0.5mm, <->, >={Latex[round]}] (-#3, 0) -- (#3, 0);
			\draw[line width=0.5mm, <->, >={Latex[round]}] (0, -#3) -- (0, #3);	
			\input{#1}
		\end{tikzpicture}  
\end{center}} 

\newcommand{\plotpoint}[7]{\begin{center}
		\begin{tikzpicture}[scale=#7]		
			\coordinate (a) at (#1, #2);
			\draw [help lines] (-#5, -#5) grid (#5, #5);
			\draw[line width=#6 mm, <->, >={Latex[round]}] (-#5, 0) -- (#5, 0);
			\draw[line width=#6 mm, <->, >={Latex[round]}] (0, -#5) -- (0, #5);	
			\fill [fill=black] (a) circle (#3 pt);
			\node[anchor=#4, inner sep=2pt, rotate=0] (a-label) at (a) {$(#1, #2)$};
			\end{tikzpicture}  
	\end{center}} 

\newcommand{\plotoverlay}[5]{\begin{tikzpicture}[remember picture, overlay]
		\node at (current page.south west) (sw) {};
		\node[anchor=south west, inner sep=0pt] at ($(sw) + (#1, #2) $)   (a) {
			\begin{minipage}[t]{0.75\textwidth}
				 \begin{tikzpicture}[scale=#4, remember picture, overlay]		

						\draw [help lines] (-#5, -#5) grid (#5, #5);

						\draw[line width=0.5mm, <->, >={Latex[round]}] (-#5, 0) -- (#5, 0);

						\draw[line width=0.5mm, <->, >={Latex[round]}] (0, -#5) -- (0, #5);	

						\input{#3}

				\end{tikzpicture}
		\end{minipage}};
\end{tikzpicture}}

\newcommand{\lcmthreebythree}[9]{\begin{tikzpicture}[remember picture, overlay]
		\color{red}
		\node at (current page.south west) (sw) {};
		\node[anchor=south west, inner sep=0pt] at ($(sw) + (#1, #2) $)   (a) {
			\begin{minipage}[t]{0.75\textwidth}
				\pause Find the LCM: \\

				\begin{tabular}{rclll}

					$ \pause #3 $ & $=$ & $ \pause #6 $ & & \\

					$ \pause #4 $ & $=$ &  & $ \pause #7$ & \\

					$ \pause #5  $ & $=$ &  & & $ \pause #8 $ \\

					\hline

					\pause LCM & $=$ & \pause $ (#6) $ & \pause $ (#7) $ & \pause $ (#8) \pause = #9 $  \\

				\end{tabular}
		\end{minipage}};
\end{tikzpicture}}


\newcommand{\lcmtwobytwolineone}[3]{
	\def \lcmtwobytwolineonenumber {#1}
	\def \lcmtwobytwolineonefactorone {#2}
	\def \lcmtwobytwolineonefactortwo {#3}}

\newcommand{\lcmtwobytwolinetwo}[3]{
	\def \lcmtwobytwolinetwonumber {#1}
	\def \lcmtwobytwolinetwofactorone {#2}
	\def \lcmtwobytwolinetwofactortwo {#3}
%	\lcmtwobytwolineone
}

\newcommand{\lcmtwobytwolinethree}[3]{
	\def \lcmtwobytwolinethreelcm {#3}
	\def \lcmtwobytwolinethreefactorone {#1}
	\def \lcmtwobytwolinethreefactortwo {#2}
%	\lcmtwobytwolinetwo
}

\newcommand{\lcmtwobytwo}[2]{\begin{tikzpicture}[remember picture, overlay]
%		\lcmtwobytwolinethree
		\color{red}
		\node at (current page.south west) (sw) {};
		\node[anchor=south west, inner sep=0pt] at ($(sw) + (#1, #2) $)   (a) {
			\begin{minipage}[t]{0.75\textwidth}
				\pause Find the LCM: 
				
				\begin{tabular}{rccccc}

					$ \pause \lcmtwobytwolineonenumber $ & $=$ & $ \pause \lcmtwobytwolineonefactorone $ & $ \pause \lcmtwobytwolineonefactortwo $ & & \\

					$ \pause \lcmtwobytwolinetwonumber $ & $=$ & $ \pause \lcmtwobytwolinetwofactorone $ & $ \pause \lcmtwobytwolinetwofactortwo $ & & \\

					\hline

					\pause LCM & $=$ & \pause $ (\lcmtwobytwolinethreefactorone) $ & \pause $ (\lcmtwobytwolinethreefactortwo) $ & $=$ & \pause $ \lcmtwobytwolinethreelcm $  \\

				\end{tabular}
		\end{minipage}};
\end{tikzpicture}}


\newcommand{\lcmtwobythreelineone}[4]{
	\def \lcmtwobythreelineonenumber {#1}
	\def \lcmtwobythreelineonefactorone {#2}
	\def \lcmtwobythreelineonefactortwo {#3}
	\def \lcmtwobythreelineonefactorthree {#4}}

\newcommand{\lcmtwobythreelinetwo}[4]{
	\def \lcmtwobythreelinetwonumber {#1}
	\def \lcmtwobythreelinetwofactorone {#2}
	\def \lcmtwobythreelinetwofactortwo {#3}
	\def \lcmtwobythreelinetwofactorthree {#4}
}

\newcommand{\lcmtwobythreelinethree}[4]{
	\def \lcmtwobythreelinethreefactorone {#1}
	\def \lcmtwobythreelinethreefactortwo {#2}
	\def \lcmtwobythreelinethreefactorthree {#3}
	\def \lcmtwobythreelinethreelcm {#4}
}

\newcommand{\lcmtwobythree}[2]{\begin{tikzpicture}[remember picture, overlay]
		\color{red}
		\node at (current page.south west) (sw) {};
		\node[anchor=south west, inner sep=0pt] at ($(sw) + (#1, #2) $)   (a) {
			\begin{minipage}[t]{0.75\textwidth}
				\pause Find the LCM: 
				
				\begin{tabular}{rcccccc}
					
					$ \pause \lcmtwobythreelineonenumber $ & $=$ & $ \pause \lcmtwobythreelineonefactorone $ & $ \pause \lcmtwobythreelineonefactortwo $ & $ \pause \lcmtwobythreelineonefactorthree $ & & \\
					
					$ \pause \lcmtwobythreelinetwonumber $ & $=$ & $ \pause \lcmtwobythreelinetwofactorone $ & $ \pause \lcmtwobythreelinetwofactortwo $ & $ \pause \lcmtwobythreelinetwofactorthree $ & & \\
					
					\hline
					
					\pause LCM & $=$ & \pause $ (\lcmtwobythreelinethreefactorone) $ & \pause $ (\lcmtwobythreelinethreefactortwo) $ & \pause $ (\lcmtwobythreelinethreefactorthree) $ & $=$ & \pause $ \lcmtwobythreelinethreelcm $  \\
					
				\end{tabular}
		\end{minipage}};
\end{tikzpicture}}

\newcommand{\plotsystvars}[8]{
	\def \eqone {#1}
	\def \eqtwo {#2}
	\def \solx {#3}
	\def \soly {#4}
	\def \solanchor {#5}
	\def \labelxshift {#6}
	\def \labelyshift {#7}
	\def \solmarksize {#8}
}

\newcommand{\plotsyst}[7]{
\begin{tikzpicture}[scale=#1]
	
	\begin{axis} 
		[
		xticklabels={}, 
		yticklabels={}, 
		ymin=-#2, ymax=#2,
		xmin=-#2, xmax=#2,
		axis lines = center, 
		inner axis line style={Latex-Latex,very thick}, 
		grid=both,
		minor tick num=#7, 
		tick align=inside,
		after end axis/.code={
			\path (axis cs: \solx,\soly) 
			node [anchor=\solanchor, xshift=\labelxshift pt, yshift=\labelyshift pt] {$ (\solx, \soly) $}; } 
		] 
		
		\addplot[<->, >={Latex[round]},  ultra thick, domain=#3:#4, samples=200]{\eqone}node[]{};
		
		\addplot[<->, >={Latex[round]},  ultra thick, domain=#5:#6, samples=200]{\eqtwo}node[]{};
			
		\pause \addplot[only marks, mark=*, mark size=\solmarksize pt, color=black,] coordinates {(\solx, \soly)};
	\end{axis} 

\end{tikzpicture} 
}
\usetheme{default}
\usecolortheme{seahorse}

\newcolumntype{g}{>{\columncolor{Lgray}\centering\arraybackslash}X}
\newcolumntype{e}{>{\columncolor{Dgray}}X}


\title[] {Problem Solving Involving Rational Algebraic Expressions}
\author{Jonathan R. Bacolod}
\institute[SHS]{Sauyo High School}
\date{}

\begin{document}
	\frame{\titlepage}
	
	\begin{frame}
		\frametitle{What are Rational Equations?}
		Rational equations are equations that contain rational algebraic expressions.
	\end{frame}

    \begin{frame}
    	\frametitle{Examples}
    	\begin{tabular}{ll}
    		$ \dfrac{x}{2y} $ & \pause Rational Expression \\[1em]
    		$ \pause \dfrac{3m^2 - 1}{3} = 5 $ & \pause Rational Equation\\[1em]
    		$ \pause \dfrac{x^2 - 9}{x - 3} $ & \pause Rational Expression \\[1em]
    		$ \pause \dfrac{m^2 - 1}{m + 1} = \dfrac{m^2 - 1}{m - 1} $ & \pause Rational Equation\\[1em]
    	\end{tabular}
    \end{frame}

    \begin{frame}
    	\frametitle{How to Solve Rational Equations?}
    	\begin{enumerate}
    		\item<1-> Find the least common denominator (LCD).
    		\item<2-> Multiply the entire problem by the LCD.
    		\item<3-> Simplify and solve the resulting equation.
    		\item<4-> Check each solution to remove any extraneous solutions.
       	\end{enumerate}
    \end{frame}

    \begin{frame}
    	\frametitle{Example}
    	Solve $ \dfrac{1}{x - 2} + \dfrac{1}{x^2 - 7x + 10} = \dfrac{6}{x - 2} $
    \end{frame}

    \begin{frame}
    	\frametitle{How to Solve Rational Equations?}
    	\begin{enumerate}
    		\item Find the least common denominator (LCD).
    	\end{enumerate}
    \end{frame}

   \begin{frame}
   	\frametitle{Example}
   	$ \dfrac{1}{x - 2} + \dfrac{1}{x^2 - 7x + 10} = \dfrac{6}{x - 2} $
   	
   	\sol{2}{0.7}{\pause Find the LCM: \\
   		\begin{tabular}{rcll}
   			$ \pause x - 2 $ & $=$ & $ \pause x - 2$ & \\
   			$ \pause x^2 - 7x + 10 $ & $=$ & $ \pause (x - 2) $ & $ \pause (x - 5)$\\
   			$ \pause x - 2 $ & $=$ & $ \pause x - 2$ & \\
   			\hline
   			\pause LCM & $=$ & \pause $(x - 2)$ & \pause $ (x - 5) $ \\
   	\end{tabular}}
   \end{frame}

    \begin{frame}
    	\frametitle{How to Solve Rational Equations?}
    	\begin{enumerate}
    		\item Find the least common denominator (LCD).
    		\item Multiply the entire problem by the LCD.
    	\end{enumerate}
    \end{frame}

    \begin{frame}
    	\frametitle{Example}
    	$ \dfrac{1}{x - 2} + \dfrac{1}{x^2 - 7x + 10} = \dfrac{6}{x - 2} $
    \end{frame}

   \begin{frame}
   	\frametitle{Example}
   	$ (x - 2 )(x - 5)\left[\dfrac{1}{x - 2} + \dfrac{1}{x^2 - 7x + 10}\right] = \dfrac{6}{x - 2} $
   \end{frame}

    \begin{frame}
    	\frametitle{Example}
    	$ (x - 2 )(x - 5) \left[\dfrac{1}{x - 2} + \dfrac{1}{x^2 - 7x + 10} \right] = \left[\dfrac{6}{x - 2}\right] (x - 2) ( x - 5 ) $
    \end{frame}

     \begin{frame}
    	\frametitle{Example}
    	$ (x - 2 )(x - 5) \left[\dfrac{1}{x - 2} + \dfrac{1}{x^2 - 7x + 10} \right] = \left[\dfrac{6}{x - 2}\right] (x - 2) ( x - 5 ) $
    	
    	\vspace{1em} $ x - 5 \pause + 1 \pause = 6 (x - 5) $
    \end{frame}

    \begin{frame}
    	\frametitle{How to Solve Rational Equations?}
    	\begin{enumerate}
    		\item Find the least common denominator (LCD).
    		\item Multiply the entire problem by the LCD.
    		\item Simplify and solve the resulting equation.
    	\end{enumerate}
    \end{frame}

    \begin{frame}
    	\frametitle{Example}
    	$ (x - 2 )(x - 5) \left[\dfrac{1}{x - 2} + \dfrac{1}{x^2 - 7x + 10} \right] = \left[\dfrac{6}{x - 2}\right] (x - 2) ( x - 5 ) $
    	
    	\vspace{1em} $ x - 5 + 1 = 6 (x - 5) $
    	
    	\vspace{1em} $ \pause x \pause - 4 = \pause 6x \pause - 30 $
    \end{frame}

    \begin{frame}
    	\frametitle{Example}
    	$ (x - 2 )(x - 5) \left[\dfrac{1}{x - 2} + \dfrac{1}{x^2 - 7x + 10} \right] = \left[\dfrac{6}{x - 2}\right] (x - 2) ( x - 5 ) $
    	
    	\vspace{1em} $ x - 5 + 1 = 6 (x - 5) $
    	
    	\vspace{1em} $  x - 4 =  6x - 30 $
    	
    	\vspace{1em} $  x -6x - 4 =  6x -6x  - 30 $
    \end{frame}

    \begin{frame}
    	\frametitle{Example}
    	$ (x - 2 )(x - 5) \left[\dfrac{1}{x - 2} + \dfrac{1}{x^2 - 7x + 10} \right] = \left[\dfrac{6}{x - 2}\right] (x - 2) ( x - 5 ) $
    	
    	\vspace{1em} $ x - 5 + 1 = 6 (x - 5) $
    	
    	\vspace{1em} $  x - 4 =  6x - 30 $
    	
    	\vspace{1em} $  x -6x - 4 + 4 =  6x -6x  - 30 + 4 $
    	
    	\pause \vspace{1em} $  -5x \pause =  - 26 $
    \end{frame}

    \begin{frame}
    	\frametitle{Example}
    	$ (x - 2 )(x - 5) \left[\dfrac{1}{x - 2} + \dfrac{1}{x^2 - 7x + 10} \right] = \left[\dfrac{6}{x - 2}\right] (x - 2) ( x - 5 ) $
    	
    	\vspace{0.5em} $ x - 5 + 1 = 6 (x - 5) $
    	
    	\vspace{0.5em} $  x - 4 =  6x - 30 $
    	
    	\vspace{0.5em} $  x -6x - 4 + 4 =  6x -6x  - 30 + 4 $
    	
    	\vspace{0.5em} $  \dfrac{-5x}{-5} = \dfrac{ - 26}{-5} $
    	
    	\pause \vspace{0.5em} $ x = \dfrac{ 26}{5} $ 
    \end{frame}

    \begin{frame}
    	\frametitle{How to Solve Rational Equations?}
    	\begin{enumerate}
    		\item Find the least common denominator (LCD).
    		\item Multiply the entire problem by the LCD.
    		\item Simplify and solve the resulting equation.
    		\item Check each solution to remove any extraneous solutions.
    	\end{enumerate}
    \end{frame}

    \begin{frame}
    	\frametitle{Example}
    	$ (x - 2 )(x - 5) \left[\dfrac{1}{x - 2} + \dfrac{1}{x^2 - 7x + 10} \right] = \left[\dfrac{6}{x - 2}\right] (x - 2) ( x - 5 ) $
    	
    	\vspace{0.5em} $ x - 5 + 1 = 6 (x - 5) $
    	
    	\vspace{0.5em} $  x - 4 =  6x - 30 $
    	
    	\vspace{0.5em} $  x -6x - 4 + 4 =  6x -6x  - 30 + 4 $
    	
    	\vspace{0.5em} $  \dfrac{-5x}{-5} = \dfrac{ - 26}{-5} $
    	
    	\vspace{0.5em} $ x = \dfrac{ 26}{5} $ \pause \redcheck
    \end{frame}

    \begin{frame}
    	\frametitle{How to Solve Problems Involving Rational Algebraic Expressions?}
    	\begin{enumerate}
    		\item<1-> Identify the quantity being asked in the problem.
    		\item<2-> Use the facts of the problem to form an equation.
    	    \item<3-> Solve the equation.
    	    \item<4-> Check and interpret the answer.
       	\end{enumerate}
    \end{frame}

    \begin{frame}
    	\frametitle{Sample Problem 1}
    	Shaina can clean their house in 3 hours and Ronnie can do it in 4 hours. How long will it take them cleaning if they will work together?
    \end{frame}

    \begin{frame}
    	\frametitle{How to Solve Problems Involving Rational Algebraic Expressions?}
    	\begin{enumerate}
    		\item Identify the quantity being asked in the problem.
    	\end{enumerate}
    \end{frame}

    \begin{frame}
    	\frametitle{Sample Problem 1}
    	Shaina can clean their house in 3 hours and Ronnie can do it in 4 hours. How long will it take them cleaning if they will work together?    	
    \end{frame}

    \begin{frame}
    	\frametitle{Sample Problem 1}
    	Shaina can clean their house in 3 hours and Ronnie can do it in 4 hours. How long will it take them cleaning if they will work together?    	
    	
    	\vspace{1em}
    	\noindent\begin{tabular}{rlcl}
    		\pause Given: & 3 hours & = & Shaina can finish cleaning \\
    		\pause & 4 hours & = & Ronnie can finish cleaning \\
    		\pause Find: & $ x $ hours & = & Shaina and Ronnie, \\
    		 & & & working together, can finish \\
    		 & & & cleaning the house \\
    	\end{tabular}
    \end{frame}

    \begin{frame}
    	\frametitle{How to Solve Problems Involving Rational Algebraic Expressions?}
    	\begin{enumerate}
    		\item Identify the quantity being asked in the problem.
    		\item Use the facts of the problem to form an equation.
    	\end{enumerate}
    \end{frame}

    \begin{frame}
    	\frametitle{Sample Problem 1}
    	Shaina can clean their house in 3 hours and Ronnie can do it in 4 hours. How long will it take them cleaning if they will work together?    	
    	
    	\vspace{1em}
    	\begin{tabularx}{\linewidth}{  e  e  e  }
    		%\hline
    		\phantom{1 hour} & \phantom{2 hours}  & \phantom{3 hours} \\
    		\phantom{$ \dfrac{1}{3} $} & \phantom{$ \dfrac{2}{3} $}  & \phantom{$ \dfrac{3}{3}  $ or 1} \\
    		%\hline
    	\end{tabularx}
    \end{frame}

    \begin{frame}
    	\frametitle{Sample Problem 1}
    	Shaina can clean their house in 3 hours and Ronnie can do it in 4 hours. How long will it take them cleaning if they will work together?    	
    	
    	\vspace{1em}
    	\begin{tabularx}{\linewidth}{  g e e }
    		 1 hour & \phantom{2 hours}  & \phantom{3 hours} \\
    		 $ \dfrac{1}{3} $ & \phantom{$ \dfrac{2}{3} $}  & \phantom{$ \dfrac{3}{3}  $ or 1} \\
    	\end{tabularx}
    \end{frame}

    \begin{frame}
    	\frametitle{Sample Problem 1}
    	Shaina can clean their house in 3 hours and Ronnie can do it in 4 hours. How long will it take them cleaning if they will work together?    	
    	
    	\vspace{1em}
    	\begin{tabularx}{\linewidth}{  g g e }
    		1 hour & 2 hours  & \phantom{3 hours} \\
    		$ \dfrac{1}{3} $ & $ \dfrac{2}{3} $  & \phantom{$ \dfrac{3}{3}  $ or 1} \\
    	\end{tabularx}
    \end{frame}

    \begin{frame}
    	\frametitle{Sample Problem 1}
    	Shaina can clean their house in 3 hours and Ronnie can do it in 4 hours. How long will it take them cleaning if they will work together?    	
    	
    	\vspace{1em}
    	\begin{tabularx}{\linewidth}{  g g g }
    		1 hour & 2 hours  & 3 hours \\
    		$ \dfrac{1}{3} $ & $ \dfrac{2}{3} $  & $ \dfrac{3}{3}  $ or 1 \\
    	\end{tabularx}
    
    \pause \vspace{1em} Therefore, Shaina can clean the house at the rate of $ \dfrac{1}{3} $ per hour.
    \end{frame}

    \begin{frame}
    	\frametitle{Sample Problem 1}
    	Shaina can clean their house in 3 hours and Ronnie can do it in 4 hours. How long will it take them cleaning if they will work together?    	
    	
    	\vspace{1em}
    	\begin{tabularx}{\linewidth}{  e  e  e  e }
    		\phantom{1 hour} & \phantom{2 hours}  & \phantom{3 hours} & \phantom{4 hours} \\
    		\phantom{$ \dfrac{1}{4} $} & \phantom{$ \dfrac{2}{4} $ or $ \dfrac{1}{2} $}  & \phantom{$ \dfrac{3}{4}$} & \phantom{$ \dfrac{4}{4}  $ or 1 } \\    
    	\end{tabularx}
    \end{frame}

    \begin{frame}
    	\frametitle{Sample Problem 1}
    	Shaina can clean their house in 3 hours and Ronnie can do it in 4 hours. How long will it take them cleaning if they will work together?    	
    	
    	\vspace{1em}
    	\begin{tabularx}{\linewidth}{ g  e  e  e }
    		1 hour & \phantom{2 hours}  & \phantom{3 hours} & \phantom{4 hours} \\
    		$ \dfrac{1}{4} $ & \phantom{$ \dfrac{2}{4} $ or $ \dfrac{1}{2} $}  & \phantom{$ \dfrac{3}{4}$} & \phantom{$ \dfrac{4}{4}  $ or 1 } \\    
    	\end{tabularx}
    \end{frame}

    \begin{frame}
    	\frametitle{Sample Problem 1}
    	Shaina can clean their house in 3 hours and Ronnie can do it in 4 hours. How long will it take them cleaning if they will work together?    	
    	
    	\vspace{1em}
    	\begin{tabularx}{\linewidth}{ g g  e  e }
    		1 hour & 2 hours  & \phantom{3 hours} & \phantom{4 hours} \\
    		$ \dfrac{1}{4} $ & $ \dfrac{2}{4} $ or $ \dfrac{1}{2} $  & \phantom{$ \dfrac{3}{4}$} & \phantom{$ \dfrac{4}{4}  $ or 1 } \\    
    	\end{tabularx}
    \end{frame}

    \begin{frame}
    	\frametitle{Sample Problem 1}
    	Shaina can clean their house in 3 hours and Ronnie can do it in 4 hours. How long will it take them cleaning if they will work together?    	
    	
    	\vspace{1em}
    	\begin{tabularx}{\linewidth}{ g g g e }
    		1 hour & 2 hours  & 3 hours & \phantom{4 hours} \\
    		$ \dfrac{1}{4} $ & $ \dfrac{2}{4} $ or $ \dfrac{1}{2} $  & $ \dfrac{3}{4}$ & \phantom{$ \dfrac{4}{4}  $ or 1 } \\    
    	\end{tabularx}
    \end{frame}

    \begin{frame}
    	\frametitle{Sample Problem 1}
    	Shaina can clean their house in 3 hours and Ronnie can do it in 4 hours. How long will it take them cleaning if they will work together?    	
    	
    	\vspace{1em}
    	\begin{tabularx}{\linewidth}{ g g g g }
    		1 hour & 2 hours  & 3 hours & 4 hours \\
    		$ \dfrac{1}{4} $ & $ \dfrac{2}{4} $ or $ \dfrac{1}{2} $  & $ \dfrac{3}{4}$ & $ \dfrac{4}{4}  $ or 1 \\    
    	\end{tabularx}
    
     \pause \vspace{1em} Therefore, Ronnie can clean the house at the rate of $ \dfrac{1}{4} $ per hour.
    \end{frame}

    \begin{frame}
    	\frametitle{Sample Problem 1}
    	Facts of the problem:
    	\begin{enumerate}
    		\item<1->  $ \dfrac{1}{3} $ per hour: Shaina's rate
    		\item<2->  $ \dfrac{1}{4} $ per hour: Ronnie's rate
    		\item<3-> Together, they can clean in $ x $ hours, then in one hour they can clean $ \dfrac{1}{x} $ of the house.
    		\item<4-> $ \dfrac{1}{3} + \dfrac{1}{4} = $ their combined rate per hour
    		\item<5->[Then,] $ \dfrac{1}{3} + \dfrac{1}{4} = \dfrac{1}{x}$
    	\end{enumerate}
    \end{frame}

    \begin{frame}
    	\frametitle{How to Solve Problems Involving Rational Algebraic Expressions?}
    	\begin{enumerate}
    		\item Identify the quantity being asked in the problem.
    		\item Use the facts of the problem to form an equation.
    		\item Solve the equation.
    	\end{enumerate}
    \end{frame}

    \begin{frame}
    	\frametitle{Sample Problem 1}
    	Solve $ \dfrac{1}{3} + \dfrac{1}{4} = \dfrac{1}{x}$
    \end{frame}

   \begin{frame}
   	\frametitle{How to Solve Rational Equations?}
   	\begin{enumerate}
   		\item Find the least common denominator (LCD).
   	\end{enumerate}
   \end{frame}

    \begin{frame}
    	\frametitle{Sample Problem 1}
    	$ \dfrac{1}{3} + \dfrac{1}{4} = \dfrac{1}{x}$
    	
    	\sol{1}{1.4}{\pause Find the LCM: \\
    		\begin{tabular}{rclll}
    			$ \pause 3 $ & $=$ &  &$ \pause 3 $& \\
    			$ \pause 4 $ & $=$ & $ \pause 2^2$ &  & \\
    			$ \pause x  $ & $=$ &  & & $ \pause x $ \\
    			\hline
    			\pause LCM & $=$ & \pause $ (2^2)$ & \pause $ (3) $ & \pause $ (x) \pause = 12x $  \\
    	\end{tabular}}
    \end{frame}
    
   \begin{frame}
   	\frametitle{How to Solve Rational Equations?}
   	\begin{enumerate}
   		\item Find the least common denominator (LCD).
   		\item Multiply the entire problem by the LCD.
   	\end{enumerate}
   \end{frame}

    \begin{frame}
    	\frametitle{Sample Problem 1}
    	$ \dfrac{1}{3} + \dfrac{1}{4} = \dfrac{1}{x}$
    \end{frame}

    \begin{frame}
    	\frametitle{Sample Problem 1}
    	$ (12x)\left[\dfrac{1}{3} + \dfrac{1}{4} \right] = \dfrac{1}{x} $
    \end{frame}

    \begin{frame}
    	\frametitle{Sample Problem 1}
    	$ (12x)\left[\dfrac{1}{3} + \dfrac{1}{4} \right] = \left[\dfrac{1}{x} \right](12x)$
    	
    	\pause \vspace{1em} $ 4x \pause + 3x \pause = 12 $
    \end{frame}

    \begin{frame}
    	\frametitle{How to Solve Rational Equations?}
    	\begin{enumerate}
    		\item Find the least common denominator (LCD).
    		\item Multiply the entire problem by the LCD.
    		\item Simplify and solve the resulting equation.
    	\end{enumerate}
    \end{frame}

    \begin{frame}
    	\frametitle{Sample Problem 1}
    	$ (12x)\left[\dfrac{1}{3} + \dfrac{1}{4} \right] = \left[\dfrac{1}{x} \right](12x)$
    	
    	 \vspace{1em} $ 4x  + 3x  = 12 $
    	 
    	 \pause \vspace{1em} $ 7x = 12 $
    \end{frame}

    \begin{frame}
    	\frametitle{Sample Problem 1}
    	$ (12x)\left[\dfrac{1}{3} + \dfrac{1}{4} \right] = \left[\dfrac{1}{x} \right](12x)$
    	
    	\vspace{1em} $ 4x  + 3x  = 12 $
    	
    	\vspace{1em} $ \dfrac{7x}{7} = \dfrac{12}{7} $
    	
    	\pause $ x = \dfrac{12}{7} $
    \end{frame}

    \begin{frame}
    	\frametitle{How to Solve Rational Equations?}
    	\begin{enumerate}
    		\item Find the least common denominator (LCD).
    		\item Multiply the entire problem by the LCD.
    		\item Simplify and solve the resulting equation.
    		\item Check each solution to remove any extraneous solutions.
    	\end{enumerate}
    \end{frame}

    \begin{frame}
    	\frametitle{Sample Problem 1}
    	$ (12x)\left[\dfrac{1}{3} + \dfrac{1}{4} \right] = \left[\dfrac{1}{x} \right](12x)$
    	
    	\vspace{1em} $ 4x  + 3x  = 12 $
    	
    	\vspace{1em} $ \dfrac{7x}{7} = \dfrac{12}{7} $
    	
    	$ x = \dfrac{12}{7} $
    \end{frame}

    \begin{frame}
    	\frametitle{Sample Problem 1}
    	$ \dfrac{1}{3} + \dfrac{1}{4} = \dfrac{1}{x}$
    	
    	\pause \vspace{1em} $ \dfrac{1}{3} + \dfrac{1}{4} = \dfrac{1}{\frac{12}{7}}$
    	
    	\pause \vspace{1em} $ \dfrac{1}{3} + \dfrac{1}{4} = \dfrac{7}{12} $
    	
    	\pause \vspace{1em} $ 12\left[\dfrac{1}{3} + \dfrac{1}{4} \right] \pause = \left[\dfrac{7}{12} \right]12 $
    	
    	\pause \vspace{1em} $ 4 \pause + 3 \pause = 7 $
    	
    	\pause \vspace{1em} $ 7 = 7 $ \pause \redcheck
    	
    	\pause \sol{4}{5}{Therefore, it will take Shaina and Ronnie $ \dfrac{12}{7} $ or $ 1\dfrac{5}{7} $ hours to finish cleaning the house.}
    \end{frame}

    \begin{frame}
    	\frametitle{Sample Problem 2}
    	The ratio of an angle to its complement is $ \dfrac{2}{3} $. Find the angle.
    \end{frame}

   \begin{frame}
   	\frametitle{How to Solve Problems Involving Rational Algebraic Expressions?}
   	\begin{enumerate}
   		\item Identify the quantity being asked in the problem. 
   	\end{enumerate}
   \end{frame}

   \begin{frame}
   	\frametitle{Sample Problem 2}
   	The ratio of an angle to its complement is $ \dfrac{2}{3} $. Find the angle.
   	
   	\pause \vone  \begin{center}
   		\begin{tikzpicture}
   		\coordinate (a) at (5, 0);
   		\coordinate (b) at (0, 0);
   		\coordinate (c) at (0, 5);
   		\coordinate (d) at (4, 3);
   		\draw [line width=0.3mm, ->, >={Latex[round]}] (b) -- (a);
   		\draw [line width=0.3mm, ->, >={Latex[round]}] (b) -- (c);
   		\draw (0,.4) -| (.4,0);
   		
   		\pause \draw [line width=0.3mm, ->, >={Latex[round]}] (b) -- (d);
   		   		
   		\pic [draw, line width=0.3mm, angle radius=1cm] {angle=a--b--d};
   		\node(x-label) at ($(b)+(20:1.5cm)$) {$ x $};
   		
   		\pause \pic [draw, line width=0.3mm, angle radius=1cm] {angle=d--b--c};
   		\pause \node(comple-label) at ($(b)+(60:1.7cm)$) {$ 90 - x $};
   	\end{tikzpicture}
   	\end{center}
   \end{frame}

    \begin{frame}
    	\frametitle{Sample Problem 2}
    	The ratio of an angle to its complement is $ \dfrac{2}{3} $. Find the angle.
    	
    	\vone
    	\begin{tabular}{rrcl}
    		Let: & $ x $ & = & the angle \\
    		\pause	& $ 90 - x $ & = & the complement of $ x $ \\
    		\pause Find: & $ x $ & & \\
    	\end{tabular}
        \end{frame}

    \begin{frame}
    	\frametitle{How to Solve Problems Involving Rational Algebraic Expressions?}
    	\begin{enumerate}
    		\item Identify the quantity being asked in the problem.
    		\item Use the facts of the problem to form an equation.
    	\end{enumerate}
    \end{frame}

    \begin{frame}
    	\frametitle{Sample Problem 2}
    	The ratio of an angle to its complement is $ \dfrac{2}{3} $. Find the angle.
    	
    	\vone
    	\pause Facts of the problem:
    	\begin{enumerate}
    		\item<3-> The angle is $ x $.
    		\item<4-> The complement is $ 90 - x $.
    		\item<5-> The ratio of $ x $ to $ 90 - x $ is $ \dfrac{2}{3} $.
    		\item<6-> [Then,] $ \dfrac{x}{90 - x} = \dfrac{2}{3} $.
    	\end{enumerate}
    \end{frame}

    \begin{frame}
    	\frametitle{How to Solve Problems Involving Rational Algebraic Expressions?}
    	\begin{enumerate}
    		\item Identify the quantity being asked in the problem.
    		\item Use the facts of the problem to form an equation.
    		\item Solve the equation.
    	\end{enumerate}
    \end{frame}

    \begin{frame}
    	\frametitle{Sample Problem 2}
    	Solve $ \dfrac{x}{90 - x} = \dfrac{2}{3} $.
    \end{frame}

   \begin{frame}
   	\frametitle{How to Solve Rational Equations?}
   	\begin{enumerate}
   		\item Find the least common denominator (LCD).
   	\end{enumerate}
   \end{frame}

    \begin{frame}
    	\frametitle{Sample Problem 2}
    	$ \dfrac{x}{90 - x} = \dfrac{2}{3} $
    	
    	\sol{2}{1}{\pause Find the LCM: \\
    		\begin{tabular}{rcccc}
    			$ \pause 90 - x $ & $=$ & $ \pause 90 - x $ & & \\
    			$ \pause 3 $ & $=$ & & $ \pause 3 $ & \\
    			\hline
    			\pause LCM & $=$ & \pause $(90 - x)$ & \pause $ (3) $ & $  \pause = 3(90-x)$ \\
    	\end{tabular}}
    \end{frame}

    \begin{frame}
    	\frametitle{How to Solve Rational Equations?}
    	\begin{enumerate}
    		\item Find the least common denominator (LCD).
    		\item Multiply the entire problem by the LCD.
    	\end{enumerate}
    \end{frame}

    \begin{frame}
    	\frametitle{Sample Problem 2}
    	$ \dfrac{x}{90 - x} = \dfrac{2}{3} $
    	
    	\pause \vone $ 3(90-x)\left[ \dfrac{x}{90 - x} \right] = \pause \left[\dfrac{2}{3} \right] 3(90-x) $
    	
    	\pause \vone $ 3x \pause = 2(90 - x) $
    \end{frame}

    \begin{frame}
    	\frametitle{How to Solve Rational Equations?}
    	\begin{enumerate}
    		\item Find the least common denominator (LCD).
    		\item Multiply the entire problem by the LCD.
    		\item Simplify and solve the resulting equation.
    	\end{enumerate}
    \end{frame}

    \begin{frame}
    	\frametitle{Sample Problem 2}
    	$ \dfrac{x}{90 - x} = \dfrac{2}{3} $
    	
    	\vone $ 3(90-x)\left[ \dfrac{x}{90 - x} \right] =  \left[\dfrac{2}{3} \right] 3(90-x) $
    	
    	\vspace{0.5em} $ 3x  = 2(90 - x) $
    	
    	\pause \vspace{0.5em} $ 3x  \pause = 180 \pause - 2x $
    	
    	\pause \vspace{0.5em} $ 3x + 2x  \pause = 180 $
    	
    	\pause \vone $ 5x  \pause = 180 $
    \end{frame}

    \begin{frame}
    	\frametitle{Sample Problem 2}
    	$ \dfrac{x}{90 - x} = \dfrac{2}{3} $
    	
    	\vone $ 3(90-x)\left[ \dfrac{x}{90 - x} \right] =  \left[\dfrac{2}{3} \right] 3(90-x) $
    	
    	\vspace{0.5em} $ 3x  = 2(90 - x) $
    	
    	 \vspace{0.5em} $ 3x   = 180 - 2x $
    	
    	 \vspace{0.5em} $ 3x + 2x   = 180 $
    	
    	 \vspace{0.5em} $ \dfrac{5x}{5}   = \dfrac{180}{5} $
    	 
    	 \pause \vone $ x = \pause 36 $
    \end{frame}

    \begin{frame}
    	\frametitle{How to Solve Problems Involving Rational Algebraic Expressions?}
    	\begin{enumerate}
    		\item Identify the quantity being asked in the problem.
    		\item Use the facts of the problem to form an equation.
    		\item Solve the equation.
    		\item Check and interpret the answer.
    	\end{enumerate}
    \end{frame}

    \begin{frame}
    	\frametitle{Sample Problem 2}
    	Checking: Substitute $ 36 $ for $ x $.
    	
    	\vone
    	$ \dfrac{x}{90 - x} = \dfrac{2}{3} $
    	
        \pause \vhalf $ \dfrac{36}{90 - 36} = \dfrac{2}{3} $

        \pause \vhalf $ \dfrac{36}{54} = \dfrac{2}{3} $
            	
    	\pause \vhalf $ \dfrac{2}{3} = \dfrac{2}{3} $ \pause \redcheck
    	
    	\pause \vone Therefore, the angle measures $ 36 \degree $.
    \end{frame}

    \begin{frame}
    	\begin{center}
    		\textbf{\LARGE Thank you for watching.}
    	\end{center}
    \end{frame}
	
\end{document}