\documentclass[14pt]{beamer}
\usepackage[utf8]{inputenc}
\usepackage{bookman} %font style
\usepackage{multicol}
\usepackage{cancel}
\usepackage{tikz}
\usepackage{xcolor} % for font color
\usetikzlibrary{calc}
\usetikzlibrary{positioning}
\usetikzlibrary{arrows.meta}
\usetikzlibrary{angles,quotes}
%\usetikzlibrary{decorations.pathreplacing}
\usepackage{wasysym} %for checked symbol 
\usepackage{tabularx} 
\usepackage{color, colortbl} %coloring table cells
\usepackage{gensymb} %degree symbol
\usepackage{amsfonts} %integer symbol
\usepackage{pgfplots} %graphs
\usepackage{easytable} % TAB
\usepackage{siunitx} % celsius symbol
\usepackage{stackengine} %to define \pesos 
\usepackage{systeme} % system of equations
\usepackage{pifont} % cross mark

\newcommand\pesos{\stackengine{-1.1ex}{P}{\stackengine{-1ex}{$-$}{$-$}{O}{c}{F}{F}{S}}{O}{c}{F}{T}{S}} 

\definecolor{Lgray}{gray}{0.9}
\definecolor{Dgray}{gray}{0.6}

\newcolumntype{Y}{>{\centering\arraybackslash}X} %for tabularx

\makeatletter
\newlength\beamerleftmargin
\setlength\beamerleftmargin{\Gm@lmargin}
\makeatother

\newcommand{\void}{\text{\hspace{2em}}}

\newcommand{\minivoid}{\text{\hspace{1em}}}

\newcommand{\sol}[3]{\begin{tikzpicture}[remember picture, overlay]
		\color{red}
        \node at (current page.south west) (sw) {};
		\node[anchor=south west, inner sep=0pt] at ($(sw) + (#1, #2) $)   (a) {
			\begin{minipage}[t]{0.75\textwidth}
				#3
			\end{minipage}};
    \end{tikzpicture}}


\newcommand \redcheck {{\color{red}\checkmark}}

\newcommand \redcross {{\color{red}\ding{56}}}

\newcommand{\vone}{\vspace{1em}}

\newcommand{\vhalf}{\vspace{0.5em}}

\newcommand{\mygrid}[2]{\begin{center}
		\begin{tikzpicture}[scale=#1]		
			\draw [help lines] (-10, -10) grid (10, 10);
			\draw[line width=0.5mm, <->, >={Latex[round]}] (-10, 0) -- (10, 0);
			\draw[line width=0.5mm, <->, >={Latex[round]}] (0, -10) -- (0, 10);	
			\input{#2}
		\end{tikzpicture}  
	\end{center}} 

\newcommand{\arrowcomment}[9]{\begin{tikzpicture}[remember picture, overlay]
		\node at (current page.south west) (sw) {};
		\node[anchor=south west, inner sep=0pt] at ($(sw) + (0, 0) $)   (a) {
			\begin{tikzpicture}[->,>=stealth, thick, main node/.style={rectangle,font=\sffamily\bfseries}, remember picture, overlay]

				\node (1)  at (#1, #2) {};

				\node[main node] (2)  at (#3, #4) {#5};

				\draw [->, red] (1.#6) to [out=#8,in=#9] (2.#7);

			\end{tikzpicture}
		};
\end{tikzpicture}} 

\newcommand{\plotit}[3]{\begin{center}
		\begin{tikzpicture}[scale=#2, main node/.style={rectangle,font=\sffamily\bfseries}]		
			\draw [help lines] (-#3, -#3) grid (#3, #3);
			\draw[line width=0.5mm, <->, >={Latex[round]}] (-#3, 0) -- (#3, 0);
			\draw[line width=0.5mm, <->, >={Latex[round]}] (0, -#3) -- (0, #3);	
			\input{#1}
		\end{tikzpicture}  
\end{center}} 

\newcommand{\plotpoint}[7]{\begin{center}
		\begin{tikzpicture}[scale=#7]		
			\coordinate (a) at (#1, #2);
			\draw [help lines] (-#5, -#5) grid (#5, #5);
			\draw[line width=#6 mm, <->, >={Latex[round]}] (-#5, 0) -- (#5, 0);
			\draw[line width=#6 mm, <->, >={Latex[round]}] (0, -#5) -- (0, #5);	
			\fill [fill=black] (a) circle (#3 pt);
			\node[anchor=#4, inner sep=2pt, rotate=0] (a-label) at (a) {$(#1, #2)$};
			\end{tikzpicture}  
	\end{center}} 

\newcommand{\plotoverlay}[5]{\begin{tikzpicture}[remember picture, overlay]
		\node at (current page.south west) (sw) {};
		\node[anchor=south west, inner sep=0pt] at ($(sw) + (#1, #2) $)   (a) {
			\begin{minipage}[t]{0.75\textwidth}
				 \begin{tikzpicture}[scale=#4, remember picture, overlay]		

						\draw [help lines] (-#5, -#5) grid (#5, #5);

						\draw[line width=0.5mm, <->, >={Latex[round]}] (-#5, 0) -- (#5, 0);

						\draw[line width=0.5mm, <->, >={Latex[round]}] (0, -#5) -- (0, #5);	

						\input{#3}

				\end{tikzpicture}
		\end{minipage}};
\end{tikzpicture}}

\newcommand{\lcmthreebythree}[9]{\begin{tikzpicture}[remember picture, overlay]
		\color{red}
		\node at (current page.south west) (sw) {};
		\node[anchor=south west, inner sep=0pt] at ($(sw) + (#1, #2) $)   (a) {
			\begin{minipage}[t]{0.75\textwidth}
				\pause Find the LCM: \\

				\begin{tabular}{rclll}

					$ \pause #3 $ & $=$ & $ \pause #6 $ & & \\

					$ \pause #4 $ & $=$ &  & $ \pause #7$ & \\

					$ \pause #5  $ & $=$ &  & & $ \pause #8 $ \\

					\hline

					\pause LCM & $=$ & \pause $ (#6) $ & \pause $ (#7) $ & \pause $ (#8) \pause = #9 $  \\

				\end{tabular}
		\end{minipage}};
\end{tikzpicture}}


\newcommand{\lcmtwobytwolineone}[3]{
	\def \lcmtwobytwolineonenumber {#1}
	\def \lcmtwobytwolineonefactorone {#2}
	\def \lcmtwobytwolineonefactortwo {#3}}

\newcommand{\lcmtwobytwolinetwo}[3]{
	\def \lcmtwobytwolinetwonumber {#1}
	\def \lcmtwobytwolinetwofactorone {#2}
	\def \lcmtwobytwolinetwofactortwo {#3}
%	\lcmtwobytwolineone
}

\newcommand{\lcmtwobytwolinethree}[3]{
	\def \lcmtwobytwolinethreelcm {#3}
	\def \lcmtwobytwolinethreefactorone {#1}
	\def \lcmtwobytwolinethreefactortwo {#2}
%	\lcmtwobytwolinetwo
}

\newcommand{\lcmtwobytwo}[2]{\begin{tikzpicture}[remember picture, overlay]
%		\lcmtwobytwolinethree
		\color{red}
		\node at (current page.south west) (sw) {};
		\node[anchor=south west, inner sep=0pt] at ($(sw) + (#1, #2) $)   (a) {
			\begin{minipage}[t]{0.75\textwidth}
				\pause Find the LCM: 
				
				\begin{tabular}{rccccc}

					$ \pause \lcmtwobytwolineonenumber $ & $=$ & $ \pause \lcmtwobytwolineonefactorone $ & $ \pause \lcmtwobytwolineonefactortwo $ & & \\

					$ \pause \lcmtwobytwolinetwonumber $ & $=$ & $ \pause \lcmtwobytwolinetwofactorone $ & $ \pause \lcmtwobytwolinetwofactortwo $ & & \\

					\hline

					\pause LCM & $=$ & \pause $ (\lcmtwobytwolinethreefactorone) $ & \pause $ (\lcmtwobytwolinethreefactortwo) $ & $=$ & \pause $ \lcmtwobytwolinethreelcm $  \\

				\end{tabular}
		\end{minipage}};
\end{tikzpicture}}


\newcommand{\lcmtwobythreelineone}[4]{
	\def \lcmtwobythreelineonenumber {#1}
	\def \lcmtwobythreelineonefactorone {#2}
	\def \lcmtwobythreelineonefactortwo {#3}
	\def \lcmtwobythreelineonefactorthree {#4}}

\newcommand{\lcmtwobythreelinetwo}[4]{
	\def \lcmtwobythreelinetwonumber {#1}
	\def \lcmtwobythreelinetwofactorone {#2}
	\def \lcmtwobythreelinetwofactortwo {#3}
	\def \lcmtwobythreelinetwofactorthree {#4}
}

\newcommand{\lcmtwobythreelinethree}[4]{
	\def \lcmtwobythreelinethreefactorone {#1}
	\def \lcmtwobythreelinethreefactortwo {#2}
	\def \lcmtwobythreelinethreefactorthree {#3}
	\def \lcmtwobythreelinethreelcm {#4}
}

\newcommand{\lcmtwobythree}[2]{\begin{tikzpicture}[remember picture, overlay]
		\color{red}
		\node at (current page.south west) (sw) {};
		\node[anchor=south west, inner sep=0pt] at ($(sw) + (#1, #2) $)   (a) {
			\begin{minipage}[t]{0.75\textwidth}
				\pause Find the LCM: 
				
				\begin{tabular}{rcccccc}
					
					$ \pause \lcmtwobythreelineonenumber $ & $=$ & $ \pause \lcmtwobythreelineonefactorone $ & $ \pause \lcmtwobythreelineonefactortwo $ & $ \pause \lcmtwobythreelineonefactorthree $ & & \\
					
					$ \pause \lcmtwobythreelinetwonumber $ & $=$ & $ \pause \lcmtwobythreelinetwofactorone $ & $ \pause \lcmtwobythreelinetwofactortwo $ & $ \pause \lcmtwobythreelinetwofactorthree $ & & \\
					
					\hline
					
					\pause LCM & $=$ & \pause $ (\lcmtwobythreelinethreefactorone) $ & \pause $ (\lcmtwobythreelinethreefactortwo) $ & \pause $ (\lcmtwobythreelinethreefactorthree) $ & $=$ & \pause $ \lcmtwobythreelinethreelcm $  \\
					
				\end{tabular}
		\end{minipage}};
\end{tikzpicture}}

\newcommand{\plotsystvars}[8]{
	\def \eqone {#1}
	\def \eqtwo {#2}
	\def \solx {#3}
	\def \soly {#4}
	\def \solanchor {#5}
	\def \labelxshift {#6}
	\def \labelyshift {#7}
	\def \solmarksize {#8}
}

\newcommand{\plotsyst}[7]{
\begin{tikzpicture}[scale=#1]
	
	\begin{axis} 
		[
		xticklabels={}, 
		yticklabels={}, 
		ymin=-#2, ymax=#2,
		xmin=-#2, xmax=#2,
		axis lines = center, 
		inner axis line style={Latex-Latex,very thick}, 
		grid=both,
		minor tick num=#7, 
		tick align=inside,
		after end axis/.code={
			\path (axis cs: \solx,\soly) 
			node [anchor=\solanchor, xshift=\labelxshift pt, yshift=\labelyshift pt] {$ (\solx, \soly) $}; } 
		] 
		
		\addplot[<->, >={Latex[round]},  ultra thick, domain=#3:#4, samples=200]{\eqone}node[]{};
		
		\addplot[<->, >={Latex[round]},  ultra thick, domain=#5:#6, samples=200]{\eqtwo}node[]{};
			
		\pause \addplot[only marks, mark=*, mark size=\solmarksize pt, color=black,] coordinates {(\solx, \soly)};
	\end{axis} 

\end{tikzpicture} 
}
\usetheme{default}
\usecolortheme{seahorse}

\title[] {Finding the Equation of a Line Given the Slope and a Point or Two Points}
\author{Jonathan R. Bacolod}
\institute[SHS]{Sauyo High School}
\date{}

\begin{document}
	\frame{\titlepage}
	
	\begin{frame}
		\frametitle{How to Find the Equation of a Line?}
		Use the formulae:
		\begin{enumerate}  
			\item Slope-intercept form: $ y = mx + b $
			\item \pause Point-slope form: $ y - y_1 = m(x - x_1) $
			\item \pause Two-point form: $  \dfrac{y - y_1}{y_2 - y_1} = \dfrac{x - x_1}{x_2 - x_1}$
		\end{enumerate}  
	\end{frame}

    \begin{frame}
    	\frametitle{How to Find the Equation of a Line Given the Slope and a Point?}
    	Use the point-slope form:  $ y - y_1 = m(x - x_1) $
    \end{frame}

    \begin{frame}
    	\frametitle{Example 1}
    	Find the equation of the line with a slope equal to $ 2 $ and passes through point $ (1, 3) $.
    	    	
    	\begin{TAB}(@, 1mm, 5mm)[2mm]{ll}{ccccccc}
    		\pause Given: $ m = 2 $,  & \pause $ (x_1, y_1) = (1, 3) $  \\
    		
    		\pause $ y - y_1 = m(x - x_1)  $  &  \\
    		
    		\pause $ y - 3 = 2(x - 1)  $ & \pause Substitution \\
    		
    		\pause $ y - 3 = 2x - 2  $ & \pause Distributive Property \\
    		
    		\pause $ y = 2x - 2 + 3 $ & \pause Addition Property \\
    		
    		\pause $ y = 2x + 1 $ & \pause Simplification \\
    		& \\
    	\end{TAB}
  
      $ \therefore $ the equation of the line is $ y = 2x + 1 $.
    \end{frame}

    \begin{frame}
    	\frametitle{Example 2}
    	\footnotesize Find the equation of the line with a slope equal to $ \dfrac{2}{3} $ and contains the point $ (9, 7) $.
    	
    	\begin{TAB}(@, 1mm, 5mm)[2mm]{ll}{cccccc}
    		\pause Given: $ m = \dfrac{2}{3} $,  & \pause $ (x_1, y_1) = (9, 7) $  \\
    		
    		\pause $ y - y_1 = m(x - x_1)  $  &  \\
    		
    		\pause $ y - 7 = \dfrac{2}{3}(x - 9)  $ & \pause Substitution \\
    		
    		\pause $ y - 7 = \dfrac{2}{3}x - 6  $ & \pause Distributive Property \\
    		
    		\pause $ y = \dfrac{2}{3}x - 6 + 7 $ & \pause Addition Property \\
    		
    		\pause $ y = \dfrac{2}{3}x + 1 $ &  Simplification \\
%    		& \\
    	\end{TAB}
    	
    	$ \therefore $ the equation of the line is $ y = \dfrac{2}{3}x + 1 $.
    \end{frame}

    \begin{frame}
    	\frametitle{Example 3}
    	Find the equation of the line parallel to $ y = 3x - 2 $ and passes through point $ (1, -1) $.
    	
    	\begin{TAB}(@, 1mm, 5mm)[2mm]{ll}{ccccccc}
    		\pause Given: $ m = 3 $,  & \pause $ (x_1, y_1) = (1, -1) $  \\
    		
    		\pause $ y - y_1 = m(x - x_1)  $  &  \\
    		
    		\pause $ y - (-1) = 3(x - 1)  $ & \pause Substitution \\
    		
    		\pause $ y + 1 = 3x - 3  $ & \pause Distributive Property \\
    		
    		\pause $ y = 3x - 3 - 1 $ & \pause Subtraction Property \\
    		
    		\pause $ y = 3x - 4 $ & \pause Simplification \\
    		& \\
    	\end{TAB}
    	
    	$ \therefore $ the equation of the line is $ y = 3x - 4 $.
    \end{frame}

    \begin{frame}
    	\frametitle{How to Find the Equation of a Line Given Two Points?}
    	Use the two-point form: $  \dfrac{y - y_1}{y_2 - y_1} = \dfrac{x - x_1}{x_2 - x_1}$
    \end{frame}

    \begin{frame}
    	\frametitle{Example 1}
    	Write the equation of the line containing $ (1, 1) $ and $ (3, -1) $. 
    	
    \lcmtwobytwolineone{-2}{(2)}{(-1)}
    \lcmtwobytwolinetwo{2}{(2)}{}
    \lcmtwobytwolinethree{2}{-1}{-2}
    
       	\begin{TAB}(@, 1mm, 5mm)[2mm]{ll}{cccccc}
    		\pause Given: $ (x_1, y_1) = (1, 1)  $,  & \pause $ (x_2, y_2) = (3, -1) $  \\
    		
    		\pause $ \dfrac{y - y_1}{y_2 - y_1} = \dfrac{x - x_1}{x_2 - x_1}  $  &  \\
    		
    		\pause $ \dfrac{y - 1}{-1 - 1} = \dfrac{x - 1}{3 - 1}  $ & \pause Substitution \\
    		
    		\pause $ \dfrac{y - 1}{-2} = \dfrac{x - 1}{2} $ & \pause Simplification \\
    		
    	    \only<7-19>{\footnotesize \lcmtwobytwo{1}{0.1} \normalsize} \only<20->{$ -2 \left[\dfrac{y - 1}{-2} \right] = -2\left[\dfrac{x - 1}{2} \right] $} & \only<21->{ Multiplication} \\
    		& \\
    	\end{TAB}
    \end{frame}

    \begin{frame}
    	\frametitle{Example 1}
    	\begin{TAB}(@, 1mm, 5mm)[2mm]{ll}{ccccc}
    	  $ -2 \left[\dfrac{y - 1}{-2} \right] = -2\left[\dfrac{x - 1}{2} \right] $ &  Multiplication \\
    		
    		\pause $ y - 1 = -x + 1 $ &\pause Distributive Property \\
    		
    		\pause $ y = -x + 1 + 1 $ &\pause Addition Property \\
    		
    		\pause $ y = -x + 2 $ & \pause Simplification \\
    		& \\
    	\end{TAB}
    	
    	$ \therefore $ the equation of the line is $ y = -x + 2 $.
    \end{frame}
    
    \begin{frame}
    	\frametitle{Example 2}
    	Write the equation of the line that passes through $ (2, 4) $ and whose x-intercept is $ 1 $. 
    	
    	\lcmtwobytwolineone{-1}{(-1)}{}
    	\lcmtwobytwolinetwo{-4}{(-1)}{(2^2)}
    	\lcmtwobytwolinethree{-1}{2^2}{-4}
    	
    	\begin{TAB}(@, 1mm, 5mm)[2mm]{ll}{cccccc}
    		\pause Given: $ (x_1, y_1) = (2, 4)  $,  & \pause $ (x_2, y_2) = (1, 0) $  \\
    		
    		\pause $ \dfrac{y - y_1}{y_2 - y_1} = \dfrac{x - x_1}{x_2 - x_1}  $  &  \\
    		
    		\pause $ \dfrac{y - 4}{0 - 4} = \dfrac{x - 2}{1 - 2}  $ & \pause Substitution \\
    		
    		\pause $ \dfrac{y - 4}{-4} = \dfrac{x - 2}{-1} $ & \pause Simplification \\
    		
    		 \only<7-19>{\footnotesize \lcmtwobytwo{1}{0.1} \normalsize} \only<20->{$ -4 \left[\dfrac{y - 4}{-4} \right] = -4\left[\dfrac{x - 2}{-1} \right] $} & \only<21->{ Multiplication} \\
    		    		& \\
    	\end{TAB}
    \end{frame}

    \begin{frame}
	\frametitle{Example 2}
	
	\begin{TAB}(@, 1mm, 5mm)[2mm]{ll}{ccccc}
		$ -4 \left[\dfrac{y - 4}{-4} \right] = -4\left[\dfrac{x - 2}{-1} \right] $ &  Multiplication \\
		
		\pause $ y - 4 = 4x - 8 $ & \pause Distributive Property \\
		
		\pause $ y = 4x - 8 + 4 $ & \pause Addition Property \\
		
		\pause $ y = 4x - 4 $ & Simplification \\
		    		& \\
	\end{TAB}
	
	$ \therefore $ the equation of the line is $ y = 4x - 4 $.
\end{frame}


    \begin{frame}
    	\frametitle{Example 3}
    	Write the equation of the line passing through the points $ (-7, -1) $ and $ (1, -3) $. 
    	
    	\lcmtwobytwolineone{-2}{(2)}{(-1)}
    	\lcmtwobytwolinetwo{8}{(2^3)}{}
    	\lcmtwobytwolinethree{2^3}{-1}{-8}
    	
    	\begin{TAB}(@, 1mm, 5mm)[2mm]{ll}{cccccc}
    		\pause Given: $ (x_1, y_1) = (-7, -1)  $,  & \pause $ (x_2, y_2) = (1, -3) $  \\
    		
    		\pause $ \dfrac{y - y_1}{y_2 - y_1} = \dfrac{x - x_1}{x_2 - x_1}  $  &  \\
    		
    		\pause $ \dfrac{y - (-1)}{-3 - (-1)} = \dfrac{x - (-7)}{1 - (-7)}  $ & \pause Substitution \\
    		
    		\pause $ \dfrac{y + 1}{-2} = \dfrac{x + 7}{8} $ & \pause Simplification \\
    		
    		\only<7-19>{\footnotesize \lcmtwobytwo{1}{0.1} \normalsize} \only<20->{$ -8 \left[\dfrac{y + 1}{-2} \right] = -8\left[\dfrac{x + 7}{8} \right] $} & \only<21->{ Multiplication} \\
    		    		& \\
    	\end{TAB}
    \end{frame}

    \begin{frame}
	\frametitle{Example 3}
	\begin{TAB}(@, 1mm, 5mm)[2mm]{ll}{cccccc}
		$ -8 \left[\dfrac{y + 1}{-2} \right] = -8\left[\dfrac{x + 7}{8} \right] $ &  Multiplication \\
		
		\pause $ 4y + 4 = -x - 7 $ & \pause Distributive Property \\
		
		\pause $ 4y = -x - 7 - 4 $ & \pause Subtraction Property \\
		
		\pause $ 4y = -x - 11 $ & \pause Simplification \\
		
		\pause $ \dfrac{4y}{4} = \dfrac{-x}{4} - \dfrac{11}{4} $ & \pause Division Property \\
		
		\pause $ y = -\dfrac{1}{4}x - \dfrac{11}{4} $ & Simplification \\
	\end{TAB}
	
	$ \therefore $ the equation of the line is $ y = -\dfrac{1}{4}x - \dfrac{11}{4} $.
\end{frame}

\begin{frame}
	\frametitle{How to Find the Equation of a Line?}
	Use the formulae:
	\begin{enumerate}  
		\item Slope-intercept form: $ y = mx + b $
		\item \pause Point-slope form: $ y - y_1 = m(x - x_1) $
		\item \pause Two-point form: $  \dfrac{y - y_1}{y_2 - y_1} = \dfrac{x - x_1}{x_2 - x_1}$
	\end{enumerate}  
\end{frame}

\begin{frame}
	\frametitle{How to Find the Equation of a Line Given the Slope and a Point?}
	Use the point-slope form:  $ y - y_1 = m(x - x_1) $
\end{frame}

    \begin{frame}
	\frametitle{How to Find the Equation of a Line Given Two Points?}
	Use the two-point form: $  \dfrac{y - y_1}{y_2 - y_1} = \dfrac{x - x_1}{x_2 - x_1}$
\end{frame}

    \begin{frame}
    	\begin{center}
    		\textbf{\LARGE Thank you for watching.}
    	\end{center}
    \end{frame}
	
\end{document}