\documentclass[14pt]{beamer}
\usepackage[utf8]{inputenc}
\usepackage{bookman} %font style
\usepackage{multicol}
\usepackage{cancel}
\usepackage{tikz}
\usepackage{xcolor} % for font color
\usetikzlibrary{calc}
\usetikzlibrary{positioning}
\usetikzlibrary{arrows.meta}
\usetikzlibrary{angles,quotes}
%\usetikzlibrary{decorations.pathreplacing}
\usepackage{wasysym} %for checked symbol 
\usepackage{tabularx} 
\usepackage{color, colortbl} %coloring table cells
\usepackage{gensymb} %degree symbol
\usepackage{amsfonts} %integer symbol
\usepackage{pgfplots} %graphs
\usepackage{easytable} % TAB
\usepackage{siunitx} % celsius symbol
\usepackage{stackengine} %to define \pesos 
\usepackage{systeme} % system of equations
\usepackage{pifont} % cross mark

\newcommand\pesos{\stackengine{-1.1ex}{P}{\stackengine{-1ex}{$-$}{$-$}{O}{c}{F}{F}{S}}{O}{c}{F}{T}{S}} 

\definecolor{Lgray}{gray}{0.9}
\definecolor{Dgray}{gray}{0.6}

\newcolumntype{Y}{>{\centering\arraybackslash}X} %for tabularx

\makeatletter
\newlength\beamerleftmargin
\setlength\beamerleftmargin{\Gm@lmargin}
\makeatother

\newcommand{\void}{\text{\hspace{2em}}}

\newcommand{\minivoid}{\text{\hspace{1em}}}

\newcommand{\sol}[3]{\begin{tikzpicture}[remember picture, overlay]
		\color{red}
        \node at (current page.south west) (sw) {};
		\node[anchor=south west, inner sep=0pt] at ($(sw) + (#1, #2) $)   (a) {
			\begin{minipage}[t]{0.75\textwidth}
				#3
			\end{minipage}};
    \end{tikzpicture}}


\newcommand \redcheck {{\color{red}\checkmark}}

\newcommand \redcross {{\color{red}\ding{56}}}

\newcommand{\vone}{\vspace{1em}}

\newcommand{\vhalf}{\vspace{0.5em}}

\newcommand{\mygrid}[2]{\begin{center}
		\begin{tikzpicture}[scale=#1]		
			\draw [help lines] (-10, -10) grid (10, 10);
			\draw[line width=0.5mm, <->, >={Latex[round]}] (-10, 0) -- (10, 0);
			\draw[line width=0.5mm, <->, >={Latex[round]}] (0, -10) -- (0, 10);	
			\input{#2}
		\end{tikzpicture}  
	\end{center}} 

\newcommand{\arrowcomment}[9]{\begin{tikzpicture}[remember picture, overlay]
		\node at (current page.south west) (sw) {};
		\node[anchor=south west, inner sep=0pt] at ($(sw) + (0, 0) $)   (a) {
			\begin{tikzpicture}[->,>=stealth, thick, main node/.style={rectangle,font=\sffamily\bfseries}, remember picture, overlay]

				\node (1)  at (#1, #2) {};

				\node[main node] (2)  at (#3, #4) {#5};

				\draw [->, red] (1.#6) to [out=#8,in=#9] (2.#7);

			\end{tikzpicture}
		};
\end{tikzpicture}} 

\newcommand{\plotit}[3]{\begin{center}
		\begin{tikzpicture}[scale=#2, main node/.style={rectangle,font=\sffamily\bfseries}]		
			\draw [help lines] (-#3, -#3) grid (#3, #3);
			\draw[line width=0.5mm, <->, >={Latex[round]}] (-#3, 0) -- (#3, 0);
			\draw[line width=0.5mm, <->, >={Latex[round]}] (0, -#3) -- (0, #3);	
			\input{#1}
		\end{tikzpicture}  
\end{center}} 

\newcommand{\plotpoint}[7]{\begin{center}
		\begin{tikzpicture}[scale=#7]		
			\coordinate (a) at (#1, #2);
			\draw [help lines] (-#5, -#5) grid (#5, #5);
			\draw[line width=#6 mm, <->, >={Latex[round]}] (-#5, 0) -- (#5, 0);
			\draw[line width=#6 mm, <->, >={Latex[round]}] (0, -#5) -- (0, #5);	
			\fill [fill=black] (a) circle (#3 pt);
			\node[anchor=#4, inner sep=2pt, rotate=0] (a-label) at (a) {$(#1, #2)$};
			\end{tikzpicture}  
	\end{center}} 

\newcommand{\plotoverlay}[5]{\begin{tikzpicture}[remember picture, overlay]
		\node at (current page.south west) (sw) {};
		\node[anchor=south west, inner sep=0pt] at ($(sw) + (#1, #2) $)   (a) {
			\begin{minipage}[t]{0.75\textwidth}
				 \begin{tikzpicture}[scale=#4, remember picture, overlay]		

						\draw [help lines] (-#5, -#5) grid (#5, #5);

						\draw[line width=0.5mm, <->, >={Latex[round]}] (-#5, 0) -- (#5, 0);

						\draw[line width=0.5mm, <->, >={Latex[round]}] (0, -#5) -- (0, #5);	

						\input{#3}

				\end{tikzpicture}
		\end{minipage}};
\end{tikzpicture}}

\newcommand{\lcmthreebythree}[9]{\begin{tikzpicture}[remember picture, overlay]
		\color{red}
		\node at (current page.south west) (sw) {};
		\node[anchor=south west, inner sep=0pt] at ($(sw) + (#1, #2) $)   (a) {
			\begin{minipage}[t]{0.75\textwidth}
				\pause Find the LCM: \\

				\begin{tabular}{rclll}

					$ \pause #3 $ & $=$ & $ \pause #6 $ & & \\

					$ \pause #4 $ & $=$ &  & $ \pause #7$ & \\

					$ \pause #5  $ & $=$ &  & & $ \pause #8 $ \\

					\hline

					\pause LCM & $=$ & \pause $ (#6) $ & \pause $ (#7) $ & \pause $ (#8) \pause = #9 $  \\

				\end{tabular}
		\end{minipage}};
\end{tikzpicture}}


\newcommand{\lcmtwobytwolineone}[3]{
	\def \lcmtwobytwolineonenumber {#1}
	\def \lcmtwobytwolineonefactorone {#2}
	\def \lcmtwobytwolineonefactortwo {#3}}

\newcommand{\lcmtwobytwolinetwo}[3]{
	\def \lcmtwobytwolinetwonumber {#1}
	\def \lcmtwobytwolinetwofactorone {#2}
	\def \lcmtwobytwolinetwofactortwo {#3}
%	\lcmtwobytwolineone
}

\newcommand{\lcmtwobytwolinethree}[3]{
	\def \lcmtwobytwolinethreelcm {#3}
	\def \lcmtwobytwolinethreefactorone {#1}
	\def \lcmtwobytwolinethreefactortwo {#2}
%	\lcmtwobytwolinetwo
}

\newcommand{\lcmtwobytwo}[2]{\begin{tikzpicture}[remember picture, overlay]
%		\lcmtwobytwolinethree
		\color{red}
		\node at (current page.south west) (sw) {};
		\node[anchor=south west, inner sep=0pt] at ($(sw) + (#1, #2) $)   (a) {
			\begin{minipage}[t]{0.75\textwidth}
				\pause Find the LCM: 
				
				\begin{tabular}{rccccc}

					$ \pause \lcmtwobytwolineonenumber $ & $=$ & $ \pause \lcmtwobytwolineonefactorone $ & $ \pause \lcmtwobytwolineonefactortwo $ & & \\

					$ \pause \lcmtwobytwolinetwonumber $ & $=$ & $ \pause \lcmtwobytwolinetwofactorone $ & $ \pause \lcmtwobytwolinetwofactortwo $ & & \\

					\hline

					\pause LCM & $=$ & \pause $ (\lcmtwobytwolinethreefactorone) $ & \pause $ (\lcmtwobytwolinethreefactortwo) $ & $=$ & \pause $ \lcmtwobytwolinethreelcm $  \\

				\end{tabular}
		\end{minipage}};
\end{tikzpicture}}


\newcommand{\lcmtwobythreelineone}[4]{
	\def \lcmtwobythreelineonenumber {#1}
	\def \lcmtwobythreelineonefactorone {#2}
	\def \lcmtwobythreelineonefactortwo {#3}
	\def \lcmtwobythreelineonefactorthree {#4}}

\newcommand{\lcmtwobythreelinetwo}[4]{
	\def \lcmtwobythreelinetwonumber {#1}
	\def \lcmtwobythreelinetwofactorone {#2}
	\def \lcmtwobythreelinetwofactortwo {#3}
	\def \lcmtwobythreelinetwofactorthree {#4}
}

\newcommand{\lcmtwobythreelinethree}[4]{
	\def \lcmtwobythreelinethreefactorone {#1}
	\def \lcmtwobythreelinethreefactortwo {#2}
	\def \lcmtwobythreelinethreefactorthree {#3}
	\def \lcmtwobythreelinethreelcm {#4}
}

\newcommand{\lcmtwobythree}[2]{\begin{tikzpicture}[remember picture, overlay]
		\color{red}
		\node at (current page.south west) (sw) {};
		\node[anchor=south west, inner sep=0pt] at ($(sw) + (#1, #2) $)   (a) {
			\begin{minipage}[t]{0.75\textwidth}
				\pause Find the LCM: 
				
				\begin{tabular}{rcccccc}
					
					$ \pause \lcmtwobythreelineonenumber $ & $=$ & $ \pause \lcmtwobythreelineonefactorone $ & $ \pause \lcmtwobythreelineonefactortwo $ & $ \pause \lcmtwobythreelineonefactorthree $ & & \\
					
					$ \pause \lcmtwobythreelinetwonumber $ & $=$ & $ \pause \lcmtwobythreelinetwofactorone $ & $ \pause \lcmtwobythreelinetwofactortwo $ & $ \pause \lcmtwobythreelinetwofactorthree $ & & \\
					
					\hline
					
					\pause LCM & $=$ & \pause $ (\lcmtwobythreelinethreefactorone) $ & \pause $ (\lcmtwobythreelinethreefactortwo) $ & \pause $ (\lcmtwobythreelinethreefactorthree) $ & $=$ & \pause $ \lcmtwobythreelinethreelcm $  \\
					
				\end{tabular}
		\end{minipage}};
\end{tikzpicture}}

\newcommand{\plotsystvars}[8]{
	\def \eqone {#1}
	\def \eqtwo {#2}
	\def \solx {#3}
	\def \soly {#4}
	\def \solanchor {#5}
	\def \labelxshift {#6}
	\def \labelyshift {#7}
	\def \solmarksize {#8}
}

\newcommand{\plotsyst}[7]{
\begin{tikzpicture}[scale=#1]
	
	\begin{axis} 
		[
		xticklabels={}, 
		yticklabels={}, 
		ymin=-#2, ymax=#2,
		xmin=-#2, xmax=#2,
		axis lines = center, 
		inner axis line style={Latex-Latex,very thick}, 
		grid=both,
		minor tick num=#7, 
		tick align=inside,
		after end axis/.code={
			\path (axis cs: \solx,\soly) 
			node [anchor=\solanchor, xshift=\labelxshift pt, yshift=\labelyshift pt] {$ (\solx, \soly) $}; } 
		] 
		
		\addplot[<->, >={Latex[round]},  ultra thick, domain=#3:#4, samples=200]{\eqone}node[]{};
		
		\addplot[<->, >={Latex[round]},  ultra thick, domain=#5:#6, samples=200]{\eqtwo}node[]{};
			
		\pause \addplot[only marks, mark=*, mark size=\solmarksize pt, color=black,] coordinates {(\solx, \soly)};
	\end{axis} 

\end{tikzpicture} 
}
\usetheme{default}
\usecolortheme{seahorse}

\title[] {Solving Problems Involving Linear Equations -- Distance, Speed, and Time}
\author{Jonathan R. Bacolod}
\institute[SHS]{Sauyo High School}
\date{}

\begin{document}
	\frame{\titlepage}
	
	\begin{frame}
		\frametitle{What are Motion-related Problems?}
		Problems that make use of the formula $ d = ax + b $, where
	
		\begin{itemize} 
			\item \pause $ d $ = distance
			\item \pause $ a $ = constant rate the object is moving per unit time
			\item \pause  $ x $ = time the object has moved 
			\item \pause $ b $ = initial distance
		\end{itemize} 
	\end{frame}

    \begin{frame}
    	\frametitle{How  to Solve Motion-related Problems Involving Linear Equations?}
    	
    	\begin{enumerate}  
    		\item Read, understand, and analyze the problem. 
    		\item \pause Use the facts of the problem to form a working equation.
    		\item \pause Solve the equation.
    	\end{enumerate}  
    \end{frame}

    \begin{frame}
    	\frametitle{Example 1}
    	Eugene was engaged in reading his favorite Manga while going to his 5th class.
    	Unknowingly, Jenny who is 50 meters away coming from the opposite direction is also reading her report for her next class. If the distance between them is gradually getting smaller by one meter per second, how long will it take before they bump into each other?
    \end{frame}

    \begin{frame}
    	\frametitle{How  to Solve Motion-related Problems Involving Linear Equations?}
    	
    	\begin{enumerate}  
    		\item Read, understand, and analyze the problem. 
    	\end{enumerate}  
    \end{frame}

   \begin{frame}
   	\frametitle{Example 1}
   	\small Eugene was engaged in reading his favorite Manga while going to his 5th class.
   	Unknowingly, Jenny who is 50 meters away coming from the opposite direction is also reading her report for her next class. If the distance between them is gradually getting smaller by one meter per second, how long will it take before they bump into each other?
   	
   	\vone
   	\normalsize
   	\begin{tabular}{llll}
   		Given: & \pause $ b $ & $ = $ & 50 m. \\
   		 & \pause $ a $ & $ = $ & $ -1 \frac{m}{s}$ \\
   		 & \pause $ d $ & $ = $ & 0 m. \\
   		\pause Find: & $ x $ & $ = $ & time before they bump \\  
   	\end{tabular}
   \end{frame}  

    \begin{frame}
    	\frametitle{How  to Solve Motion-related Problems Involving Linear Equations?}
    	
    	\begin{enumerate}  
    		\item Read, understand, and analyze the problem. 
    		\item Use the facts of the problem to form a working equation.
    	\end{enumerate}  
    \end{frame}

    \begin{frame}
    	\frametitle{Example 1}
    	\begin{tabular}{llll}
    		Given: &  $ b $ & $ = $ & 50 m. \\
    		&  $ a $ & $ = $ & $ -1 \frac{m}{s}$ \\
    		&  $ d $ & $ = $ & 0 m. \\
    		 Find: & $ x $ & $ = $ & time before they bump \\  
    	\end{tabular}
    
        \begin{TAB}(@, 1mm, 5mm)[2mm]{ll}{ccc}
        	\pause $ d = ax + b $ &  \\
        	
        	\pause $ 0 m = \pause \left(-1 \frac{m}{s}\right)\pause x \pause + 50 m $  &\pause Substitution Property \\
        	& \\
        \end{TAB}
        
       \pause $ \therefore $ the working equation is $ 0 m = \left(-1 \frac{m}{s} \right) x + 50 m $.
    \end{frame}  

    \begin{frame}
    	\frametitle{How  to Solve Motion-related Problems Involving Linear Equations?}
    	
    	\begin{enumerate}  
    		\item Read, understand, and analyze the problem. 
    		\item Use the facts of the problem to form a working equation.
    		\item Solve the equation.
    	\end{enumerate}  
    \end{frame}

    \begin{frame}
    	\frametitle{Example 1}
    	Solve the equation $ 0 m = \left(-1 \frac{m}{s} \right) x + 50 m $.
    	
    	\begin{TAB}(@, 1mm, 5mm)[2mm]{ll}{cccc}
    		\pause $ 0 m - 50 m = \left(-1 \frac{m}{s} \right) x + 50 m - 50 m $ & \\
    		
    		\pause $ - 50 m = \left(-1 \frac{m}{s} \right) x $ & \\
    		
    		\pause $ \dfrac{- 50 m}{-1 \frac{m}{s}} = \dfrac{\left(-1 \frac{m}{s} \right) x}{-1 \frac{m}{s}} $ & \\
   
    		\pause $ 50 s  = x $ & \\
    	\end{TAB}
    	
    	\pause $ \therefore $ the time before they bump is 50 seconds.
    \end{frame}

\begin{frame}
	\frametitle{Example 2}
	The distance between two towns is 380 km. At the same moment, a passenger
	car and a truck start moving towards each other from different towns. They meet 4
	hours later. If the car drives 5 kph faster than the truck, what are their speeds in
	kilometers per hour?
\end{frame}

\begin{frame}
	\frametitle{How  to Solve Motion-related Problems Involving Linear Equations?}
	
	\begin{enumerate}  
		\item Read, understand, and analyze the problem. 
	\end{enumerate}  
\end{frame}

\begin{frame}
	\frametitle{Example 2}
	\footnotesize The distance between two towns is 380 km. At the same moment, a passenger car and a truck start moving towards each other from different towns. They meet 4	hours later. If the car drives 5 kph faster than the truck, what are their speeds in kilometers per hour?
	
	\vone
	\normalsize
	\begin{tabular}{llll}
		Given: & \pause $ b $ & $ = $ & 380 km. \\
		& \pause $ d $ & $ = $ & 0 km. \\
		& \pause $ x $ & $ = $ & $ 4h $ \\
		\pause Let: & \pause $ a_t $ & $ = $ & $ a_t $ kph \\
		& \pause $ a_c $ & $ = $ & $ a_t + 5 $ kph \\
		& \pause $ a $ & $ = $ & $ -(a_t + a_c) = -(2a_t + 5) $ kph \\
		&  & $ = $ & $-2a_t - 5 $ kph \\
		\pause Find: & \pause $ a_t $ & $ = $ & speed of truck \\
		   & \pause $ a_c $ & $ = $ & speed of car \\
	\end{tabular}
\end{frame}  

\begin{frame}
	\frametitle{How  to Solve Motion-related Problems Involving Linear Equations?}
	
	\begin{enumerate}  
		\item Read, understand, and analyze the problem. 
		\item Use the facts of the problem to form a working equation.
	\end{enumerate}  
\end{frame}

\begin{frame}
	\frametitle{Example 2}
	\begin{tabular}{llll}
		Given: &  $ b $ & $ = $ & 380 km. \\
		&  $ d $ & $ = $ & 0 km. \\
		&  $ x $ & $ = $ & $ 4h $ \\
		 Let: &  $ a_t $ & $ = $ & $ a_t $ kph \\
		&  $ a_c $ & $ = $ & $ a_t + 5 $ kph \\
		&  $ a $ & $ = $ & $-2a_t - 5 $ kph \\
	\end{tabular}
	
	\begin{TAB}(@, 1mm, 5mm)[2mm]{ll}{ccc}
		\pause $ d = ax + b $ &  \\
		
		\pause $ 0 km = \pause (-2a_t - 5  kph) \pause (4 h) \pause + 380 km $  & \\
		& \\
	\end{TAB}
	
	\pause $ \therefore $ the working equation is $ 0 km = (-2a_t - 5  kph) (4 h) + 380 km $.
\end{frame}  

\begin{frame}
	\frametitle{How  to Solve Motion-related Problems Involving Linear Equations?}
	
	\begin{enumerate}  
		\item Read, understand, and analyze the problem. 
		\item Use the facts of the problem to form a working equation.
		\item Solve the equation.
	\end{enumerate}  
\end{frame}

\begin{frame}
	\frametitle{Example 2}
	\small Solve the equation $ 0 km = (-2a_t - 5  kph) (4 h) + 380 km $.
	
	\begin{TAB}(@, 1mm, 5mm)[2mm]{ll}{cccccccc}
		\pause $ 0 km - 380 km = (-2a_t - 5  kph) (4 h) + 380 km - 380 km $ & \\
		
		\pause $ - 380 km = (-2a_t - 5  kph) (4 h) $ & \\
		
		\pause $ \dfrac{- 380 km}{(4 h)} = \dfrac{(-2a_t - 5  kph) (4 h)}{(4 h)} $ & \\
		
		\pause $ -95 kph  = -2a_t - 5  kph $ & \\
		
		\pause $ -95 kph + 5 kph  = -2a_t - 5 kph + 5 kph $ & \\
		
		\pause $ -90 kph  = -2a_t $ & \\
		
		\pause $ \dfrac{-90}{-2} kph  = \dfrac{-2a_t}{-2} $ & \\
		
		\pause $ 45 kph  = a_t $ & \\
	\end{TAB}
	\sol{5}{0.6}{\color{black}$ \therefore $ the truck's speed is 45 kph}
\end{frame}

    \begin{frame}
    	\frametitle{Example 2}
    	Find the car's speed.
    	
    	\begin{TAB}(@, 1mm, 5mm)[2mm]{ll}{cccc}
    		\pause $ a_c = a_t + 5 $ kph & \\
    		
    		\pause $ a_c = 45 + 5  $ kph & \pause Substitution \\
    		
    		\pause $ a_c = 50  $ kph & \pause Simplification \\
    		& \\
    	\end{TAB}
    	
    	$ \therefore $ the car's speed is 50 kph.
    \end{frame}
    
    \begin{frame}
    	\frametitle{Example 3}
    	Two trains start from the same station at the same time, train A is going north
    	bound while train B is going south bound. After 5 minutes the trains are 10km
    	apart, in 15 minutes they are 30km apart. How many minutes will it take for the
    	trains to be 50km apart?
    \end{frame}
    
    \begin{frame}
    	\frametitle{How  to Solve Motion-related Problems Involving Linear Equations?}
    	
    	\begin{enumerate}  
    		\item Read, understand, and analyze the problem. 
    	\end{enumerate}  
    \end{frame}
    
    \begin{frame}
    	\frametitle{Example 3}
    	\small Two trains start from the same station at the same time, train A is going north
bound while train B is going south bound. After 5 minutes the trains are 10km
apart, in 15 minutes they are 30km apart. How many minutes will it take for the
trains to be 50km apart?
    	    	
    	\vone
    	\normalsize
    	\begin{tabular}{llll}
    		Given: & \pause $ b $ & $ = $ & 0 km. \\
    		& \pause $ a_1 $ & $ = $ & $ \frac{10km}{5min}$ \\
    		& \pause $ a_2 $ & $ = $ & $ \frac{30km}{15min}$ \\
    		& \pause $ a $ & $ = $ & $ \frac{2km}{min}$ \\
    		& \pause $ d $ & $ = $ & 50 km. \\
    		\pause Find: & $ x $ & $ = $ & time to be 50 km apart \\  
    	\end{tabular}
    \end{frame}  
    
    \begin{frame}
    	\frametitle{How  to Solve Motion-related Problems Involving Linear Equations?}
    	
    	\begin{enumerate}  
    		\item Read, understand, and analyze the problem. 
    		\item Use the facts of the problem to form a working equation.
    	\end{enumerate}  
    \end{frame}
    
    \begin{frame}
    	\frametitle{Example 3}
    	\begin{tabular}{llll}
    		Given: &  $ b $ & $ = $ & 0 km. \\
    		&  $ a_1 $ & $ = $ & $ \frac{10km}{5min}$ \\
    		&  $ a_2 $ & $ = $ & $ \frac{30km}{15min}$ \\
    		&  $ a $ & $ = $ & $ \frac{2km}{min}$ \\
    		&  $ d $ & $ = $ & 50 km. \\
    		 Find: & $ x $ & $ = $ & time to be 50 km apart \\   
    	\end{tabular}
    	
    	\begin{TAB}(@, 1mm, 5mm)[2mm]{ll}{ccc}
    		\pause $ d = ax + b $ &  \\
    		
    		\pause $ 50 km = \pause \left(2 \frac{km}{min}\right)\pause x \pause + 0 km $  &\pause Substitution Property \\
    		& \\
    	\end{TAB}
    	
    	\pause $ \therefore $ the working equation is $ 50 km = \left(2 \frac{km}{min} \right) x$.
    \end{frame}  
    
    \begin{frame}
    	\frametitle{How  to Solve Motion-related Problems Involving Linear Equations?}
    	
    	\begin{enumerate}  
    		\item Read, understand, and analyze the problem. 
    		\item Use the facts of the problem to form a working equation.
    		\item Solve the equation.
    	\end{enumerate}  
    \end{frame}
    
    \begin{frame}
    	\frametitle{Example 3}
    	Solve the equation $ 50 km = \left(2 \frac{km}{min} \right) x$.
    	
    	\begin{TAB}(@, 1mm, 5mm)[2mm]{ll}{cccc}
    		\pause $ 50 km = \left(2 \frac{km}{min} \right) x$& \\
    		
    		\pause $ \dfrac{50 km}{\left(2 \frac{km}{min} \right)} = \dfrac{\left(2 \frac{km}{min} \right) x}{\left(2 \frac{km}{min} \right)}$ & \pause Division Property\\
    		
    		\pause $ 25 min  = x $ & \pause Simplification \\
    		& \\
    	\end{TAB}
    	
    	\pause $ \therefore $ the time it will take for the
trains to be 50km apart is 25 minutes.
    \end{frame}

    \begin{frame}
    	\frametitle{Example 4}
    	A long-distance runner started a course running at an average speed of 6 mph.
    	One and one-half hours later, a cyclist traveled the same course at an average
    	speed of 12 mph. How long after the runner started did the cyclist overtake the
    	runner?
    \end{frame}
    
    \begin{frame}
    	\frametitle{How  to Solve Motion-related Problems Involving Linear Equations?}
    	
    	\begin{enumerate}  
    		\item Read, understand, and analyze the problem. 
    	\end{enumerate}  
    \end{frame}
    
    \begin{frame}
    	\frametitle{Example 4}
    	\small A long-distance runner started a course running at an average speed of 6 mph.
One and one-half hours later, a cyclist traveled the same course at an average
speed of 12 mph. How long after the runner started did the cyclist overtake the
runner?
    	  
    	\vone
    	\normalsize
    	\begin{tabular}{llll}
    		Given: & \pause $ a_r $ & $ = $ & 6 mph \\
    		& \pause $ x_r $ & $ = $ & 1.5 h \\
    		& \pause $ b $ & $ = $ & $ (a_r)(x_r) = (6 mph)(1.5 h) = 9 mi. $ \\
    		& \pause $ a_c $ & $ = $ & 12 mph \\
    		& \pause $ a $ & $ = $ & $ a_r - a_c  = 6 - 12 = -6 mph $\\
    		& \pause $ d $ & $ = $ & 0 mi \\
    		\pause Find: & $ x $ & $ = $ & \small time for cyclist to overtake the runner \normalsize \\  
    		\pause & $ x_T $ & $ = $ & $ x + x_r $ \footnotesize (total time after the runner started)  \\  
    	\end{tabular}
    \end{frame}  
    
    \begin{frame}
    	\frametitle{How  to Solve Motion-related Problems Involving Linear Equations?}
    	
    	\begin{enumerate}  
    		\item Read, understand, and analyze the problem. 
    		\item Use the facts of the problem to form a working equation.
    	\end{enumerate}  
    \end{frame}
    
    \begin{frame}
    	\frametitle{Example 4}
    	\begin{tabular}{llll}
    		Given: &  $ a_r $ & $ = $ & 6 mph \\
    		&  $ x_r $ & $ = $ & 1.5 h \\
    		&  $ b $ & $ = $ & $ (a_r)(x_r) = (6 mph)(1.5 h) = 9 mi. $ \\
    		&  $ a_c $ & $ = $ & 12 mph \\
    		&  $ a $ & $ = $ & $ a_r - a_c = 6 - 12 = -6 mph $\\
    		&  $ d $ & $ = $ & 0 mi \\
    		 Find: & $ x $ & $ = $ & \small time for cyclist to overtake the runner \normalsize \\  
    		 & $ x_T $ & $ = $ & $ x + x_r $ \footnotesize (total time after the runner started)  \\  
    	\end{tabular}
    	
    	\begin{TAB}(@, 1mm, 5mm)[2mm]{ll}{ccc}
    		\pause $ d = ax + b $ &  \\
    		
    		\pause $ 0 mi = \pause \left(-6 mph\right)\pause (x) \pause + 9 mi $  &\pause Substitution Property \\
    		& \\
    	\end{TAB}
    	
    	\pause $ \therefore $ the working equation is $ 0 mi = (-6 mph) (x) + 9 mi $.
    \end{frame}  
    
    \begin{frame}
    	\frametitle{How  to Solve Motion-related Problems Involving Linear Equations?}
    	
    	\begin{enumerate}  
    		\item Read, understand, and analyze the problem. 
    		\item Use the facts of the problem to form a working equation.
    		\item Solve the equation.
    	\end{enumerate}  
    \end{frame}
    
    \begin{frame}
    	\frametitle{Example 4}
    	Solve the equation $ 0 mi = (-6 mph) (x) + 9 mi $.
    	
    	\vhalf
    	\begin{TAB}(@, 1mm, 5mm)[2mm]{ll}{cccc}
    		\pause $ 0 mi - 9 mi  = (-6 mph) (x) + 9 mi - 9 mi $ &  \\
    		
    		\pause $ - 9 mi  = (-6 mph) (x) $ & \\
    		
    		\pause $ \dfrac{- 9 mi}{(-6 mph)}  = \dfrac{(-6 mph) (x)}{(-6 mph)} $ &  \\
    		
    		\pause $ \dfrac{3}{2} h  = x $ & \\
    	\end{TAB}
    	
    	\pause $ \therefore $ the time for cyclist to overtake the runner is $ \dfrac{3}{2} $ hours or $ 1 \dfrac{1}{2} $ hours or 1.5 hours.
    \end{frame}

    \begin{frame}
    	\frametitle{Example 2}
    	Find the total time after the runner started. 
    	
    	\vhalf
    	\begin{TAB}(@, 1mm, 5mm)[2mm]{ll}{cccc}
    		\pause $ x_T = x + x_r $ & \\
    		
    		\pause $ x_T = 1.5 h + 1.5 h $ & \pause Substitution \\
    		
    		\pause $ x_T = 3 h$  & \pause Simplification \\
    		& \\
    	\end{TAB}
    	
    	$ \therefore $  the total time after the runner started is 3 hours. 
    \end{frame}
    
    \begin{frame}
    	\begin{center}
    		\textbf{\LARGE Thank you for watching.}
    	\end{center}
    \end{frame}
	
\end{document}