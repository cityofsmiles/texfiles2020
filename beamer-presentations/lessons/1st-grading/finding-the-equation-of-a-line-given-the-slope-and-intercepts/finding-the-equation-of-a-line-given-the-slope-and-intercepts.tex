\documentclass[14pt]{beamer}
\usepackage[utf8]{inputenc}
\usepackage{bookman} %font style
\usepackage{multicol}
\usepackage{cancel}
\usepackage{tikz}
\usepackage{xcolor} % for font color
\usetikzlibrary{calc}
\usetikzlibrary{positioning}
\usetikzlibrary{arrows.meta}
\usetikzlibrary{angles,quotes}
%\usetikzlibrary{decorations.pathreplacing}
\usepackage{wasysym} %for checked symbol 
\usepackage{tabularx} 
\usepackage{color, colortbl} %coloring table cells
\usepackage{gensymb} %degree symbol
\usepackage{amsfonts} %integer symbol
\usepackage{pgfplots} %graphs
\usepackage{easytable} % TAB
\usepackage{siunitx} % celsius symbol
\usepackage{stackengine} %to define \pesos 
\usepackage{systeme} % system of equations
\usepackage{pifont} % cross mark

\newcommand\pesos{\stackengine{-1.1ex}{P}{\stackengine{-1ex}{$-$}{$-$}{O}{c}{F}{F}{S}}{O}{c}{F}{T}{S}} 

\definecolor{Lgray}{gray}{0.9}
\definecolor{Dgray}{gray}{0.6}

\newcolumntype{Y}{>{\centering\arraybackslash}X} %for tabularx

\makeatletter
\newlength\beamerleftmargin
\setlength\beamerleftmargin{\Gm@lmargin}
\makeatother

\newcommand{\void}{\text{\hspace{2em}}}

\newcommand{\minivoid}{\text{\hspace{1em}}}

\newcommand{\sol}[3]{\begin{tikzpicture}[remember picture, overlay]
		\color{red}
        \node at (current page.south west) (sw) {};
		\node[anchor=south west, inner sep=0pt] at ($(sw) + (#1, #2) $)   (a) {
			\begin{minipage}[t]{0.75\textwidth}
				#3
			\end{minipage}};
    \end{tikzpicture}}


\newcommand \redcheck {{\color{red}\checkmark}}

\newcommand \redcross {{\color{red}\ding{56}}}

\newcommand{\vone}{\vspace{1em}}

\newcommand{\vhalf}{\vspace{0.5em}}

\newcommand{\mygrid}[2]{\begin{center}
		\begin{tikzpicture}[scale=#1]		
			\draw [help lines] (-10, -10) grid (10, 10);
			\draw[line width=0.5mm, <->, >={Latex[round]}] (-10, 0) -- (10, 0);
			\draw[line width=0.5mm, <->, >={Latex[round]}] (0, -10) -- (0, 10);	
			\input{#2}
		\end{tikzpicture}  
	\end{center}} 

\newcommand{\arrowcomment}[9]{\begin{tikzpicture}[remember picture, overlay]
		\node at (current page.south west) (sw) {};
		\node[anchor=south west, inner sep=0pt] at ($(sw) + (0, 0) $)   (a) {
			\begin{tikzpicture}[->,>=stealth, thick, main node/.style={rectangle,font=\sffamily\bfseries}, remember picture, overlay]

				\node (1)  at (#1, #2) {};

				\node[main node] (2)  at (#3, #4) {#5};

				\draw [->, red] (1.#6) to [out=#8,in=#9] (2.#7);

			\end{tikzpicture}
		};
\end{tikzpicture}} 

\newcommand{\plotit}[3]{\begin{center}
		\begin{tikzpicture}[scale=#2, main node/.style={rectangle,font=\sffamily\bfseries}]		
			\draw [help lines] (-#3, -#3) grid (#3, #3);
			\draw[line width=0.5mm, <->, >={Latex[round]}] (-#3, 0) -- (#3, 0);
			\draw[line width=0.5mm, <->, >={Latex[round]}] (0, -#3) -- (0, #3);	
			\input{#1}
		\end{tikzpicture}  
\end{center}} 

\newcommand{\plotpoint}[7]{\begin{center}
		\begin{tikzpicture}[scale=#7]		
			\coordinate (a) at (#1, #2);
			\draw [help lines] (-#5, -#5) grid (#5, #5);
			\draw[line width=#6 mm, <->, >={Latex[round]}] (-#5, 0) -- (#5, 0);
			\draw[line width=#6 mm, <->, >={Latex[round]}] (0, -#5) -- (0, #5);	
			\fill [fill=black] (a) circle (#3 pt);
			\node[anchor=#4, inner sep=2pt, rotate=0] (a-label) at (a) {$(#1, #2)$};
			\end{tikzpicture}  
	\end{center}} 

\newcommand{\plotoverlay}[5]{\begin{tikzpicture}[remember picture, overlay]
		\node at (current page.south west) (sw) {};
		\node[anchor=south west, inner sep=0pt] at ($(sw) + (#1, #2) $)   (a) {
			\begin{minipage}[t]{0.75\textwidth}
				 \begin{tikzpicture}[scale=#4, remember picture, overlay]		

						\draw [help lines] (-#5, -#5) grid (#5, #5);

						\draw[line width=0.5mm, <->, >={Latex[round]}] (-#5, 0) -- (#5, 0);

						\draw[line width=0.5mm, <->, >={Latex[round]}] (0, -#5) -- (0, #5);	

						\input{#3}

				\end{tikzpicture}
		\end{minipage}};
\end{tikzpicture}}

\newcommand{\lcmthreebythree}[9]{\begin{tikzpicture}[remember picture, overlay]
		\color{red}
		\node at (current page.south west) (sw) {};
		\node[anchor=south west, inner sep=0pt] at ($(sw) + (#1, #2) $)   (a) {
			\begin{minipage}[t]{0.75\textwidth}
				\pause Find the LCM: \\

				\begin{tabular}{rclll}

					$ \pause #3 $ & $=$ & $ \pause #6 $ & & \\

					$ \pause #4 $ & $=$ &  & $ \pause #7$ & \\

					$ \pause #5  $ & $=$ &  & & $ \pause #8 $ \\

					\hline

					\pause LCM & $=$ & \pause $ (#6) $ & \pause $ (#7) $ & \pause $ (#8) \pause = #9 $  \\

				\end{tabular}
		\end{minipage}};
\end{tikzpicture}}


\newcommand{\lcmtwobytwolineone}[3]{
	\def \lcmtwobytwolineonenumber {#1}
	\def \lcmtwobytwolineonefactorone {#2}
	\def \lcmtwobytwolineonefactortwo {#3}}

\newcommand{\lcmtwobytwolinetwo}[3]{
	\def \lcmtwobytwolinetwonumber {#1}
	\def \lcmtwobytwolinetwofactorone {#2}
	\def \lcmtwobytwolinetwofactortwo {#3}
%	\lcmtwobytwolineone
}

\newcommand{\lcmtwobytwolinethree}[3]{
	\def \lcmtwobytwolinethreelcm {#3}
	\def \lcmtwobytwolinethreefactorone {#1}
	\def \lcmtwobytwolinethreefactortwo {#2}
%	\lcmtwobytwolinetwo
}

\newcommand{\lcmtwobytwo}[2]{\begin{tikzpicture}[remember picture, overlay]
%		\lcmtwobytwolinethree
		\color{red}
		\node at (current page.south west) (sw) {};
		\node[anchor=south west, inner sep=0pt] at ($(sw) + (#1, #2) $)   (a) {
			\begin{minipage}[t]{0.75\textwidth}
				\pause Find the LCM: 
				
				\begin{tabular}{rccccc}

					$ \pause \lcmtwobytwolineonenumber $ & $=$ & $ \pause \lcmtwobytwolineonefactorone $ & $ \pause \lcmtwobytwolineonefactortwo $ & & \\

					$ \pause \lcmtwobytwolinetwonumber $ & $=$ & $ \pause \lcmtwobytwolinetwofactorone $ & $ \pause \lcmtwobytwolinetwofactortwo $ & & \\

					\hline

					\pause LCM & $=$ & \pause $ (\lcmtwobytwolinethreefactorone) $ & \pause $ (\lcmtwobytwolinethreefactortwo) $ & $=$ & \pause $ \lcmtwobytwolinethreelcm $  \\

				\end{tabular}
		\end{minipage}};
\end{tikzpicture}}


\newcommand{\lcmtwobythreelineone}[4]{
	\def \lcmtwobythreelineonenumber {#1}
	\def \lcmtwobythreelineonefactorone {#2}
	\def \lcmtwobythreelineonefactortwo {#3}
	\def \lcmtwobythreelineonefactorthree {#4}}

\newcommand{\lcmtwobythreelinetwo}[4]{
	\def \lcmtwobythreelinetwonumber {#1}
	\def \lcmtwobythreelinetwofactorone {#2}
	\def \lcmtwobythreelinetwofactortwo {#3}
	\def \lcmtwobythreelinetwofactorthree {#4}
}

\newcommand{\lcmtwobythreelinethree}[4]{
	\def \lcmtwobythreelinethreefactorone {#1}
	\def \lcmtwobythreelinethreefactortwo {#2}
	\def \lcmtwobythreelinethreefactorthree {#3}
	\def \lcmtwobythreelinethreelcm {#4}
}

\newcommand{\lcmtwobythree}[2]{\begin{tikzpicture}[remember picture, overlay]
		\color{red}
		\node at (current page.south west) (sw) {};
		\node[anchor=south west, inner sep=0pt] at ($(sw) + (#1, #2) $)   (a) {
			\begin{minipage}[t]{0.75\textwidth}
				\pause Find the LCM: 
				
				\begin{tabular}{rcccccc}
					
					$ \pause \lcmtwobythreelineonenumber $ & $=$ & $ \pause \lcmtwobythreelineonefactorone $ & $ \pause \lcmtwobythreelineonefactortwo $ & $ \pause \lcmtwobythreelineonefactorthree $ & & \\
					
					$ \pause \lcmtwobythreelinetwonumber $ & $=$ & $ \pause \lcmtwobythreelinetwofactorone $ & $ \pause \lcmtwobythreelinetwofactortwo $ & $ \pause \lcmtwobythreelinetwofactorthree $ & & \\
					
					\hline
					
					\pause LCM & $=$ & \pause $ (\lcmtwobythreelinethreefactorone) $ & \pause $ (\lcmtwobythreelinethreefactortwo) $ & \pause $ (\lcmtwobythreelinethreefactorthree) $ & $=$ & \pause $ \lcmtwobythreelinethreelcm $  \\
					
				\end{tabular}
		\end{minipage}};
\end{tikzpicture}}

\newcommand{\plotsystvars}[8]{
	\def \eqone {#1}
	\def \eqtwo {#2}
	\def \solx {#3}
	\def \soly {#4}
	\def \solanchor {#5}
	\def \labelxshift {#6}
	\def \labelyshift {#7}
	\def \solmarksize {#8}
}

\newcommand{\plotsyst}[7]{
\begin{tikzpicture}[scale=#1]
	
	\begin{axis} 
		[
		xticklabels={}, 
		yticklabels={}, 
		ymin=-#2, ymax=#2,
		xmin=-#2, xmax=#2,
		axis lines = center, 
		inner axis line style={Latex-Latex,very thick}, 
		grid=both,
		minor tick num=#7, 
		tick align=inside,
		after end axis/.code={
			\path (axis cs: \solx,\soly) 
			node [anchor=\solanchor, xshift=\labelxshift pt, yshift=\labelyshift pt] {$ (\solx, \soly) $}; } 
		] 
		
		\addplot[<->, >={Latex[round]},  ultra thick, domain=#3:#4, samples=200]{\eqone}node[]{};
		
		\addplot[<->, >={Latex[round]},  ultra thick, domain=#5:#6, samples=200]{\eqtwo}node[]{};
			
		\pause \addplot[only marks, mark=*, mark size=\solmarksize pt, color=black,] coordinates {(\solx, \soly)};
	\end{axis} 

\end{tikzpicture} 
}
\usetheme{default}
\usecolortheme{seahorse}

\title[] {Finding the Equation of a Line Given the Slope and Intercepts}
\author{Jonathan R. Bacolod}
\institute[SHS]{Sauyo High School}
\date{}

\begin{document}
	\frame{\titlepage}
	
	\begin{frame}
		\frametitle{Review}
		What is the x-intercept and y-intercept?
		\begin{itemize}
			\item If a line crosses the x-axis at the point $ (a,0) $, then the
			number $ a $ is the x-intercept of the line.
			\item \pause If a line crosses the y-axis at the point $ (0,b) $, then the
			number $ b $ is the y-intercept of the line. 
		\end{itemize}
	\end{frame}
		
	\begin{frame}
		\frametitle{How to Find the Equation of a Line?}
		Use the formulae:
		\begin{enumerate}  
			\item Slope-intercept form: $ y = mx + b $
			\item \pause Two-intercept form: $  \dfrac{x}{a} + \dfrac{y}{b} = 1$
		\end{enumerate}  
	\end{frame}

    \begin{frame}
    	\frametitle{How to Find the Equation of a Line Given the Slope and the y-intercept?}
    	Use the slope-intercept form: $ y = mx + b $
    \end{frame}

    \begin{frame}
    	\frametitle{Example 1}
    	Find the equation of the line with a slope equal to $ \dfrac{2}{3} $ and a y-intercept equal to $ 5 $.
    	    	
    	\begin{TAB}(@, 1mm, 5mm)[2mm]{ll}{cccc}
    		\pause Given: $ m = \dfrac{2}{3} $,  & \pause $ b = 5 $  \\
    		
    		\pause $ y = mx + b  $  &  \\
    		
    		\pause $ y = \dfrac{2}{3}x + 5  $ & \pause Substitution \\
    		& \\
    	\end{TAB}
  
      $ \therefore $ the equation of the line is $ y = \dfrac{2}{3}x + 5 $.
    \end{frame}

    \begin{frame}
    	\frametitle{Example 2}
    	 If the slope of a line is $ 4 $ and it passes through the origin, what is its
    	 equation?
    	     	
    	\begin{TAB}(@, 1mm, 5mm)[2mm]{ll}{cccc}
    		\pause Given: $ m = 4 $,  & \pause $ (x_1, y_1) = (0, 0), \pause b = 0 $  \\
    		
    		\pause $ y = mx + b  $  &  \\
    		
    		\pause $ y = 4x + 0  $ & \pause Substitution \\
    		& \\
    	\end{TAB}
    	
    	$ \therefore $ the equation of the line is $ y = 4x $.
    \end{frame}

    \begin{frame}
    	\frametitle{Example 3}
    	A line that passes through $ (0,-2) $ has a slope equal to $ 6 $. Find its equation.
    	    	
    	\begin{TAB}(@, 1mm, 5mm)[2mm]{ll}{cccc}
    		\pause Given: $ m = 6 $,  & \pause $ b = -2 $  \\
    		
    		\pause $ y = mx + b $  &  \\
    		
    		\pause $ y = 6x - 2 $ & \pause Substitution \\
    		& \\
    	\end{TAB}
    	
    	$ \therefore $ the equation of the line is $ y = 6x - 2 $.
    \end{frame}

    \begin{frame}
    	\frametitle{How to Find the Equation of a Line Given the Intercepts?}
    	Use the two-intercept form: $  \dfrac{x}{a} + \dfrac{y}{b} = 1$.
    \end{frame}

    \begin{frame}
    	\frametitle{Example 1}
    	Find equation of a line that has with an x–intercept $ a = 3 $ and y–
    	intercept $ b = 4 $.
    	
    \lcmtwobytwolineone{3}{(3)}{}
    \lcmtwobytwolinetwo{4}{}{(2^2)}
    \lcmtwobytwolinethree{3}{2^2}{12}
    
       	\begin{TAB}(@, 1mm, 5mm)[2mm]{ll}{cccccc}
    		\pause Given: $ a = 3 $,  & \pause $ b = 4 $  \\
    		
    		\pause $ \dfrac{x}{a} + \dfrac{y}{b} = 1 $  &  \\
    		
    		\pause $ \dfrac{x}{3} + \dfrac{y}{4} = 1 $ &\pause Substitution \\
    		    		
    	    \only<7-17>{\lcmtwobytwo{1}{0.1}} \only<18->{$ 12\left[\dfrac{x}{3} + \dfrac{y}{4} \right] = 12(1) $} & \only<19->{MPE} \\
    	    
    	    \only<20->{$ 4x + 3y = 12 $} & \only<21->{Distributive Property} \\
    		& \\
    	\end{TAB}
   
    	\only<22>{$ \therefore $ the equation of the line is $ 4x + 3y = 12 $.}
    \end{frame}
    
    \begin{frame}
    	\frametitle{Example 2}
    	What is the equation of the line that passes through points $ (0,2) $ and $ (7,0) $?
    	    	
    	\lcmtwobytwolineone{7}{(7)}{}
    	\lcmtwobytwolinetwo{2}{}{(2)}
    	\lcmtwobytwolinethree{7}{2}{14}
    	
    	\begin{TAB}(@, 1mm, 5mm)[2mm]{ll}{cccccc}
    		\pause Given: $ P_1 (0, 2), \pause b = 2 $,  & \pause $ P_2 (7, 0), \pause a = 7 $  \\
    		
    		\pause $ \dfrac{x}{a} + \dfrac{y}{b} = 1 $  &  \\
    		
    		\pause $ \dfrac{x}{7} + \dfrac{y}{2} = 1 $ &\pause Substitution \\
    		
    		\only<7-19>{\lcmtwobytwo{1}{0.1}} \only<20->{$ 14\left[\dfrac{x}{7} + \dfrac{y}{2} \right] = 14(1) $} & \only<21->{MPE} \\
    		
    		\only<22->{$ 2x + 7y = 14 $} & \only<23->{Distributive Property} \\
    		& \\
    	\end{TAB}
    	
    	\only<24>{$ \therefore $ the equation of the line is $ 2x + 7y = 14 $.}
    \end{frame}

    \begin{frame}
    	\frametitle{Example 3}
    	\small Find the equation of the line with $ 6 $ and $ -3 $ as its x- and y-intercepts
    	respectively.
    	    	
    	\lcmtwobythreelineone{6}{(2)}{(3)}{}
    	\lcmtwobythreelinetwo{-3}{}{(3)}{(-1)}
    	\lcmtwobythreelinethree{2}{3}{-1}{-6}
    	
    	\begin{TAB}(@, 1mm, 5mm)[2mm]{ll}{cccccc}
    		\pause Given: $ a = 6 $,  & \pause $ b = -3 $  \\
    		
    		\pause $ \dfrac{x}{a} + \dfrac{y}{b} = 1 $  &  \\
    		
    		\pause $ \dfrac{x}{6} + \dfrac{y}{-3} = 1 $ &\pause Substitution \\
    		
    		\only<7-20>{\lcmtwobythree{1}{0.1}} \only<21->{$ 6\left[\dfrac{x}{6} + \dfrac{y}{-3} \right] = 6(1) $} & \only<22->{MPE} \\
    		
    		\only<23->{$ x - 2y = 6 $} & \only<24->{Distributive Property} \\
    		& \\
    	\end{TAB}
    	
    	\only<25>{$ \therefore $ the equation of the line is $ x - 2y = 6 $.}
    \end{frame}

    \begin{frame}
    	\begin{center}
    		\textbf{\LARGE Thank you for watching.}
    	\end{center}
    \end{frame}
	
\end{document}