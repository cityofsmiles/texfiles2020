\documentclass[14pt]{beamer}
\usepackage[utf8]{inputenc}
\usepackage{bookman} %font style
\usepackage{multicol}
\usepackage{cancel}
\usepackage{tikz}
\usepackage{xcolor} % for font color
\usetikzlibrary{calc}
\usetikzlibrary{positioning}
\usetikzlibrary{arrows.meta}
\usetikzlibrary{angles,quotes}
%\usetikzlibrary{decorations.pathreplacing}
\usepackage{wasysym} %for checked symbol 
\usepackage{tabularx} 
\usepackage{color, colortbl} %coloring table cells
\usepackage{gensymb} %degree symbol
\usepackage{amsfonts} %integer symbol
\usepackage{pgfplots} %graphs
\usepackage{easytable} % TAB
\usepackage{siunitx} % celsius symbol
\usepackage{stackengine} %to define \pesos 
\usepackage{systeme} % system of equations
\usepackage{pifont} % cross mark

\newcommand\pesos{\stackengine{-1.1ex}{P}{\stackengine{-1ex}{$-$}{$-$}{O}{c}{F}{F}{S}}{O}{c}{F}{T}{S}} 

\definecolor{Lgray}{gray}{0.9}
\definecolor{Dgray}{gray}{0.6}

\newcolumntype{Y}{>{\centering\arraybackslash}X} %for tabularx

\makeatletter
\newlength\beamerleftmargin
\setlength\beamerleftmargin{\Gm@lmargin}
\makeatother

\newcommand{\void}{\text{\hspace{2em}}}

\newcommand{\minivoid}{\text{\hspace{1em}}}

\newcommand{\sol}[3]{\begin{tikzpicture}[remember picture, overlay]
		\color{red}
        \node at (current page.south west) (sw) {};
		\node[anchor=south west, inner sep=0pt] at ($(sw) + (#1, #2) $)   (a) {
			\begin{minipage}[t]{0.75\textwidth}
				#3
			\end{minipage}};
    \end{tikzpicture}}


\newcommand \redcheck {{\color{red}\checkmark}}

\newcommand \redcross {{\color{red}\ding{56}}}

\newcommand{\vone}{\vspace{1em}}

\newcommand{\vhalf}{\vspace{0.5em}}

\newcommand{\mygrid}[2]{\begin{center}
		\begin{tikzpicture}[scale=#1]		
			\draw [help lines] (-10, -10) grid (10, 10);
			\draw[line width=0.5mm, <->, >={Latex[round]}] (-10, 0) -- (10, 0);
			\draw[line width=0.5mm, <->, >={Latex[round]}] (0, -10) -- (0, 10);	
			\input{#2}
		\end{tikzpicture}  
	\end{center}} 

\newcommand{\arrowcomment}[9]{\begin{tikzpicture}[remember picture, overlay]
		\node at (current page.south west) (sw) {};
		\node[anchor=south west, inner sep=0pt] at ($(sw) + (0, 0) $)   (a) {
			\begin{tikzpicture}[->,>=stealth, thick, main node/.style={rectangle,font=\sffamily\bfseries}, remember picture, overlay]

				\node (1)  at (#1, #2) {};

				\node[main node] (2)  at (#3, #4) {#5};

				\draw [->, red] (1.#6) to [out=#8,in=#9] (2.#7);

			\end{tikzpicture}
		};
\end{tikzpicture}} 

\newcommand{\plotit}[3]{\begin{center}
		\begin{tikzpicture}[scale=#2, main node/.style={rectangle,font=\sffamily\bfseries}]		
			\draw [help lines] (-#3, -#3) grid (#3, #3);
			\draw[line width=0.5mm, <->, >={Latex[round]}] (-#3, 0) -- (#3, 0);
			\draw[line width=0.5mm, <->, >={Latex[round]}] (0, -#3) -- (0, #3);	
			\input{#1}
		\end{tikzpicture}  
\end{center}} 

\newcommand{\plotpoint}[7]{\begin{center}
		\begin{tikzpicture}[scale=#7]		
			\coordinate (a) at (#1, #2);
			\draw [help lines] (-#5, -#5) grid (#5, #5);
			\draw[line width=#6 mm, <->, >={Latex[round]}] (-#5, 0) -- (#5, 0);
			\draw[line width=#6 mm, <->, >={Latex[round]}] (0, -#5) -- (0, #5);	
			\fill [fill=black] (a) circle (#3 pt);
			\node[anchor=#4, inner sep=2pt, rotate=0] (a-label) at (a) {$(#1, #2)$};
			\end{tikzpicture}  
	\end{center}} 

\newcommand{\plotoverlay}[5]{\begin{tikzpicture}[remember picture, overlay]
		\node at (current page.south west) (sw) {};
		\node[anchor=south west, inner sep=0pt] at ($(sw) + (#1, #2) $)   (a) {
			\begin{minipage}[t]{0.75\textwidth}
				 \begin{tikzpicture}[scale=#4, remember picture, overlay]		

						\draw [help lines] (-#5, -#5) grid (#5, #5);

						\draw[line width=0.5mm, <->, >={Latex[round]}] (-#5, 0) -- (#5, 0);

						\draw[line width=0.5mm, <->, >={Latex[round]}] (0, -#5) -- (0, #5);	

						\input{#3}

				\end{tikzpicture}
		\end{minipage}};
\end{tikzpicture}}

\newcommand{\lcmthreebythree}[9]{\begin{tikzpicture}[remember picture, overlay]
		\color{red}
		\node at (current page.south west) (sw) {};
		\node[anchor=south west, inner sep=0pt] at ($(sw) + (#1, #2) $)   (a) {
			\begin{minipage}[t]{0.75\textwidth}
				\pause Find the LCM: \\

				\begin{tabular}{rclll}

					$ \pause #3 $ & $=$ & $ \pause #6 $ & & \\

					$ \pause #4 $ & $=$ &  & $ \pause #7$ & \\

					$ \pause #5  $ & $=$ &  & & $ \pause #8 $ \\

					\hline

					\pause LCM & $=$ & \pause $ (#6) $ & \pause $ (#7) $ & \pause $ (#8) \pause = #9 $  \\

				\end{tabular}
		\end{minipage}};
\end{tikzpicture}}


\newcommand{\lcmtwobytwolineone}[3]{
	\def \lcmtwobytwolineonenumber {#1}
	\def \lcmtwobytwolineonefactorone {#2}
	\def \lcmtwobytwolineonefactortwo {#3}}

\newcommand{\lcmtwobytwolinetwo}[3]{
	\def \lcmtwobytwolinetwonumber {#1}
	\def \lcmtwobytwolinetwofactorone {#2}
	\def \lcmtwobytwolinetwofactortwo {#3}
%	\lcmtwobytwolineone
}

\newcommand{\lcmtwobytwolinethree}[3]{
	\def \lcmtwobytwolinethreelcm {#3}
	\def \lcmtwobytwolinethreefactorone {#1}
	\def \lcmtwobytwolinethreefactortwo {#2}
%	\lcmtwobytwolinetwo
}

\newcommand{\lcmtwobytwo}[2]{\begin{tikzpicture}[remember picture, overlay]
%		\lcmtwobytwolinethree
		\color{red}
		\node at (current page.south west) (sw) {};
		\node[anchor=south west, inner sep=0pt] at ($(sw) + (#1, #2) $)   (a) {
			\begin{minipage}[t]{0.75\textwidth}
				\pause Find the LCM: 
				
				\begin{tabular}{rccccc}

					$ \pause \lcmtwobytwolineonenumber $ & $=$ & $ \pause \lcmtwobytwolineonefactorone $ & $ \pause \lcmtwobytwolineonefactortwo $ & & \\

					$ \pause \lcmtwobytwolinetwonumber $ & $=$ & $ \pause \lcmtwobytwolinetwofactorone $ & $ \pause \lcmtwobytwolinetwofactortwo $ & & \\

					\hline

					\pause LCM & $=$ & \pause $ (\lcmtwobytwolinethreefactorone) $ & \pause $ (\lcmtwobytwolinethreefactortwo) $ & $=$ & \pause $ \lcmtwobytwolinethreelcm $  \\

				\end{tabular}
		\end{minipage}};
\end{tikzpicture}}


\newcommand{\lcmtwobythreelineone}[4]{
	\def \lcmtwobythreelineonenumber {#1}
	\def \lcmtwobythreelineonefactorone {#2}
	\def \lcmtwobythreelineonefactortwo {#3}
	\def \lcmtwobythreelineonefactorthree {#4}}

\newcommand{\lcmtwobythreelinetwo}[4]{
	\def \lcmtwobythreelinetwonumber {#1}
	\def \lcmtwobythreelinetwofactorone {#2}
	\def \lcmtwobythreelinetwofactortwo {#3}
	\def \lcmtwobythreelinetwofactorthree {#4}
}

\newcommand{\lcmtwobythreelinethree}[4]{
	\def \lcmtwobythreelinethreefactorone {#1}
	\def \lcmtwobythreelinethreefactortwo {#2}
	\def \lcmtwobythreelinethreefactorthree {#3}
	\def \lcmtwobythreelinethreelcm {#4}
}

\newcommand{\lcmtwobythree}[2]{\begin{tikzpicture}[remember picture, overlay]
		\color{red}
		\node at (current page.south west) (sw) {};
		\node[anchor=south west, inner sep=0pt] at ($(sw) + (#1, #2) $)   (a) {
			\begin{minipage}[t]{0.75\textwidth}
				\pause Find the LCM: 
				
				\begin{tabular}{rcccccc}
					
					$ \pause \lcmtwobythreelineonenumber $ & $=$ & $ \pause \lcmtwobythreelineonefactorone $ & $ \pause \lcmtwobythreelineonefactortwo $ & $ \pause \lcmtwobythreelineonefactorthree $ & & \\
					
					$ \pause \lcmtwobythreelinetwonumber $ & $=$ & $ \pause \lcmtwobythreelinetwofactorone $ & $ \pause \lcmtwobythreelinetwofactortwo $ & $ \pause \lcmtwobythreelinetwofactorthree $ & & \\
					
					\hline
					
					\pause LCM & $=$ & \pause $ (\lcmtwobythreelinethreefactorone) $ & \pause $ (\lcmtwobythreelinethreefactortwo) $ & \pause $ (\lcmtwobythreelinethreefactorthree) $ & $=$ & \pause $ \lcmtwobythreelinethreelcm $  \\
					
				\end{tabular}
		\end{minipage}};
\end{tikzpicture}}

\newcommand{\plotsystvars}[8]{
	\def \eqone {#1}
	\def \eqtwo {#2}
	\def \solx {#3}
	\def \soly {#4}
	\def \solanchor {#5}
	\def \labelxshift {#6}
	\def \labelyshift {#7}
	\def \solmarksize {#8}
}

\newcommand{\plotsyst}[7]{
\begin{tikzpicture}[scale=#1]
	
	\begin{axis} 
		[
		xticklabels={}, 
		yticklabels={}, 
		ymin=-#2, ymax=#2,
		xmin=-#2, xmax=#2,
		axis lines = center, 
		inner axis line style={Latex-Latex,very thick}, 
		grid=both,
		minor tick num=#7, 
		tick align=inside,
		after end axis/.code={
			\path (axis cs: \solx,\soly) 
			node [anchor=\solanchor, xshift=\labelxshift pt, yshift=\labelyshift pt] {$ (\solx, \soly) $}; } 
		] 
		
		\addplot[<->, >={Latex[round]},  ultra thick, domain=#3:#4, samples=200]{\eqone}node[]{};
		
		\addplot[<->, >={Latex[round]},  ultra thick, domain=#5:#6, samples=200]{\eqtwo}node[]{};
			
		\pause \addplot[only marks, mark=*, mark size=\solmarksize pt, color=black,] coordinates {(\solx, \soly)};
	\end{axis} 

\end{tikzpicture} 
}
\usetheme{default}
\usecolortheme{seahorse}

\title[] {Graphs of Linear Equations}
\author{Jonathan R. Bacolod}
\institute[SHS]{Sauyo High School}
\date{}

\begin{document}
	\frame{\titlepage}
	
	\begin{frame}
		\frametitle{How to Graph Linear Equations?}
		Graphing linear equations can be done using any of these methods: 
		\begin{enumerate}  
			\item \pause Using two points 
			\item \pause Using the x- and  y-intercepts
			\item \pause Using the slope  and a point
		\end{enumerate}  
	\end{frame}

    \begin{frame}
    	\frametitle{How to Graph Linear Equations Using Two Points?}
    	\begin{enumerate}  
    		\item Assign any two values for $ x $.
    		\item \pause Find the values for $ y $ to determine the ordered pairs of two points.
    		\item \pause Plot the two points and connect them.
    	\end{enumerate}  
    \end{frame}

    \begin{frame}
    	\frametitle{Example 1}
    	Graph the equation $ y = 2x + 1 $.
    \end{frame}

    \begin{frame}
    	\frametitle{How to Graph Linear Equations Using Two Points?}
    	\begin{enumerate}  
    		\item Assign any two values for $ x $.
    	\end{enumerate}  
    \end{frame}

    \begin{frame}
    	\frametitle{Example 1}
    	Graph the equation $ y = 2x + 1 $.
    	
    	\pause \vone Let $ x = 0 $
    	
    	\vone \pause Let $ x = 1 $
    \end{frame}

    \begin{frame}
    	\frametitle{How to Graph Linear Equations Using Two Points?}
    	\begin{enumerate}  
    		\item Assign any two values for $ x $.
    		\item Find the values for $ y $ to determine the ordered pairs of two points.
    	\end{enumerate}  
    \end{frame}

    \begin{frame}
    	\frametitle{Example 1}
    	Graph the equation $ y = 2x + 1 $.
    	
    	\vone Let $ x = 0: $
    	
    	\vone \pause $ y = 2(0) + 1 $ \pause \void Substitution
    	
    	\pause \vone $ y = 0 + 1 $
    	
    	\pause \vone $ y = 1 $
    	
    	\pause \vone $ \therefore $ the first ordered pair is $ (0, 1) $.
    \end{frame}

    \begin{frame}
    	\frametitle{Example 1}
    	Graph the equation $ y = 2x + 1 $.
    	
    	\vone Let $ x = 1: $
    	
    	\vone \pause $ y = 2(1) + 1 $ \pause \void Substitution
    	
    	\pause \vone $ y = 2 + 1 $
    	
    	\pause \vone $ y = 3 $
    	
    	\pause \vone $ \therefore $ the second ordered pair is $ (1, 3) $.
    \end{frame}

   \begin{frame}
   	\frametitle{How to Graph Linear Equations Using Two Points?}
   	\begin{enumerate}  
   		\item Assign any two values for $ x $.
   		\item Find the values for $ y $ to determine the ordered pairs of two points.
   		\item Plot the two points and connect them.
   	\end{enumerate}  
   \end{frame}

    \begin{frame}
    	\frametitle{Example 1}
    	Plot $ (0, 1) $ and $ (1, 3) $.
    	\plotit{graph1}{0.5}{6}
    \end{frame}

    \begin{frame}
    	\frametitle{Example 2}
    	Graph the equation $ x + 2y = 6 $.
    \end{frame}
    
    \begin{frame}
    	\frametitle{How to Graph Linear Equations Using Two Points?}
    	\begin{enumerate}  
    		\item Assign any two values for $ x $.
    	\end{enumerate}  
    \end{frame}
    
    \begin{frame}
    	\frametitle{Example 2}
    	Graph the equation $ x + 2y = 6 $.
    	
    	\pause \vone Let $ x = -2 $
    	
    	\vone \pause Let $ x = 2 $
    \end{frame}
    
    \begin{frame}
    	\frametitle{How to Graph Linear Equations Using Two Points?}
    	\begin{enumerate}  
    		\item Assign any two values for $ x $.
    		\item Find the values for $ y $ to determine the ordered pairs of two points.
    	\end{enumerate}  
    \end{frame}
    
    \begin{frame}
    	\frametitle{Example 2}
    	Graph the equation $ x + 2y = 6 $.
    	
    	\vone Let $ x = -2: $
    	
    	\vhalf
    	
        \begin{TAB}(@, 1mm, 5mm)[2mm]{ll}{ccccc}
        	\pause $ -2 + 2y = 6 $ & \pause Substitution Property \\
        	
        	\pause $ 2y = 6 + 2 $ & \pause Addition Property\\
        	
        	\pause  $ 2y = 8 $ & \pause Simplification \\
        	
        	\pause  $ \dfrac{2y}{2} = \dfrac{8}{2} $ & \pause  Division Property \\
        	
        	\pause  $ y = 4$ & Simplification \\
        \end{TAB}
    		
    	\pause $ \therefore $ the first ordered pair is $ (-2, 4) $.
    \end{frame}
    
    \begin{frame}
    	\frametitle{Example 2}
    	Graph the equation $ x + 2y = 6 $.
    	
    	\vone Let $ x = 2: $
    	
    	\vhalf
    	
    	\begin{TAB}(@, 1mm, 5mm)[2mm]{ll}{ccccc}
    		\pause $ 2 + 2y = 6 $ & \pause Substitution Property \\
    		
    		\pause $ 2y = 6 - 2 $ & \pause Subtraction Property\\
    		
    		\pause  $ 2y = 4 $ & \pause Simplification \\
    		
    		\pause  $ \dfrac{2y}{2} = \dfrac{4}{2} $ & \pause  Division Property \\
    		
    		\pause  $ y = 2$ & Simplification \\
    	\end{TAB}
    	
    	\pause $ \therefore $ the second ordered pair is $ (2, 2) $.
    \end{frame}
    
    \begin{frame}
    	\frametitle{How to Graph Linear Equations Using Two Points?}
    	\begin{enumerate}  
    		\item Assign any two values for $ x $.
    		\item Find the values for $ y $ to determine the ordered pairs of two points.
    		\item Plot the two points and connect them.
    	\end{enumerate}  
    \end{frame}
    
    \begin{frame}
    	\frametitle{Example 2}
    	Plot $ (-2, 4) $ and $ (2, 2) $.
    	\plotit{graph2}{0.4}{8}
    \end{frame}

    \begin{frame}
    	\frametitle{What is the x-intercept?}
    	\begin{itemize}
    		\item If a line crosses the x-axis at the point $ (a,0) $, then the
    		number $ a $ is the x-intercept of the line.
    		\item \pause It is the x-coordinate of the point where the line crosses the x-axis.
    	\end{itemize}
    \end{frame}

    \begin{frame}
    	\frametitle{What is the y-intercept?}
    	\begin{itemize}
    		\item If a line crosses the y-axis at the point $ (0,b) $, then the
    		number $ b $ is the y-intercept of the line. 
    		\item \pause It is the y-coordinate of the point where the graph crosses the y-axis.
    	\end{itemize}
    \end{frame}

    \begin{frame}
    	\frametitle{x- and y-intercepts}
    	\plotit{intercepts}{0.4}{8}
    	
    	\pause \arrowcomment{7.95}{5}{4}{3}{$ a = 3 $}{south}{east}{270}{0}
    	
    	\pause \arrowcomment{7.3}{6.67}{4}{6}{$ b= 4 $}{south}{east}{270}{0}
    \end{frame}

    \begin{frame}
    	\frametitle{How to Graph Linear Equations Using the x- and y-intercepts?}
    	\begin{enumerate}  
    		\item Let $ y = 0 $ to find the x-intercept.
    		\item \pause Let $ x = 0 $ to find the y-intercept.
    		\item \pause Plot the two points and connect them.
    	\end{enumerate} 
    \end{frame} 

    \begin{frame}
    	\frametitle{Example 1}
    	Graph the equation $ 4x - 3y = 12 $.
    \end{frame} 

    \begin{frame}
    	\frametitle{How to Graph Linear Equations Using the x- and y-intercepts?}
    	\begin{enumerate}  
    		\item Let $ y = 0 $ to find the x-intercept.
    	\end{enumerate} 
    \end{frame} 

    \begin{frame}
    	\frametitle{Example 1}
    	Graph the equation $ 4x - 3y = 12 $.
    	
    	\vone Let $ y = 0: $
    	
    	\vhalf
    	
    	\begin{TAB}(@, 1mm, 5mm)[2mm]{ll}{cccc}
    		\pause $ 4x - 3(0) = 12 $ & \pause Substitution Property \\
    		
    		\pause $ 4x - 0 = 12 $ & \pause Simplification\\
    		
    		\pause  $ \dfrac{4x}{4} = \dfrac{12}{4} $ & \pause Division Property\\
    		
    		\pause  $ x = 3 $ & Simplification \\
    	\end{TAB}
    	
    	\pause $ \therefore $ the x-intercept $ a $ is $ 3 $ and the point is $ (3, 0) $.
    \end{frame}

    \begin{frame}
    	\frametitle{How to Graph Linear Equations Using the x- and y-intercepts?}
    	\begin{enumerate}  
    		\item Let $ y = 0 $ to find the x-intercept.
    		\item Let $ x = 0 $ to find the y-intercept.
    	\end{enumerate} 
    \end{frame} 

    \begin{frame}
    	\frametitle{Example 1}
    	Graph the equation $ 4x - 3y = 12 $.
    	
    	\vone Let $ x = 0: $
    	
    	\vhalf
    	
    	\begin{TAB}(@, 1mm, 5mm)[2mm]{ll}{cccc}
    		\pause $ 4(0) - 3y = 12 $ & \pause Substitution Property \\
    		
    		\pause $ 0 - 3y = 12 $ & \pause Simplification\\
    		
    		\pause  $ \dfrac{-3y}{-3} = \dfrac{12}{-3} $ & \pause Division Property\\
    		
    		\pause  $ y = -4 $ & Simplification \\
    	\end{TAB}
    	
    	\pause $ \therefore $ the y-intercept $ b $ is $ -4 $ and the point is $ (0, -4) $.
    \end{frame}
    

    \begin{frame}
    	\frametitle{How to Graph Linear Equations Using the x- and y-intercepts?}
    	\begin{enumerate}  
    		\item Let $ y = 0 $ to find the x-intercept.
    		\item Let $ x = 0 $ to find the y-intercept.
    		\item Plot the two points and connect them.
    	\end{enumerate} 
    \end{frame} 

    \begin{frame}
    	\frametitle{Example 1}
    	Plot $ (3, 0) $ and $ (0, -4) $.
    	\plotit{graph3}{0.4}{8}
    \end{frame}

   \begin{frame}
   	\frametitle{Example 2}
   	Graph the equation $ y = 2x + 1 $.
   \end{frame} 
   
   \begin{frame}
   	\frametitle{How to Graph Linear Equations Using the x- and y-intercepts?}
   	\begin{enumerate}  
   		\item Let $ y = 0 $ to find the x-intercept.
   	\end{enumerate} 
   \end{frame} 
   
   \begin{frame}
   	\frametitle{Example 2}
   	Graph the equation $ y = 2x + 1 $.
   	
   	\vone Let $ y = 0: $
   	
   	\vhalf
   	
   	\begin{TAB}(@, 1mm, 5mm)[2mm]{ll}{cccc}
   		\pause $ 0 = 2x + 1 $ & \pause Substitution Property \\
   		
   		\pause $ -2x = 1 $ & \pause Subtraction Property \\
   		   		
   		\pause  $ \dfrac{-2x}{-2} = \dfrac{1}{-2} $ & \pause Division Property\\
   		
   		\pause  $ x = -\dfrac{1}{2} $ & Simplification \\
   	\end{TAB}
   	
   	\pause $ \therefore a $ is $ -\dfrac{1}{2} $ and the point is $ \left(-\dfrac{1}{2}, 0 \right) $.
   \end{frame}
   
   \begin{frame}
   	\frametitle{How to Graph Linear Equations Using the x- and y-intercepts?}
   	\begin{enumerate}  
   		\item Let $ y = 0 $ to find the x-intercept.
   		\item Let $ x = 0 $ to find the y-intercept.
   	\end{enumerate} 
   \end{frame} 
   
   \begin{frame}
   	\frametitle{Example 1}
   	Graph the equation $ y = 2x + 1 $.
   	
   	\vone Let $ x = 0: $
   	
   	\vhalf
   	
   	\begin{TAB}(@, 1mm, 5mm)[2mm]{ll}{ccc}
   		\pause $ y = 2(0) + 1 $ & \pause Substitution Property \\
   		
   		\pause $ y = 0 + 1 $ & \pause Simplification\\
   		
   		\pause  $ y = 1 $ & Simplification \\
   	\end{TAB}
   	
   	\pause $ \therefore $ the y-intercept $ b $ is $ 1 $ and the point is $ (0, 1) $.
   \end{frame}
   
   
   \begin{frame}
   	\frametitle{How to Graph Linear Equations Using the x- and y-intercepts?}
   	\begin{enumerate}  
   		\item Let $ y = 0 $ to find the x-intercept.
   		\item Let $ x = 0 $ to find the y-intercept.
   		\item Plot the two points and connect them.
   	\end{enumerate} 
   \end{frame} 
   
   \begin{frame}
   	\frametitle{Example 1}
   	Plot $ \left(-\dfrac{1}{2}, 0 \right) $ and $ (0, 1) $.
   	\plotit{graph4}{1}{3}
   \end{frame}

    \begin{frame}
    	\frametitle{How to Graph Linear Equations Using the Slope and a Point?}
    	\begin{enumerate}  
    		\item Plot the given point.
    		\item \pause Use the slope to get the other point.
    		\item \pause Connect the two points.
    	\end{enumerate} 
    \end{frame} 

    \begin{frame}
    	\frametitle{Example 1}
    	Graph the linear equation given the point $ (1, 3) $ and the slope $ \dfrac{2}{3} $.
    \end{frame} 

    \begin{frame}
    	\frametitle{How to Graph Linear Equations Using the Slope and a Point?}
    	\begin{enumerate}  
    		\item Plot the given point.
    	\end{enumerate} 
    \end{frame} 

    \begin{frame}
    	\frametitle{Example 1}
    	Plot $ (1, 3) $.
    	\plotit{graph5}{0.4}{8}
    \end{frame}

    \begin{frame}
    	\frametitle{How to Graph Linear Equations Using the Slope and a Point?}
    	\begin{enumerate}  
    		\item Plot the given point.
    		\item Use the slope to get the other point.
    	\end{enumerate} 
    \end{frame} 

    \begin{frame}
    	\frametitle{Example 1}
    	Slope $m = \dfrac{\text{rise}}{\text{run}} = \dfrac{2}{3} $
    	\plotit{graph6}{0.4}{8}
    \end{frame}

    \begin{frame}
    	\frametitle{How to Graph Linear Equations Using the Slope and a Point?}
    	\begin{enumerate}  
    		\item Plot the given point.
    		\item Use the slope to get the other point.
    		\item Connect the two points.
    	\end{enumerate} 
    \end{frame} 

    \begin{frame}
    	\frametitle{Example 1}
    	Slope $m = \dfrac{\text{rise}}{\text{run}} = \dfrac{2}{3} $
    	\plotit{graph7}{0.4}{8}
    \end{frame}
 
    \begin{frame}
    	\frametitle{Example 2}
    	Graph the linear equation given the point $ (-2, 2) $ and the slope $ -3 $.
    \end{frame} 
    
    \begin{frame}
    	\frametitle{How to Graph Linear Equations Using the Slope and a Point?}
    	\begin{enumerate}  
    		\item Plot the given point.
    	\end{enumerate} 
    \end{frame} 
    
    \begin{frame}
    	\frametitle{Example 2}
    	Plot $ (-2, 2) $.
    	\plotit{graph8}{0.4}{8}
    \end{frame}
    
    \begin{frame}
    	\frametitle{How to Graph Linear Equations Using the Slope and a Point?}
    	\begin{enumerate}  
    		\item Plot the given point.
    		\item Use the slope to get the other point.
    	\end{enumerate} 
    \end{frame} 
    
    \begin{frame}
    	\frametitle{Example 2}
    	Slope $m = \dfrac{\text{rise}}{\text{run}} = -3 = \dfrac{-3}{1} $
    	\plotit{graph9}{0.4}{8}
    \end{frame}
    
    \begin{frame}
    	\frametitle{How to Graph Linear Equations Using the Slope and a Point?}
    	\begin{enumerate}  
    		\item Plot the given point.
    		\item Use the slope to get the other point.
    		\item Connect the two points.
    	\end{enumerate} 
    \end{frame} 
    
    \begin{frame}
    	\frametitle{Example2}
    	Slope $m = \dfrac{\text{rise}}{\text{run}} = -3 = \dfrac{-3}{1} $
    	\plotit{graph10}{0.4}{8}
    \end{frame} 
 
    \begin{frame}
    	\begin{center}
    		\textbf{\LARGE Thank you for watching.}
    	\end{center}
    \end{frame}
	
\end{document}