\documentclass[14pt]{beamer}
\usepackage[utf8]{inputenc}
\usepackage{bookman} %font style
\usepackage{multicol}
\usepackage{cancel}
\usepackage{tikz}
\usepackage{xcolor} % for font color
\usetikzlibrary{calc}
\usetikzlibrary{positioning}
\usetikzlibrary{arrows.meta}
\usetikzlibrary{angles,quotes}
%\usetikzlibrary{decorations.pathreplacing}
\usepackage{wasysym} %for checked symbol 
\usepackage{tabularx} 
\usepackage{color, colortbl} %coloring table cells
\usepackage{gensymb} %degree symbol
\usepackage{amsfonts} %integer symbol
\usepackage{pgfplots} %graphs
\usepackage{easytable} % TAB
\usepackage{siunitx} % celsius symbol
\usepackage{stackengine} %to define \pesos 
\usepackage{systeme} % system of equations
\usepackage{pifont} % cross mark

\newcommand\pesos{\stackengine{-1.1ex}{P}{\stackengine{-1ex}{$-$}{$-$}{O}{c}{F}{F}{S}}{O}{c}{F}{T}{S}} 

\definecolor{Lgray}{gray}{0.9}
\definecolor{Dgray}{gray}{0.6}

\newcolumntype{Y}{>{\centering\arraybackslash}X} %for tabularx

\makeatletter
\newlength\beamerleftmargin
\setlength\beamerleftmargin{\Gm@lmargin}
\makeatother

\newcommand{\void}{\text{\hspace{2em}}}

\newcommand{\minivoid}{\text{\hspace{1em}}}

\newcommand{\sol}[3]{\begin{tikzpicture}[remember picture, overlay]
		\color{red}
        \node at (current page.south west) (sw) {};
		\node[anchor=south west, inner sep=0pt] at ($(sw) + (#1, #2) $)   (a) {
			\begin{minipage}[t]{0.75\textwidth}
				#3
			\end{minipage}};
    \end{tikzpicture}}


\newcommand \redcheck {{\color{red}\checkmark}}

\newcommand \redcross {{\color{red}\ding{56}}}

\newcommand{\vone}{\vspace{1em}}

\newcommand{\vhalf}{\vspace{0.5em}}

\newcommand{\mygrid}[2]{\begin{center}
		\begin{tikzpicture}[scale=#1]		
			\draw [help lines] (-10, -10) grid (10, 10);
			\draw[line width=0.5mm, <->, >={Latex[round]}] (-10, 0) -- (10, 0);
			\draw[line width=0.5mm, <->, >={Latex[round]}] (0, -10) -- (0, 10);	
			\input{#2}
		\end{tikzpicture}  
	\end{center}} 

\newcommand{\arrowcomment}[9]{\begin{tikzpicture}[remember picture, overlay]
		\node at (current page.south west) (sw) {};
		\node[anchor=south west, inner sep=0pt] at ($(sw) + (0, 0) $)   (a) {
			\begin{tikzpicture}[->,>=stealth, thick, main node/.style={rectangle,font=\sffamily\bfseries}, remember picture, overlay]

				\node (1)  at (#1, #2) {};

				\node[main node] (2)  at (#3, #4) {#5};

				\draw [->, red] (1.#6) to [out=#8,in=#9] (2.#7);

			\end{tikzpicture}
		};
\end{tikzpicture}} 

\newcommand{\plotit}[3]{\begin{center}
		\begin{tikzpicture}[scale=#2, main node/.style={rectangle,font=\sffamily\bfseries}]		
			\draw [help lines] (-#3, -#3) grid (#3, #3);
			\draw[line width=0.5mm, <->, >={Latex[round]}] (-#3, 0) -- (#3, 0);
			\draw[line width=0.5mm, <->, >={Latex[round]}] (0, -#3) -- (0, #3);	
			\input{#1}
		\end{tikzpicture}  
\end{center}} 

\newcommand{\plotpoint}[7]{\begin{center}
		\begin{tikzpicture}[scale=#7]		
			\coordinate (a) at (#1, #2);
			\draw [help lines] (-#5, -#5) grid (#5, #5);
			\draw[line width=#6 mm, <->, >={Latex[round]}] (-#5, 0) -- (#5, 0);
			\draw[line width=#6 mm, <->, >={Latex[round]}] (0, -#5) -- (0, #5);	
			\fill [fill=black] (a) circle (#3 pt);
			\node[anchor=#4, inner sep=2pt, rotate=0] (a-label) at (a) {$(#1, #2)$};
			\end{tikzpicture}  
	\end{center}} 

\newcommand{\plotoverlay}[5]{\begin{tikzpicture}[remember picture, overlay]
		\node at (current page.south west) (sw) {};
		\node[anchor=south west, inner sep=0pt] at ($(sw) + (#1, #2) $)   (a) {
			\begin{minipage}[t]{0.75\textwidth}
				 \begin{tikzpicture}[scale=#4, remember picture, overlay]		

						\draw [help lines] (-#5, -#5) grid (#5, #5);

						\draw[line width=0.5mm, <->, >={Latex[round]}] (-#5, 0) -- (#5, 0);

						\draw[line width=0.5mm, <->, >={Latex[round]}] (0, -#5) -- (0, #5);	

						\input{#3}

				\end{tikzpicture}
		\end{minipage}};
\end{tikzpicture}}

\newcommand{\lcmthreebythree}[9]{\begin{tikzpicture}[remember picture, overlay]
		\color{red}
		\node at (current page.south west) (sw) {};
		\node[anchor=south west, inner sep=0pt] at ($(sw) + (#1, #2) $)   (a) {
			\begin{minipage}[t]{0.75\textwidth}
				\pause Find the LCM: \\

				\begin{tabular}{rclll}

					$ \pause #3 $ & $=$ & $ \pause #6 $ & & \\

					$ \pause #4 $ & $=$ &  & $ \pause #7$ & \\

					$ \pause #5  $ & $=$ &  & & $ \pause #8 $ \\

					\hline

					\pause LCM & $=$ & \pause $ (#6) $ & \pause $ (#7) $ & \pause $ (#8) \pause = #9 $  \\

				\end{tabular}
		\end{minipage}};
\end{tikzpicture}}


\newcommand{\lcmtwobytwolineone}[3]{
	\def \lcmtwobytwolineonenumber {#1}
	\def \lcmtwobytwolineonefactorone {#2}
	\def \lcmtwobytwolineonefactortwo {#3}}

\newcommand{\lcmtwobytwolinetwo}[3]{
	\def \lcmtwobytwolinetwonumber {#1}
	\def \lcmtwobytwolinetwofactorone {#2}
	\def \lcmtwobytwolinetwofactortwo {#3}
%	\lcmtwobytwolineone
}

\newcommand{\lcmtwobytwolinethree}[3]{
	\def \lcmtwobytwolinethreelcm {#3}
	\def \lcmtwobytwolinethreefactorone {#1}
	\def \lcmtwobytwolinethreefactortwo {#2}
%	\lcmtwobytwolinetwo
}

\newcommand{\lcmtwobytwo}[2]{\begin{tikzpicture}[remember picture, overlay]
%		\lcmtwobytwolinethree
		\color{red}
		\node at (current page.south west) (sw) {};
		\node[anchor=south west, inner sep=0pt] at ($(sw) + (#1, #2) $)   (a) {
			\begin{minipage}[t]{0.75\textwidth}
				\pause Find the LCM: 
				
				\begin{tabular}{rccccc}

					$ \pause \lcmtwobytwolineonenumber $ & $=$ & $ \pause \lcmtwobytwolineonefactorone $ & $ \pause \lcmtwobytwolineonefactortwo $ & & \\

					$ \pause \lcmtwobytwolinetwonumber $ & $=$ & $ \pause \lcmtwobytwolinetwofactorone $ & $ \pause \lcmtwobytwolinetwofactortwo $ & & \\

					\hline

					\pause LCM & $=$ & \pause $ (\lcmtwobytwolinethreefactorone) $ & \pause $ (\lcmtwobytwolinethreefactortwo) $ & $=$ & \pause $ \lcmtwobytwolinethreelcm $  \\

				\end{tabular}
		\end{minipage}};
\end{tikzpicture}}


\newcommand{\lcmtwobythreelineone}[4]{
	\def \lcmtwobythreelineonenumber {#1}
	\def \lcmtwobythreelineonefactorone {#2}
	\def \lcmtwobythreelineonefactortwo {#3}
	\def \lcmtwobythreelineonefactorthree {#4}}

\newcommand{\lcmtwobythreelinetwo}[4]{
	\def \lcmtwobythreelinetwonumber {#1}
	\def \lcmtwobythreelinetwofactorone {#2}
	\def \lcmtwobythreelinetwofactortwo {#3}
	\def \lcmtwobythreelinetwofactorthree {#4}
}

\newcommand{\lcmtwobythreelinethree}[4]{
	\def \lcmtwobythreelinethreefactorone {#1}
	\def \lcmtwobythreelinethreefactortwo {#2}
	\def \lcmtwobythreelinethreefactorthree {#3}
	\def \lcmtwobythreelinethreelcm {#4}
}

\newcommand{\lcmtwobythree}[2]{\begin{tikzpicture}[remember picture, overlay]
		\color{red}
		\node at (current page.south west) (sw) {};
		\node[anchor=south west, inner sep=0pt] at ($(sw) + (#1, #2) $)   (a) {
			\begin{minipage}[t]{0.75\textwidth}
				\pause Find the LCM: 
				
				\begin{tabular}{rcccccc}
					
					$ \pause \lcmtwobythreelineonenumber $ & $=$ & $ \pause \lcmtwobythreelineonefactorone $ & $ \pause \lcmtwobythreelineonefactortwo $ & $ \pause \lcmtwobythreelineonefactorthree $ & & \\
					
					$ \pause \lcmtwobythreelinetwonumber $ & $=$ & $ \pause \lcmtwobythreelinetwofactorone $ & $ \pause \lcmtwobythreelinetwofactortwo $ & $ \pause \lcmtwobythreelinetwofactorthree $ & & \\
					
					\hline
					
					\pause LCM & $=$ & \pause $ (\lcmtwobythreelinethreefactorone) $ & \pause $ (\lcmtwobythreelinethreefactortwo) $ & \pause $ (\lcmtwobythreelinethreefactorthree) $ & $=$ & \pause $ \lcmtwobythreelinethreelcm $  \\
					
				\end{tabular}
		\end{minipage}};
\end{tikzpicture}}

\newcommand{\plotsystvars}[8]{
	\def \eqone {#1}
	\def \eqtwo {#2}
	\def \solx {#3}
	\def \soly {#4}
	\def \solanchor {#5}
	\def \labelxshift {#6}
	\def \labelyshift {#7}
	\def \solmarksize {#8}
}

\newcommand{\plotsyst}[7]{
\begin{tikzpicture}[scale=#1]
	
	\begin{axis} 
		[
		xticklabels={}, 
		yticklabels={}, 
		ymin=-#2, ymax=#2,
		xmin=-#2, xmax=#2,
		axis lines = center, 
		inner axis line style={Latex-Latex,very thick}, 
		grid=both,
		minor tick num=#7, 
		tick align=inside,
		after end axis/.code={
			\path (axis cs: \solx,\soly) 
			node [anchor=\solanchor, xshift=\labelxshift pt, yshift=\labelyshift pt] {$ (\solx, \soly) $}; } 
		] 
		
		\addplot[<->, >={Latex[round]},  ultra thick, domain=#3:#4, samples=200]{\eqone}node[]{};
		
		\addplot[<->, >={Latex[round]},  ultra thick, domain=#5:#6, samples=200]{\eqtwo}node[]{};
			
		\pause \addplot[only marks, mark=*, mark size=\solmarksize pt, color=black,] coordinates {(\solx, \soly)};
	\end{axis} 

\end{tikzpicture} 
}
\usetheme{default}
\usecolortheme{seahorse}


\title[] {Multiplication and Division of Rational Algebraic Expressions}
\author{Jonathan R. Bacolod}
\institute[SHS]{Sauyo High School}
\date{}

\begin{document}
	\frame{\titlepage}
	
	\begin{frame}
		\frametitle{How to Multiply Rational Algebraic Expressions?}
		If $ m $, $ n $, $ p $, and $ q $ are polynomials, such that $ n \neq 0 $ and $ q \neq 0 $, then \[ \dfrac{m}{n} \cdot \dfrac{p}{q} = \dfrac{mp}{nq} \]
	\end{frame}

    \begin{frame}
    	\frametitle{How to Multiply Rational Algebraic Expressions?}
    	\begin{enumerate}
    		\item<1-> Rewrite each numerator and denominator in prime factored form.
    		\item<2-> Divide by the common factors.
    		\item<3-> Multiply the remaining numerators and denominators.
       	\end{enumerate}
    \end{frame}
    
    \begin{frame}
    	\frametitle{Example 1}
    	Multiply $\dfrac{x^2 - 4}{2} \cdot \dfrac{4x}{x + 2}$
    \end{frame}

    \begin{frame}
    	\frametitle{How to Multiply Rational Algebraic Expressions?}
    	\begin{enumerate}
    		\item Rewrite each numerator and denominator in prime factored form.
    	\end{enumerate}
    \end{frame}

    \begin{frame}
    	\frametitle{Example 1}
    	$\dfrac{x^2 - 4}{2} \cdot \dfrac{4x}{x + 2}$
    \end{frame}

    \begin{frame}
    	\frametitle{To Factor a Binomial:}
    	\begin{enumerate}
    		\item<1-> Factor out the GCMF.
    		\item<2-> Check if it is a difference of two squares.
    		\item<3-> Check if it is a sum or difference of two cubes.
    	\end{enumerate}
    \end{frame}

    \begin{frame}
	    \frametitle{Example 1}
	    \begin{tabular}{ll}
		$\dfrac{x^2 - 4}{2} \cdot \dfrac{4x}{x + 2}$ & \pause $ = \dfrac{(\phantom{x + 2} ) ( \phantom{x - 2})}{} \cdot \dfrac{}{\phantom{x + 2}}$ \\
	    \end{tabular}
    \end{frame}

    \begin{frame}
    	\frametitle{Example 1}
    	\begin{tabular}{ll}
    		$\dfrac{x^2 - 4}{2} \cdot \dfrac{4x}{x + 2}$ & $ = \dfrac{(x \phantom{ + 2} ) ( x \phantom{ - 2})}{} \cdot \dfrac{}{\phantom{x + 2}}$ \\
    	\end{tabular}
    \end{frame}

    \begin{frame}
    	\frametitle{Example 1}
    	\begin{tabular}{ll}
    		$\dfrac{x^2 - 4}{2} \cdot \dfrac{4x}{x + 2}$ & $ = \dfrac{(x \phantom{ + } 2 ) ( x \phantom{ - } 2)}{} \cdot \dfrac{}{\phantom{x + 2}}$ \\
    	\end{tabular}
    \end{frame}

    \begin{frame}
    	\frametitle{Example 1}
    	\begin{tabular}{ll}
    		$\dfrac{x^2 - 4}{2} \cdot \dfrac{4x}{x + 2}$ & $ = \dfrac{(x  + 2 ) ( x \phantom{ - } 2)}{} \cdot \dfrac{}{\phantom{x + 2}}$ \\
    	\end{tabular}
    \end{frame}

    \begin{frame}
    	\frametitle{Example 1}
    	\begin{tabular}{ll}
    		$\dfrac{x^2 - 4}{2} \cdot \dfrac{4x}{x + 2}$ & $ = \dfrac{(x  + 2 ) ( x  -  2)}{} \cdot \dfrac{}{\phantom{x + 2}}$ \\
    	\end{tabular}
    \end{frame}

    \begin{frame}
    	\frametitle{Example 1}
    	\begin{tabular}{ll}
    		$\dfrac{x^2 - 4}{2} \cdot \dfrac{4x}{x + 2}$ & $ = \dfrac{(x  + 2 ) ( x  -  2)}{2} \cdot \dfrac{}{\phantom{x + 2}}$ \\
    	\end{tabular}
    \end{frame}

   \begin{frame}
   	\frametitle{Example 1}
   	\begin{tabular}{ll}
   		$\dfrac{x^2 - 4}{2} \cdot \dfrac{4x}{x + 2}$ & $ = \dfrac{(x  + 2 ) ( x  -  2)}{2} \cdot \dfrac{(2^2)(x)}{\phantom{x + 2}}$ \\
   	\end{tabular}
   \end{frame}

     \begin{frame}
    	\frametitle{Example 1}
    	\begin{tabular}{ll}
    		$\dfrac{x^2 - 4}{2} \cdot \dfrac{4x}{x + 2}$ & $ = \dfrac{(x  + 2 ) ( x  -  2)}{2} \cdot \dfrac{(2^2)(x)}{x + 2}$ \\
    	\end{tabular}
    \end{frame}

    \begin{frame}
    	\frametitle{How to Multiply Rational Algebraic Expressions?}
    	\begin{enumerate}
    		\item Rewrite each numerator and denominator in prime factored form.
    		\item Divide by the common factors.
    	\end{enumerate}
    \end{frame}

    \begin{frame}
    	\frametitle{Example 1}
    	\begin{tabular}{ll}
    		$\dfrac{x^2 - 4}{2} \cdot \dfrac{4x}{x + 2}$ & $ = \dfrac{(x  + 2 ) ( x  -  2)}{2} \cdot \dfrac{(2^2)(x)}{x + 2}$ \\
    	\end{tabular}
    \end{frame}

    \begin{frame}
    	\frametitle{Example 1}
    	\begin{tabular}{ll}
    		$\dfrac{x^2 - 4}{2} \cdot \dfrac{4x}{x + 2}$ & $ = \dfrac{\cancelto{1}{(x  + 2 )} ( x  -  2)}{2} \cdot \dfrac{(2^2)(x)}{x + 2}$ \\
    	\end{tabular}
    \end{frame}


\begin{frame}
	\frametitle{Example 1}
	\begin{tabular}{ll}
		$\dfrac{x^2 - 4}{2} \cdot \dfrac{4x}{x + 2}$ & $ = \dfrac{\cancelto{1}{(x  + 2 )} ( x  -  2)}{2} \cdot \dfrac{(2^2)(x)}{\cancelto{1}{(x  + 2 )}}$ \\
	\end{tabular}
\end{frame}

    \begin{frame}
    	\frametitle{Example 1}
    	\begin{tabular}{ll}
    		$\dfrac{x^2 - 4}{2} \cdot \dfrac{4x}{x + 2}$ & $ = \dfrac{\cancelto{1}{(x  + 2 )} ( x  -  2)}{\cancelto{1}{2}} \cdot \dfrac{(2^2)(x)}{\cancelto{1}{(x  + 2 )}}$ \\
    	\end{tabular}
    \end{frame}


    \begin{frame}
  	\frametitle{Example 1}
  	\begin{tabular}{ll}
  		$\dfrac{x^2 - 4}{2} \cdot \dfrac{4x}{x + 2}$ & $ = \dfrac{\cancelto{1}{(x  + 2 )} ( x  -  2)}{\cancelto{1}{2}} \cdot \dfrac{\cancelto{2}{(2^2)}(x)}{\cancelto{1}{(x  + 2 )}}$ \\
  	\end{tabular}
  \end{frame}

    \begin{frame}
    	\frametitle{How to Multiply Rational Algebraic Expressions?}
    	\begin{enumerate}
    		\item Rewrite each numerator and denominator in prime factored form.
    		\item Divide by the common factors.
    		\item Multiply the remaining numerators and denominators.
    	\end{enumerate}
    \end{frame}

    \begin{frame}
    	\frametitle{Example 1}
    	\begin{tabular}{ll}
    		$\dfrac{x^2 - 4}{2} \cdot \dfrac{4x}{x + 2}$ & $ = \dfrac{\cancelto{1}{(x  + 2 )} ( x  -  2)}{\cancelto{1}{2}} \cdot \dfrac{\cancelto{2}{(2^2)}(x)}{\cancelto{1}{(x  + 2 )}}$ \\[2em]
    		&  \pause $= \dfrac{2x (x - 2)}{} $ \\
    	\end{tabular}
    \end{frame}

    \begin{frame}
    	\frametitle{Example 1}
    	\begin{tabular}{ll}
    		$\dfrac{x^2 - 4}{2} \cdot \dfrac{4x}{x + 2}$ & $ = \dfrac{\cancelto{1}{(x  + 2 )} ( x  -  2)}{\cancelto{1}{2}} \cdot \dfrac{\cancelto{2}{(2^2)}(x)}{\cancelto{1}{(x  + 2 )}}$ \\[2em]
    		   &  $= \dfrac{2x (x - 2)}{1} $ \\[2em]
    	\pause	$\dfrac{x^2 - 4}{2} \cdot \dfrac{4x}{x + 2}$   &  $ = 2x(x - 2) $
    	\end{tabular}
    \end{frame}

    \begin{frame}
    	\frametitle{Example 2}
    	Multiply $\dfrac{x + 1}{3x^2 - 15x} \cdot \dfrac{8x -80}{x^2 - 9x - 10}$
    \end{frame}

    \begin{frame}
    	\frametitle{How to Multiply Rational Algebraic Expressions?}
    	\begin{enumerate}
    		\item Rewrite each numerator and denominator in prime factored form.
    	\end{enumerate}
    \end{frame}

     \begin{frame}
    	\frametitle{Example 2}
    	$\dfrac{x + 1}{3x^2 - 15x} \cdot \dfrac{8x -80}{x^2 - 9x - 10}$
    \end{frame}

    \begin{frame}
    	\frametitle{To Factor a Binomial:}
    	\begin{enumerate}
    		\item<1-> Factor out the GCMF.
    		\item<2-> Check if it is a difference of two squares.
    		\item<3-> Check if it is a sum or difference of two cubes.
    	\end{enumerate}
    \end{frame}

    \begin{frame}
    	\frametitle{Example 2}
    	$\dfrac{x + 1}{3x^2 - 15x} \cdot \dfrac{8x -80}{x^2 - 9x - 10}$
    \end{frame}
    
    \begin{frame}
    	\frametitle{To Factor a Trinomial:}
    	\begin{enumerate}
    		\item<1-> Factor out the GCMF.
    		\item<2-> Check if it is a perfect square trinomial.
    		\item<3-> Check if it is a trinomial with 1 as leading coefficient.
    		\item<4-> Check if it is a general trinomial.
    	\end{enumerate}
    \end{frame}

     \begin{frame}
    	\frametitle{Example 2}
    		$\dfrac{x + 1}{3x^2 - 15x} \cdot \dfrac{8x -80}{x^2 - 9x - 10}$ 
    		
    		\pause \vspace{1em} $= \dfrac{x + 1}{\pause 3x \pause (x \pause -5)} \cdot \dfrac{\pause 8 \pause (x \pause -10)}{\pause (\void)(\void)}$
    \end{frame}

    \begin{frame}
    	\frametitle{Example 2}
    	$\dfrac{x + 1}{3x^2 - 15x} \cdot \dfrac{8x -80}{x^2 - 9x - 10}$ 
    	
    	\vspace{1em} $= \dfrac{x + 1}{ 3x(x -5)} \cdot \dfrac{8 (x -10)}{ (x\void)(x\void)}$
    \end{frame}

     \begin{frame}
    	\frametitle{Example 2}
    	$\dfrac{x + 1}{3x^2 - 15x} \cdot \dfrac{8x -80}{x^2 - 9x - 10}$ 
    	
    	\vspace{1em} $= \dfrac{x + 1}{ 3x(x -5)} \cdot \dfrac{8 (x -10)}{ (x -10)(x\void)}$
    \end{frame}

    \begin{frame}
    	\frametitle{Example 2}
    	$\dfrac{x + 1}{3x^2 - 15x} \cdot \dfrac{8x -80}{x^2 - 9x - 10}$ 
    	
    	\vspace{1em} $= \dfrac{x + 1}{ 3x(x -5)} \cdot \dfrac{8 (x -10)}{ (x -10)(x + 1)}$
    \end{frame}

     \begin{frame}
     	\frametitle{How to Multiply Rational Algebraic Expressions?}
     	\begin{enumerate}
     		\item Rewrite each numerator and denominator in prime factored form.
     		\item Divide by the common factors.
     	\end{enumerate}
     \end{frame}
 
     \begin{frame}
     	\frametitle{Example 2}
     	$\dfrac{x + 1}{3x^2 - 15x} \cdot \dfrac{8x -80}{x^2 - 9x - 10}$ 
     	
     	\vspace{1em} $= \dfrac{x + 1}{ 3x(x -5)} \cdot \dfrac{8 (x -10)}{ (x -10)(x + 1)}$
     \end{frame}
 
     \begin{frame}
     	\frametitle{Example 2}
     	$\dfrac{x + 1}{3x^2 - 15x} \cdot \dfrac{8x -80}{x^2 - 9x - 10}$ 
     	
     	\vspace{1em} $= \dfrac{\cancelto{1}{x + 1}}{ 3x(x -5)} \cdot \dfrac{8 (x -10)}{ (x -10)(x + 1)}$
     \end{frame}
 
     \begin{frame}
     	\frametitle{Example 2}
     	$\dfrac{x + 1}{3x^2 - 15x} \cdot \dfrac{8x -80}{x^2 - 9x - 10}$ 
     	
     	\vspace{1em} $= \dfrac{\cancelto{1}{x + 1}}{ 3x(x -5)} \cdot \dfrac{8 (x -10)}{ (x -10)\cancelto{1}{(x + 1)}}$
     \end{frame}
 
     \begin{frame}
     	\frametitle{Example 2}
     	$\dfrac{x + 1}{3x^2 - 15x} \cdot \dfrac{8x -80}{x^2 - 9x - 10}$ 
     	
     	\vspace{1em} $= \dfrac{\cancelto{1}{x + 1}}{ 3x(x -5)} \cdot \dfrac{8 \cancelto{1}{(x -10)}}{ (x -10)\cancelto{1}{(x + 1)}}$
     \end{frame}
  
      \begin{frame}
      	\frametitle{Example 2}
      	$\dfrac{x + 1}{3x^2 - 15x} \cdot \dfrac{8x -80}{x^2 - 9x - 10}$ 
      	
      	\vspace{1em} $= \dfrac{\cancelto{1}{x + 1}}{ 3x(x -5)} \cdot \dfrac{8 \cancelto{1}{(x -10)}}{ \cancelto{1}{(x -10)}\cancelto{1}{(x + 1)}}$
      \end{frame}
 
 \begin{frame}
 	\frametitle{How to Multiply Rational Algebraic Expressions?}
 	\begin{enumerate}
 		\item Rewrite each numerator and denominator in prime factored form.
 		\item Divide by the common factors.
 		\item Multiply the remaining numerators and denominators.
 	\end{enumerate}
 \end{frame}
 
     \begin{frame}
     	\frametitle{Example 2}
     	$\dfrac{x + 1}{3x^2 - 15x} \cdot \dfrac{8x -80}{x^2 - 9x - 10}$ 
     	
     	\vspace{1em} $= \dfrac{\cancelto{1}{x + 1}}{ 3x(x -5)} \cdot \dfrac{8 \cancelto{1}{(x -10)}}{ \cancelto{1}{(x -10)}\cancelto{1}{(x + 1)}}$
     	
     	\pause \vspace{1em} $= \dfrac{8}{\pause 3x(x -5)}$
     \end{frame}


    \begin{frame}
    	\frametitle{How to Divide Rational Algebraic Expressions?}
    	If $ m $, $ n $, $ p $, and $ q $ are polynomials, such that $ n \neq 0 $, $ p \neq 0 $, and $ q \neq 0 $, then \[ \dfrac{m}{n} \div \dfrac{p}{q} = \dfrac{m}{n} \cdot \dfrac{q}{p} = \dfrac{mq}{np} \]
    \end{frame}
    
    \begin{frame}
    	\frametitle{How to Divide Rational Algebraic Expressions?}
    	\begin{enumerate}
    		\item<1-> Copy the dividend.
    		\item<2-> Get the reciprocal of the divisor.
    		\item<3-> Proceed to multiplication.
    	\end{enumerate}
    \end{frame}

    \begin{frame}
    	\frametitle{Example 1}
    	Divide $\dfrac{4x}{x - 6} \div \dfrac{8x^2}{8x - 48}$ 
    \end{frame}

    \begin{frame}
    	\frametitle{To Factor a Binomial:}
    	\begin{enumerate}
    		\item<1-> Factor out the GCMF.
    		\item<2-> Check if it is a difference of two squares.
    		\item<3-> Check if it is a sum or difference of two cubes.
    	\end{enumerate}
    \end{frame}

    \begin{frame}
    	\frametitle{How to Divide Rational Algebraic Expressions?}
    	\begin{enumerate}
    		\item Copy the dividend.
    	\end{enumerate}
    \end{frame}

    \begin{frame}
    	\frametitle{Example 1}
    	$\dfrac{4x}{x - 6} \div \dfrac{8x^2}{8x - 48} $	\pause  $ = \dfrac{4x}{x - 6} $ 
    \end{frame}

    \begin{frame}
    	\frametitle{How to Divide Rational Algebraic Expressions?}
    	\begin{enumerate}
    		\item Copy the dividend.
    		\item Get the reciprocal of the divisor.
    	\end{enumerate}
    \end{frame}

    \begin{frame}
    	\frametitle{Example 1}
    	$\dfrac{4x}{x - 6} \div \dfrac{8x^2}{8x - 48} = \dfrac{4x}{x - 6} \cdot \pause \dfrac{8x - 48}{8x^2} $ 
    \end{frame}

    \begin{frame}
    	\frametitle{How to Divide Rational Algebraic Expressions?}
    	\begin{enumerate}
    		\item Copy the dividend.
    		\item Get the reciprocal of the divisor.
    		\item Proceed to multiplication.
    	\end{enumerate}
    \end{frame}

    \begin{frame}
    	\frametitle{How to Multiply Rational Algebraic Expressions?}
    	\begin{enumerate}
    		\item Rewrite each numerator and denominator in prime factored form.
    	\end{enumerate}
    \end{frame}

     \begin{frame}
    	\frametitle{Example 1}
    	\begin{tabular}{ll}
    		$\dfrac{4x}{x - 6} \div \dfrac{8x^2}{8x - 48} $ & $= \dfrac{4x}{x - 6} \cdot \dfrac{8x - 48}{8x^2} $ \\[1em]
    	
    	     	& $= \dfrac{\pause (2^2)\pause (x)}{\pause x - 6} \cdot \dfrac{\pause 2^3 \pause (x\pause  - 6)}{\pause (2^3) \pause (x^2)} $ \\
    	\end{tabular}
    \end{frame}

    \begin{frame}
    	\frametitle{How to Multiply Rational Algebraic Expressions?}
    	\begin{enumerate}
    		\item Rewrite each numerator and denominator in prime factored form.
    		\item Divide by the common factors.
    	\end{enumerate}
    \end{frame}

     \begin{frame}
    	\frametitle{Example 1}
    	\begin{tabular}{ll}
    		$\dfrac{4x}{x - 6} \div \dfrac{8x^2}{8x - 48} $ & $= \dfrac{4x}{x - 6} \cdot \dfrac{8x - 48}{8x^2} $ \\[1em]
    		
    		& $= \dfrac{(2^2) (x)}{ x - 6} \cdot \dfrac{ \cancelto{1}{2^3}  (x  - 6)}{ (2^3)  (x^2)} $ \\
    	\end{tabular}
    \end{frame}

    \begin{frame}
   	\frametitle{Example 1}
   	\begin{tabular}{ll}
   		$\dfrac{4x}{x - 6} \div \dfrac{8x^2}{8x - 48} $ & $= \dfrac{4x}{x - 6} \cdot \dfrac{8x - 48}{8x^2} $ \\[1em]
   		
   		& $= \dfrac{(2^2) (x)}{ x - 6} \cdot \dfrac{ \cancelto{1}{2^3}  (x  - 6)}{ \cancelto{1}{(2^3)}  (x^2)} $ \\
   	\end{tabular}
   \end{frame}

    \begin{frame}
   	\frametitle{Example 1}
   	\begin{tabular}{ll}
   		$\dfrac{4x}{x - 6} \div \dfrac{8x^2}{8x - 48} $ & $= \dfrac{4x}{x - 6} \cdot \dfrac{8x - 48}{8x^2} $ \\[1em]
   		
   		& $= \dfrac{(2^2) \cancelto{1}{(x)}}{ x - 6} \cdot \dfrac{ \cancelto{1}{2^3}  (x  - 6)}{ \cancelto{1}{(2^3)}  (x^2)} $ \\
   	\end{tabular}
   \end{frame}

    \begin{frame}
    	\frametitle{Example 1}
    	\begin{tabular}{ll}
    		$\dfrac{4x}{x - 6} \div \dfrac{8x^2}{8x - 48} $ & $= \dfrac{4x}{x - 6} \cdot \dfrac{8x - 48}{8x^2} $ \\[1em]
    		
    		& $= \dfrac{(2^2) \cancelto{1}{(x)}}{ x - 6} \cdot \dfrac{ \cancelto{1}{2^3}  (x  - 6)}{ \cancelto{1}{(2^3)}  \cancelto{x}{(x^2)}} $ \\
    	\end{tabular}
    \end{frame}

    \begin{frame}
    	\frametitle{Example 1}
    	\begin{tabular}{ll}
    		$\dfrac{4x}{x - 6} \div \dfrac{8x^2}{8x - 48} $ & $= \dfrac{4x}{x - 6} \cdot \dfrac{8x - 48}{8x^2} $ \\[1em]
    		
    		& $= \dfrac{(2^2) \cancelto{1}{(x)}}{ x - 6} \cdot \dfrac{ \cancelto{1}{2^3}  \cancelto{1}{(x  - 6)}}{ \cancelto{1}{(2^3)}  \cancelto{x}{(x^2)}} $ \\
    	\end{tabular}
    \end{frame}

    \begin{frame}
    	\frametitle{Example 1}
    	\begin{tabular}{ll}
    		$\dfrac{4x}{x - 6} \div \dfrac{8x^2}{8x - 48} $ & $= \dfrac{4x}{x - 6} \cdot \dfrac{8x - 48}{8x^2} $ \\[1em]
    		
    		& $= \dfrac{(2^2) \cancelto{1}{(x)}}{\cancelto{1}{x  - 6}} \cdot \dfrac{ \cancelto{1}{2^3}  \cancelto{1}{(x  - 6)}}{ \cancelto{1}{(2^3)}  \cancelto{x}{(x^2)}} $ \\
    	\end{tabular}
    \end{frame}

    \begin{frame}
    	\frametitle{How to Multiply Rational Algebraic Expressions?}
    	\begin{enumerate}
    		\item Rewrite each numerator and denominator in prime factored form.
    		\item Divide by the common factors.
    		\item Multiply the remaining numerators and denominators.
    	\end{enumerate}
    \end{frame}

    \begin{frame}
    	\frametitle{Example 1}
    	\begin{tabular}{ll}
    		$\dfrac{4x}{x - 6} \div \dfrac{8x^2}{8x - 48} $ & $= \dfrac{4x}{x - 6} \cdot \dfrac{8x - 48}{8x^2} $ \\[1em]
    		
    		& $= \dfrac{(2^2) \cancelto{1}{(x)}}{\cancelto{1}{x  - 6}} \cdot \dfrac{ \cancelto{1}{2^3}  \cancelto{1}{(x  - 6)}}{ \cancelto{1}{(2^3)}  \cancelto{x}{(x^2)}} $ \\[1em]
    		
    		\pause $\dfrac{4x}{x - 6} \div \dfrac{8x^2}{8x - 48} $ & $= \dfrac{\pause 4}{\pause x} $ \\
    	\end{tabular}
    \end{frame}

    \begin{frame}
    	\frametitle{Example 2}
    	Divide $\dfrac{x^2 - 2x - 15}{8x + 20} \div \dfrac{2x^2 - 10x}{4x + 10}$ 
    \end{frame}

    \begin{frame}
    	\frametitle{To Factor a Binomial:}
    	\begin{enumerate}
    		\item<1-> Factor out the GCMF.
    		\item<2-> Check if it is a difference of two squares.
    		\item<3-> Check if it is a sum or difference of two cubes.
    	\end{enumerate}
    \end{frame}

   \begin{frame}
   	\frametitle{To Factor a Trinomial:}
   	\begin{enumerate}
   		\item<1-> Factor out the GCMF.
   		\item<2-> Check if it is a perfect square trinomial.
   		\item<3-> Check if it is a trinomial with 1 as leading coefficient.
   		\item<4-> Check if it is a general trinomial.
   	\end{enumerate}
   \end{frame}

     \begin{frame}
    	\frametitle{How to Divide Rational Algebraic Expressions?}
    	\begin{enumerate}
    		\item Copy the dividend.
    	\end{enumerate}
    \end{frame}

    \begin{frame}
    	\frametitle{Example 2}
    	$\dfrac{x^2 - 2x - 15}{8x + 20} \div \dfrac{2x^2 - 10x}{4x + 10}$ 
    	
    	\pause \vspace{1em} $ =  \dfrac{x^2 - 2x - 15}{8x + 20} $
    \end{frame}

 \begin{frame}
	\frametitle{How to Divide Rational Algebraic Expressions?}
	\begin{enumerate}
		\item Copy the dividend.
		\item Get the reciprocal of the divisor.
	\end{enumerate}
\end{frame}

    \begin{frame}
    	\frametitle{Example 2}
    	$\dfrac{x^2 - 2x - 15}{8x + 20} \div \dfrac{2x^2 - 10x}{4x + 10}$ 
    	
    	\vspace{1em}$ =  \dfrac{x^2 - 2x - 15}{8x + 20} \pause \cdot \dfrac{4x + 10}{2x^2 - 10x}$
    \end{frame}

     \begin{frame}
    	\frametitle{How to Divide Rational Algebraic Expressions?}
    	\begin{enumerate}
    		\item Copy the dividend.
    		\item Get the reciprocal of the divisor.
    		\item Proceed to multiplication.
    	\end{enumerate}
    \end{frame}

    \begin{frame}
    	\frametitle{How to Multiply Rational Algebraic Expressions?}
    	\begin{enumerate}
    		\item Rewrite each numerator and denominator in prime factored form.
    	\end{enumerate}
    \end{frame}

    \begin{frame}
    	\frametitle{Example 2}
    	$\dfrac{x^2 - 2x - 15}{8x + 20} \div \dfrac{2x^2 - 10x}{4x + 10}$ 
    	
    	\vspace{1em}$ =  \dfrac{x^2 - 2x - 15}{8x + 20} \cdot \dfrac{4x + 10}{2x^2 - 10x}$
    	
    	\pause \vspace{1em}$ =  \dfrac{(\void)(\void)}{\void} $
    \end{frame}

    \begin{frame}
    	\frametitle{Example 2}
    	$\dfrac{x^2 - 2x - 15}{8x + 20} \div \dfrac{2x^2 - 10x}{4x + 10}$ 
    	
    	\vspace{1em}$ =  \dfrac{x^2 - 2x - 15}{8x + 20} \cdot \dfrac{4x + 10}{2x^2 - 10x}$
    	
    	\vspace{1em}$ =  \dfrac{(x\void)(x\void)}{\void} $
    \end{frame}

    \begin{frame}
    	\frametitle{Example 2}
    	$\dfrac{x^2 - 2x - 15}{8x + 20} \div \dfrac{2x^2 - 10x}{4x + 10}$ 
    	
    	\vspace{1em}$ =  \dfrac{x^2 - 2x - 15}{8x + 20} \cdot \dfrac{4x + 10}{2x^2 - 10x}$
    	
    	\vspace{1em}$ =  \dfrac{(x-5)(x\void)}{\void} $
    \end{frame}

    \begin{frame}
    	\frametitle{Example 2}
    	$\dfrac{x^2 - 2x - 15}{8x + 20} \div \dfrac{2x^2 - 10x}{4x + 10}$ 
    	
    	\vspace{1em}$ =  \dfrac{x^2 - 2x - 15}{8x + 20} \cdot \dfrac{4x + 10}{2x^2 - 10x}$
    	
    	\vspace{1em}$ =  \dfrac{(x-5)(x + 3)}{\pause 4 \pause (2x \pause + 5)} \cdot \dfrac{\pause 2 \pause (2x \pause + 5)}{\pause 2x \pause (x \pause -5)}  $
    \end{frame}

    \begin{frame}
    	\frametitle{Example 2}
    	$\dfrac{x^2 - 2x - 15}{8x + 20} \div \dfrac{2x^2 - 10x}{4x + 10}$ 
    	
    	\vspace{1em}$ =  \dfrac{x^2 - 2x - 15}{8x + 20} \cdot \dfrac{4x + 10}{2x^2 - 10x}$
    	
    	\vspace{1em}$ =  \dfrac{(x-5)(x + 3)}{(2^2) (2x + 5)} \cdot \dfrac{2 (2x + 5)}{ 2x (x  -5)}  $
    \end{frame}

    \begin{frame}
    	\frametitle{How to Multiply Rational Algebraic Expressions?}
    	\begin{enumerate}
    		\item Rewrite each numerator and denominator in prime factored form.
    		\item Divide by the common factors.
    	\end{enumerate}
    \end{frame}

    \begin{frame}
    	\frametitle{Example 2}
    	$\dfrac{x^2 - 2x - 15}{8x + 20} \div \dfrac{2x^2 - 10x}{4x + 10}$ 
    	
    	\vspace{1em}$ =  \dfrac{x^2 - 2x - 15}{8x + 20} \cdot \dfrac{4x + 10}{2x^2 - 10x}$
    	
    	\vspace{1em}$ =  \dfrac{(x-5)(x + 3)}{(2^2) (2x + 5)} \cdot \dfrac{2 (2x + 5)}{ 2x (x  -5)}  $
    \end{frame}

    \begin{frame}
    	\frametitle{Example 2}
    	$\dfrac{x^2 - 2x - 15}{8x + 20} \div \dfrac{2x^2 - 10x}{4x + 10}$ 
    	
    	\vspace{1em}$ =  \dfrac{x^2 - 2x - 15}{8x + 20} \cdot \dfrac{4x + 10}{2x^2 - 10x}$
    	
    	\vspace{1em}$ =  \dfrac{\cancelto{1}{(x-5)}(x + 3)}{(2^2) (2x + 5)} \cdot \dfrac{2 (2x + 5)}{ 2x (x  -5)}  $
    \end{frame}

    \begin{frame}
    	\frametitle{Example 2}
    	$\dfrac{x^2 - 2x - 15}{8x + 20} \div \dfrac{2x^2 - 10x}{4x + 10}$ 
    	
    	\vspace{1em}$ =  \dfrac{x^2 - 2x - 15}{8x + 20} \cdot \dfrac{4x + 10}{2x^2 - 10x}$
    	
    	\vspace{1em}$ =  \dfrac{\cancelto{1}{(x-5)}(x + 3)}{(2^2) (2x + 5)} \cdot \dfrac{2 (2x + 5)}{ 2x \cancelto{1}{(x-5)}}  $
    \end{frame}

    \begin{frame}
    	\frametitle{Example 2}
    	$\dfrac{x^2 - 2x - 15}{8x + 20} \div \dfrac{2x^2 - 10x}{4x + 10}$ 
    	
    	\vspace{1em}$ =  \dfrac{x^2 - 2x - 15}{8x + 20} \cdot \dfrac{4x + 10}{2x^2 - 10x}$
    	
    	\vspace{1em}$ =  \dfrac{\cancelto{1}{(x-5)}(x + 3)}{(2^2) (2x + 5)} \cdot \dfrac{\cancelto{1}{2} (2x + 5)}{ 2x \cancelto{1}{(x-5)}}  $
    \end{frame}
    
     \begin{frame}
    	\frametitle{Example 2}
    	$\dfrac{x^2 - 2x - 15}{8x + 20} \div \dfrac{2x^2 - 10x}{4x + 10}$ 
    	
    	\vspace{1em}$ =  \dfrac{x^2 - 2x - 15}{8x + 20} \cdot \dfrac{4x + 10}{2x^2 - 10x}$
    	
    	\vspace{1em}$ =  \dfrac{\cancelto{1}{(x-5)}(x + 3)}{\cancelto{2}{(2^2)} (2x + 5)} \cdot \dfrac{\cancelto{1}{2} (2x + 5)}{ 2x \cancelto{1}{(x-5)}}  $
    \end{frame}

    \begin{frame}
    	\frametitle{Example 2}
    	$\dfrac{x^2 - 2x - 15}{8x + 20} \div \dfrac{2x^2 - 10x}{4x + 10}$ 
    	
    	\vspace{1em}$ =  \dfrac{x^2 - 2x - 15}{8x + 20} \cdot \dfrac{4x + 10}{2x^2 - 10x}$
    	
    	\vspace{1em}$ =  \dfrac{\cancelto{1}{(x-5)}(x + 3)}{\cancelto{2}{(2^2)} (2x + 5)} \cdot \dfrac{\cancelto{1}{2} \cancelto{1}{(2x + 5)}}{ 2x \cancelto{1}{(x-5)}}  $
    \end{frame}

    \begin{frame}
    	\frametitle{Example 2}
    	$\dfrac{x^2 - 2x - 15}{8x + 20} \div \dfrac{2x^2 - 10x}{4x + 10}$ 
    	
    	\vspace{1em}$ =  \dfrac{x^2 - 2x - 15}{8x + 20} \cdot \dfrac{4x + 10}{2x^2 - 10x}$
    	
    	\vspace{1em}$ =  \dfrac{\cancelto{1}{(x-5)}(x + 3)}{\cancelto{2}{(2^2)} \cancelto{1}{(2x + 5)}} \cdot \dfrac{\cancelto{1}{2} \cancelto{1}{(2x + 5)}}{ 2x \cancelto{1}{(x-5)}}  $
    \end{frame}

   \begin{frame}
   	\frametitle{How to Multiply Rational Algebraic Expressions?}
   	\begin{enumerate}
   		\item Rewrite each numerator and denominator in prime factored form.
   		\item Divide by the common factors.
   		\item Multiply the remaining numerators and denominators.
   	\end{enumerate}
   \end{frame}

    \begin{frame}
    	\frametitle{Example 2}
    	$\dfrac{x^2 - 2x - 15}{8x + 20} \div \dfrac{2x^2 - 10x}{4x + 10}$ 
    	
    	\vspace{1em}$ =  \dfrac{x^2 - 2x - 15}{8x + 20} \cdot \dfrac{4x + 10}{2x^2 - 10x}$
    	
    	\vspace{1em}$ =  \dfrac{\cancelto{1}{(x-5)}(x + 3)}{\cancelto{2}{(2^2)} \cancelto{1}{(2x + 5)}} \cdot \dfrac{\cancelto{1}{2} \cancelto{1}{(2x + 5)}}{ 2x \cancelto{1}{(x-5)}}  $
    	
    	\pause \vspace{1em}$ =  \dfrac{x + 3}{\pause 4x} $
    \end{frame}
	
    \begin{frame}
    	\begin{center}
    		\textbf{\LARGE Thank you for watching.}
    	\end{center}
    \end{frame}
	
\end{document}
