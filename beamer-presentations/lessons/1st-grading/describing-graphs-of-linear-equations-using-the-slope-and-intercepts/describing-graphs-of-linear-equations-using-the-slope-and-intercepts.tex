\documentclass[14pt]{beamer}
\usepackage[utf8]{inputenc}
\usepackage{bookman} %font style
\usepackage{multicol}
\usepackage{cancel}
\usepackage{tikz}
\usepackage{xcolor} % for font color
\usetikzlibrary{calc}
\usetikzlibrary{positioning}
\usetikzlibrary{arrows.meta}
\usetikzlibrary{angles,quotes}
%\usetikzlibrary{decorations.pathreplacing}
\usepackage{wasysym} %for checked symbol 
\usepackage{tabularx} 
\usepackage{color, colortbl} %coloring table cells
\usepackage{gensymb} %degree symbol
\usepackage{amsfonts} %integer symbol
\usepackage{pgfplots} %graphs
\usepackage{easytable} % TAB
\usepackage{siunitx} % celsius symbol
\usepackage{stackengine} %to define \pesos 
\usepackage{systeme} % system of equations
\usepackage{pifont} % cross mark

\newcommand\pesos{\stackengine{-1.1ex}{P}{\stackengine{-1ex}{$-$}{$-$}{O}{c}{F}{F}{S}}{O}{c}{F}{T}{S}} 

\definecolor{Lgray}{gray}{0.9}
\definecolor{Dgray}{gray}{0.6}

\newcolumntype{Y}{>{\centering\arraybackslash}X} %for tabularx

\makeatletter
\newlength\beamerleftmargin
\setlength\beamerleftmargin{\Gm@lmargin}
\makeatother

\newcommand{\void}{\text{\hspace{2em}}}

\newcommand{\minivoid}{\text{\hspace{1em}}}

\newcommand{\sol}[3]{\begin{tikzpicture}[remember picture, overlay]
		\color{red}
        \node at (current page.south west) (sw) {};
		\node[anchor=south west, inner sep=0pt] at ($(sw) + (#1, #2) $)   (a) {
			\begin{minipage}[t]{0.75\textwidth}
				#3
			\end{minipage}};
    \end{tikzpicture}}


\newcommand \redcheck {{\color{red}\checkmark}}

\newcommand \redcross {{\color{red}\ding{56}}}

\newcommand{\vone}{\vspace{1em}}

\newcommand{\vhalf}{\vspace{0.5em}}

\newcommand{\mygrid}[2]{\begin{center}
		\begin{tikzpicture}[scale=#1]		
			\draw [help lines] (-10, -10) grid (10, 10);
			\draw[line width=0.5mm, <->, >={Latex[round]}] (-10, 0) -- (10, 0);
			\draw[line width=0.5mm, <->, >={Latex[round]}] (0, -10) -- (0, 10);	
			\input{#2}
		\end{tikzpicture}  
	\end{center}} 

\newcommand{\arrowcomment}[9]{\begin{tikzpicture}[remember picture, overlay]
		\node at (current page.south west) (sw) {};
		\node[anchor=south west, inner sep=0pt] at ($(sw) + (0, 0) $)   (a) {
			\begin{tikzpicture}[->,>=stealth, thick, main node/.style={rectangle,font=\sffamily\bfseries}, remember picture, overlay]

				\node (1)  at (#1, #2) {};

				\node[main node] (2)  at (#3, #4) {#5};

				\draw [->, red] (1.#6) to [out=#8,in=#9] (2.#7);

			\end{tikzpicture}
		};
\end{tikzpicture}} 

\newcommand{\plotit}[3]{\begin{center}
		\begin{tikzpicture}[scale=#2, main node/.style={rectangle,font=\sffamily\bfseries}]		
			\draw [help lines] (-#3, -#3) grid (#3, #3);
			\draw[line width=0.5mm, <->, >={Latex[round]}] (-#3, 0) -- (#3, 0);
			\draw[line width=0.5mm, <->, >={Latex[round]}] (0, -#3) -- (0, #3);	
			\input{#1}
		\end{tikzpicture}  
\end{center}} 

\newcommand{\plotpoint}[7]{\begin{center}
		\begin{tikzpicture}[scale=#7]		
			\coordinate (a) at (#1, #2);
			\draw [help lines] (-#5, -#5) grid (#5, #5);
			\draw[line width=#6 mm, <->, >={Latex[round]}] (-#5, 0) -- (#5, 0);
			\draw[line width=#6 mm, <->, >={Latex[round]}] (0, -#5) -- (0, #5);	
			\fill [fill=black] (a) circle (#3 pt);
			\node[anchor=#4, inner sep=2pt, rotate=0] (a-label) at (a) {$(#1, #2)$};
			\end{tikzpicture}  
	\end{center}} 

\newcommand{\plotoverlay}[5]{\begin{tikzpicture}[remember picture, overlay]
		\node at (current page.south west) (sw) {};
		\node[anchor=south west, inner sep=0pt] at ($(sw) + (#1, #2) $)   (a) {
			\begin{minipage}[t]{0.75\textwidth}
				 \begin{tikzpicture}[scale=#4, remember picture, overlay]		

						\draw [help lines] (-#5, -#5) grid (#5, #5);

						\draw[line width=0.5mm, <->, >={Latex[round]}] (-#5, 0) -- (#5, 0);

						\draw[line width=0.5mm, <->, >={Latex[round]}] (0, -#5) -- (0, #5);	

						\input{#3}

				\end{tikzpicture}
		\end{minipage}};
\end{tikzpicture}}

\newcommand{\lcmthreebythree}[9]{\begin{tikzpicture}[remember picture, overlay]
		\color{red}
		\node at (current page.south west) (sw) {};
		\node[anchor=south west, inner sep=0pt] at ($(sw) + (#1, #2) $)   (a) {
			\begin{minipage}[t]{0.75\textwidth}
				\pause Find the LCM: \\

				\begin{tabular}{rclll}

					$ \pause #3 $ & $=$ & $ \pause #6 $ & & \\

					$ \pause #4 $ & $=$ &  & $ \pause #7$ & \\

					$ \pause #5  $ & $=$ &  & & $ \pause #8 $ \\

					\hline

					\pause LCM & $=$ & \pause $ (#6) $ & \pause $ (#7) $ & \pause $ (#8) \pause = #9 $  \\

				\end{tabular}
		\end{minipage}};
\end{tikzpicture}}


\newcommand{\lcmtwobytwolineone}[3]{
	\def \lcmtwobytwolineonenumber {#1}
	\def \lcmtwobytwolineonefactorone {#2}
	\def \lcmtwobytwolineonefactortwo {#3}}

\newcommand{\lcmtwobytwolinetwo}[3]{
	\def \lcmtwobytwolinetwonumber {#1}
	\def \lcmtwobytwolinetwofactorone {#2}
	\def \lcmtwobytwolinetwofactortwo {#3}
%	\lcmtwobytwolineone
}

\newcommand{\lcmtwobytwolinethree}[3]{
	\def \lcmtwobytwolinethreelcm {#3}
	\def \lcmtwobytwolinethreefactorone {#1}
	\def \lcmtwobytwolinethreefactortwo {#2}
%	\lcmtwobytwolinetwo
}

\newcommand{\lcmtwobytwo}[2]{\begin{tikzpicture}[remember picture, overlay]
%		\lcmtwobytwolinethree
		\color{red}
		\node at (current page.south west) (sw) {};
		\node[anchor=south west, inner sep=0pt] at ($(sw) + (#1, #2) $)   (a) {
			\begin{minipage}[t]{0.75\textwidth}
				\pause Find the LCM: 
				
				\begin{tabular}{rccccc}

					$ \pause \lcmtwobytwolineonenumber $ & $=$ & $ \pause \lcmtwobytwolineonefactorone $ & $ \pause \lcmtwobytwolineonefactortwo $ & & \\

					$ \pause \lcmtwobytwolinetwonumber $ & $=$ & $ \pause \lcmtwobytwolinetwofactorone $ & $ \pause \lcmtwobytwolinetwofactortwo $ & & \\

					\hline

					\pause LCM & $=$ & \pause $ (\lcmtwobytwolinethreefactorone) $ & \pause $ (\lcmtwobytwolinethreefactortwo) $ & $=$ & \pause $ \lcmtwobytwolinethreelcm $  \\

				\end{tabular}
		\end{minipage}};
\end{tikzpicture}}


\newcommand{\lcmtwobythreelineone}[4]{
	\def \lcmtwobythreelineonenumber {#1}
	\def \lcmtwobythreelineonefactorone {#2}
	\def \lcmtwobythreelineonefactortwo {#3}
	\def \lcmtwobythreelineonefactorthree {#4}}

\newcommand{\lcmtwobythreelinetwo}[4]{
	\def \lcmtwobythreelinetwonumber {#1}
	\def \lcmtwobythreelinetwofactorone {#2}
	\def \lcmtwobythreelinetwofactortwo {#3}
	\def \lcmtwobythreelinetwofactorthree {#4}
}

\newcommand{\lcmtwobythreelinethree}[4]{
	\def \lcmtwobythreelinethreefactorone {#1}
	\def \lcmtwobythreelinethreefactortwo {#2}
	\def \lcmtwobythreelinethreefactorthree {#3}
	\def \lcmtwobythreelinethreelcm {#4}
}

\newcommand{\lcmtwobythree}[2]{\begin{tikzpicture}[remember picture, overlay]
		\color{red}
		\node at (current page.south west) (sw) {};
		\node[anchor=south west, inner sep=0pt] at ($(sw) + (#1, #2) $)   (a) {
			\begin{minipage}[t]{0.75\textwidth}
				\pause Find the LCM: 
				
				\begin{tabular}{rcccccc}
					
					$ \pause \lcmtwobythreelineonenumber $ & $=$ & $ \pause \lcmtwobythreelineonefactorone $ & $ \pause \lcmtwobythreelineonefactortwo $ & $ \pause \lcmtwobythreelineonefactorthree $ & & \\
					
					$ \pause \lcmtwobythreelinetwonumber $ & $=$ & $ \pause \lcmtwobythreelinetwofactorone $ & $ \pause \lcmtwobythreelinetwofactortwo $ & $ \pause \lcmtwobythreelinetwofactorthree $ & & \\
					
					\hline
					
					\pause LCM & $=$ & \pause $ (\lcmtwobythreelinethreefactorone) $ & \pause $ (\lcmtwobythreelinethreefactortwo) $ & \pause $ (\lcmtwobythreelinethreefactorthree) $ & $=$ & \pause $ \lcmtwobythreelinethreelcm $  \\
					
				\end{tabular}
		\end{minipage}};
\end{tikzpicture}}

\newcommand{\plotsystvars}[8]{
	\def \eqone {#1}
	\def \eqtwo {#2}
	\def \solx {#3}
	\def \soly {#4}
	\def \solanchor {#5}
	\def \labelxshift {#6}
	\def \labelyshift {#7}
	\def \solmarksize {#8}
}

\newcommand{\plotsyst}[7]{
\begin{tikzpicture}[scale=#1]
	
	\begin{axis} 
		[
		xticklabels={}, 
		yticklabels={}, 
		ymin=-#2, ymax=#2,
		xmin=-#2, xmax=#2,
		axis lines = center, 
		inner axis line style={Latex-Latex,very thick}, 
		grid=both,
		minor tick num=#7, 
		tick align=inside,
		after end axis/.code={
			\path (axis cs: \solx,\soly) 
			node [anchor=\solanchor, xshift=\labelxshift pt, yshift=\labelyshift pt] {$ (\solx, \soly) $}; } 
		] 
		
		\addplot[<->, >={Latex[round]},  ultra thick, domain=#3:#4, samples=200]{\eqone}node[]{};
		
		\addplot[<->, >={Latex[round]},  ultra thick, domain=#5:#6, samples=200]{\eqtwo}node[]{};
			
		\pause \addplot[only marks, mark=*, mark size=\solmarksize pt, color=black,] coordinates {(\solx, \soly)};
	\end{axis} 

\end{tikzpicture} 
}
\usetheme{default}
\usecolortheme{seahorse}

\title[] {Describing Graphs of Linear Equations Using the Slope and Intercepts}
\author{Jonathan R. Bacolod}
\institute[SHS]{Sauyo High School}
\date{}

\begin{document}
	\frame{\titlepage}
	
	\begin{frame}
		\frametitle{How to Describe a Graph Using the Slope?}
		\begin{center}
			\begin{TAB}(@, 1mm, 5mm)[4mm]{cc}{cccccc}
			\textbf{Value/Sign of $ m $} & \textbf{Trend of Graph} \\
			
			\pause Positive & \pause Rises from left to right\\
			
			\pause  Negative & \pause Falls from left to right \\
			
			\pause  Zero & \pause  Horizontal line \\
			
			\pause  Undefined & \pause Vertical line \\
			& \\
		\end{TAB}
		\end{center}
	\end{frame}

    \begin{frame}
    	\frametitle{How to Describe a Graph Using the Slope when the Equation is Given?}
    	\begin{enumerate}  
    		\item Change the equation to the form $ y = mx + b $. $ m $ is the slope and $ b $ is the y-intercept.
    		\item \pause Describe the graph using the slope.
    	\end{enumerate}  
    \end{frame}

    \begin{frame}
    	\frametitle{Example 1}
    	Determine the slope of the linear equation $ y = 2x - 5 $ and describe the graph.
    \end{frame}

    \begin{frame}
    	\frametitle{How to Describe a Graph Using the Slope when the Equation is Given?}
    	\begin{enumerate}  
    		\item Change the equation to the form $ y = mx + b $. $ m $ is the slope and $ b $ is the y-intercept.
    	\end{enumerate}  
    \end{frame}

    \begin{frame}
    	\frametitle{Example 1}
    	    	
    	\begin{TAB}(@, 1mm, 5mm)[2mm]{ll}{cc}
    		$ y = 2x - 5 $  &  \\
    		 		
    		\pause  $ m = 2, b = -5 $ & \\
    	\end{TAB}
    	
    	\pause $ \therefore $ the slope is $ 2 $ and \\ the y-intercept is $ -5 $.
    \end{frame}

    \begin{frame}
    	\frametitle{How to Describe a Graph Using the Slope when the Equation is Given?}
    	\begin{enumerate}  
    		\item Change the equation to the form $ y = mx + b $. $ m $ is the slope and $ b $ is the y-intercept.
    		\item Describe the graph using the slope.
    	\end{enumerate}  
    \end{frame}

    \begin{frame}
    	\frametitle{Example 1}
    	
    	\begin{TAB}(@, 1mm, 5mm)[2mm]{ll}{cc}
    		$ y = 2x - 5 $  &  \\
    		
    		$ m = 2, b = -5 $ & \\
    	\end{TAB}
    	
    	$ \therefore $ the slope is $ 2 $ and \\ the y-intercept is $ -5 $.
    	
    	\vone
    	
    	\pause Since the slope is positive, \\ the graph is rising from\\ left to right.
    	
    	\pause \plotoverlay{10}{5}{graph1}{0.3}{9}
    \end{frame}

    \begin{frame}
    	\frametitle{Example 2}
    	Determine the slope of the linear equation $ 4x + 2y = 6 $ and describe the graph.
    \end{frame}
    
    \begin{frame}
    	\frametitle{How to Describe a Graph Using the Slope when the Equation is Given?}
    	\begin{enumerate}  
    		\item Change the equation to the form $ y = mx + b $. $ m $ is the slope and $ b $ is the y-intercept.
    	\end{enumerate}  
    \end{frame}
    
    \begin{frame}
    	\frametitle{Example 2}
    	
    	\begin{TAB}(@, 1mm, 5mm)[2mm]{ll}{ccccc}
    		$ 4x + 2y = 6 $  &  \\
    		
    		\pause $ 2y = -4x + 6 $  & \pause Subtraction Property \\
    		
    		\pause $ \dfrac{2y}{2} = \dfrac{-4x}{2} + \dfrac{6}{2} $  & \pause Division Property \\
    		
    		\pause $ y = -2x + 3 $  & \pause Simplification \\
    		
    		\pause  $ m = -2, b = 3 $ & \\
    	\end{TAB}
    	
    	\pause $ \therefore $ the slope is $ -2 $ and \\ the y-intercept is $ 3 $.
    \end{frame}
    
    \begin{frame}
    	\frametitle{How to Describe a Graph Using the Slope when the Equation is Given?}
    	\begin{enumerate}  
    		\item Change the equation to the form $ y = mx + b $. $ m $ is the slope and $ b $ is the y-intercept.
    		\item Describe the graph using the slope.
    	\end{enumerate}  
    \end{frame}
    
    \begin{frame}
    	\frametitle{Example 2}
    	
    	\begin{TAB}(@, 1mm, 5mm)[2mm]{ll}{cc}
    		$ y = -2x + 3 $  & \\
    		
    		$ m = -2, b = 3 $ & \\
    	\end{TAB}
    	
    	$ \therefore $ the slope is $ -2 $ and \\ the y-intercept is $ 3 $.
    	
    	\vone
    	
    	\pause Since the slope is \\ negative, the graph is \\ falling from left to right.
    	
        \pause \plotoverlay{9.6}{5}{graph2}{0.45}{6}
    \end{frame}
    
    \begin{frame}
    	\frametitle{Example 3}
    	Determine the slope of the linear equation $ -3y - 9 = 0 $ and describe the graph.
    \end{frame}
    
    \begin{frame}
    	\frametitle{How to Describe a Graph Using the Slope when the Equation is Given?}
    	\begin{enumerate}  
    		\item Change the equation to the form $ y = mx + b $. $ m $ is the slope and $ b $ is the y-intercept.
    	\end{enumerate}  
    \end{frame}
    
    \begin{frame}
    	\frametitle{Example 3}
    	
    	\begin{TAB}(@, 1mm, 5mm)[2mm]{ll}{ccccc}
    		$ -3y - 9 = 0 $  &  \\
    		
    		\pause $ -3y = 9 $  &  Addition Property \\
    		
    		\pause $ \dfrac{-3y}{-3} = \dfrac{9}{-3} $  &  Division Property \\
    		
    		\pause $ y = -3 $  &  Simplification \\
    		
    		\pause  $ m = 0, b = -3 $ & \\
    	\end{TAB}
    	
    	\pause $ \therefore $ the slope is $ 0 $ and \\ the y-intercept is $ -3 $.
    \end{frame}
    
    \begin{frame}
    	\frametitle{How to Describe a Graph Using the Slope when the Equation is Given?}
    	\begin{enumerate}  
    		\item Change the equation to the form $ y = mx + b $. $ m $ is the slope and $ b $ is the y-intercept.
    		\item Describe the graph using the slope.
    	\end{enumerate}  
    \end{frame}
    
    \begin{frame}
    	\frametitle{Example 3}
    	
    	\begin{TAB}(@, 1mm, 5mm)[2mm]{ll}{cc}
    		$ y = -3 $  &  \\
    		
    		$ m = 0, b = -3 $ & \\
    	\end{TAB}
    	
    	$ \therefore $ the slope is $ 0 $ and \\ the y-intercept is $ -3 $.
    	
    	\vone
    	
    	\pause Since the slope is $ 0 $, \\ the graph is a\\ horizontal line.
    	
    	\pause \plotoverlay{9.45}{5}{graph3}{0.6}{5}
    \end{frame}
    
    \begin{frame}
    	\frametitle{Example 4}
    	Determine the slope of the linear equation $ x = 4 $ and describe the graph.
    \end{frame}
    
    \begin{frame}
    	\frametitle{How to Describe a Graph Using the Slope when the Equation is Given?}
    	\begin{enumerate}  
    		\item Change the equation to the form $ y = mx + b $. $ m $ is the slope and $ b $ is the y-intercept.
    	\end{enumerate}  
    \end{frame}
    
    \begin{frame}
    	\frametitle{Example 4}
    	
    	\begin{TAB}(@, 1mm, 5mm)[2mm]{ll}{cc}
    		$ x = 4 $  &  \\
    		    		
    		\pause  $ m = \text{undefined}, b = \text{undefined} $ & \\
    	\end{TAB}
    	
    	\pause $ \therefore $ the slope is undefined and \\ the y-intercept is undefined.
    \end{frame}
    
    \begin{frame}
    	\frametitle{How to Describe a Graph Using the Slope when the Equation is Given?}
    	\begin{enumerate}  
    		\item Change the equation to the form $ y = mx + b $. $ m $ is the slope and $ b $ is the y-intercept.
    		\item Describe the graph using the slope.
    	\end{enumerate}  
    \end{frame}
    
    \begin{frame}
    	\frametitle{Example 4}
    	
    	\begin{TAB}(@, 1mm, 5mm)[2mm]{ll}{cc}
    		$ x = 4 $  &  \\
    		
    		$ m = \text{undefined}, b = \text{undefined} $ & \\
    	\end{TAB}
    	
    	$ \therefore $ the slope is undefined\\ and the y-intercept\\ is undefined.
    	
    	\vone
    	
    	\pause Since the slope is \\undefined, the graph \\ is a vertical line.
    	
    	\pause \plotoverlay{9.6}{3}{graph4}{0.4}{6}
    \end{frame}

    \begin{frame}
    	\frametitle{How to Describe a Graph Using the Intercepts?}
    	\begin{TAB}(@, 1mm, 5mm)[1mm]{cc}{cccccc}
    		\textbf{Signs/Values of $ a $ and $ b $} & \textbf{Trend of Graph} \\
    		
    		\pause  Same signs & \pause Falls from left to right \\
    		
    		\pause  Different signs & \pause Rises from left to right \\
    		
    		\pause  $ a = \text{undefined}, b \in \mathbb{R} $ & \pause  Horizontal line \\
    		
    		\pause  $ a \in \mathbb{R}, b = \text{undefined} $ & \pause Vertical line \\
    		& \\
    	\end{TAB}
    \end{frame}

    \begin{frame}
    	\frametitle{How to Describe a Graph Using the Intercepts when the Equation is Given?}
    	\begin{enumerate}  
    		\item Let $ y = 0 $ and solve for $ x $ to get the x-intercept $ a $.
    		\item \pause Let $ x = 0 $ and solve for $ y $ to get the y-intercept $ b $.
    		\item \pause Describe the graph using the intercepts.
    	\end{enumerate}  
    \end{frame}

    \begin{frame}
    	\frametitle{Example 1}
    	Determine the intercepts of the linear equation $ y = 2x - 6 $ and describe the graph.
    \end{frame}
    
    \begin{frame}
    	\frametitle{How to Describe a Graph Using the Intercepts when the Equation is Given?}
    	\begin{enumerate}  
    		\item Let $ y = 0 $ and solve for $ x $ to get the x-intercept $ a $.
    	\end{enumerate}  
    \end{frame}

    \begin{frame}
    	\frametitle{Example 1}
    	Let $ y = 0: $
    	
    	\begin{TAB}(@, 1mm, 5mm)[2mm]{ll}{cccccc}
    		$ y = 2x - 6 $  &  \\
    		
    		\pause $ 0 = 2x - 6 $  &  \pause  Substitution \\
    		
    		\pause $ -2x = - 6 $  &  \pause Subtraction Property \\
    		
    		\pause $ \dfrac{-2x}{-2} = \dfrac{-6}{-2} $  &  \pause Division Property \\
    		
    		\pause $ x = 3 $  &  \pause Simplification \\
    		& \\    		
    	\end{TAB}
    	
    	$ \therefore $ the x-intercept $ a $ is $ 3 $.
    \end{frame}

    \begin{frame}
    	\frametitle{How to Describe a Graph Using the Intercepts when the Equation is Given?}
    	\begin{enumerate}  
    		\item Let $ y = 0 $ and solve for $ x $ to get the x-intercept $ a $.
    		\item Let $ x = 0 $ and solve for $ y $ to get the y-intercept $ b $.
    	\end{enumerate}  
    \end{frame}

    \begin{frame}
    	\frametitle{Example 1}
    	Let $ x = 0: $
    	
    	\begin{TAB}(@, 1mm, 5mm)[2mm]{ll}{ccccc}
    		$ y = 2x - 6 $  &  \\
    		
    		\pause $ y = 2(0) - 6 $  & \pause Substitution \\
    		
    		\pause $ y = 0 - 6 $  & \pause Simplification \\
    		
    		\pause $ y = -6 $  & \pause Simplification \\
    		& \\
    	\end{TAB}
    	
    	$ \therefore $ the y-intercept $ b $ is $ -6 $.
    \end{frame}

    \begin{frame}
    	\frametitle{How to Describe a Graph Using the Intercepts when the Equation is Given?}
    	\begin{enumerate}  
    		\item Let $ y = 0 $ and solve for $ x $ to get the x-intercept $ a $.
    		\item Let $ x = 0 $ and solve for $ y $ to get the y-intercept $ b $.
    		\item Describe the graph using the intercepts.
    	\end{enumerate}  
    \end{frame}

    \begin{frame}
    	\frametitle{Example 1}
    	
    	$ y = 2x - 6 $ 
    	
    	The x-intercept $ a $ is $ 3 $ and \\ the y-intercept $ b $ is $ -6 $.
    	
    	\vone
    	
    	\pause Since the intercepts have \\ different signs, the graph \\ rises from left to right.
    	
    	\pause \plotoverlay{9.9}{5}{graph5}{0.25}{10}
    \end{frame}
    
    \begin{frame}
    	\frametitle{Example 2}
    	Determine the intercepts of the linear equation $ 4x + 2y = 8 $ and describe the graph.
    \end{frame}
    
    \begin{frame}
    	\frametitle{How to Describe a Graph Using the Intercepts when the Equation is Given?}
    	\begin{enumerate}  
    		\item Let $ y = 0 $ and solve for $ x $ to get the x-intercept $ a $.
    	\end{enumerate}  
    \end{frame}
    
    \begin{frame}
    	\frametitle{Example 2}
    	Let $ y = 0: $
    	
    	\begin{TAB}(@, 1mm, 5mm)[2mm]{ll}{cccccc}
    		$ 4x + 2y = 8 $  &  \\
    		
    		\pause $ 4x + 2(0) = 8 $  &  \pause  Substitution \\
    		
    		\pause $ 4x + 0 = 8 $  &  \pause Simplification \\
    		
    		\pause $ \dfrac{4x}{4} = \dfrac{8}{4} $  &  \pause Division Property \\
    		
    		\pause $ x = 2 $  &  \pause Simplification \\
    		& \\    		
    	\end{TAB}
    	
    	$ \therefore $ the x-intercept $ a $ is $ 2 $.
    \end{frame}
    
    \begin{frame}
    	\frametitle{How to Describe a Graph Using the Intercepts when the Equation is Given?}
    	\begin{enumerate}  
    		\item Let $ y = 0 $ and solve for $ x $ to get the x-intercept $ a $.
    		\item Let $ x = 0 $ and solve for $ y $ to get the y-intercept $ b $.
    	\end{enumerate}  
    \end{frame}
    
    \begin{frame}
    	\frametitle{Example 2}
    	Let $ x = 0: $
    	
    	\begin{TAB}(@, 1mm, 5mm)[2mm]{ll}{cccccc}
    		$ 4x + 2y = 8 $  &  \\
    		
    		\pause $ 4(0) + 2y = 8 $  & \pause Substitution \\
    		
    		\pause $ 0 + 2y = 8 $  & \pause Simplification \\
    		
    		\pause $ \dfrac{2y}{2} = \dfrac{8}{2} $  & \pause Division Property \\
    		
    		\pause $ y = 4 $  & \pause Simplification \\
    		& \\
    	\end{TAB} 
    	
    	$ \therefore $ the y-intercept $ b $ is $ 4 $.
    \end{frame}
    
    \begin{frame}
    	\frametitle{How to Describe a Graph Using the Intercepts when the Equation is Given?}
    	\begin{enumerate}  
    		\item Let $ y = 0 $ and solve for $ x $ to get the x-intercept $ a $.
    		\item Let $ x = 0 $ and solve for $ y $ to get the y-intercept $ b $.
    		\item Describe the graph using the intercepts.
    	\end{enumerate}  
    \end{frame}
    
    \begin{frame}
    	\frametitle{Example 2}
    	
    	$ 4x + 2y = 8 $ 
    	
    	The x-intercept $ a $ is $ 2 $ and \\ the y-intercept $ b $ is $ 4 $.
    	
    	\vone
    	
    	\pause Since the intercepts have \\ same signs, the graph \\ falls from left to right.
    	
    	\pause \plotoverlay{9.9}{5}{graph6}{0.25}{10}
    \end{frame}

    \begin{frame}
    	\frametitle{Example 3}
    	Determine the intercepts of the linear equation $ -3y - 9 = 0 $ and describe the graph.
    \end{frame}
    
    \begin{frame}
    	\frametitle{How to Describe a Graph Using the Intercepts when the Equation is Given?}
    	\begin{enumerate}  
    		\item Let $ y = 0 $ and solve for $ x $ to get the x-intercept $ a $.
    	\end{enumerate}  
    \end{frame}
    
    \begin{frame}
    	\frametitle{Example 3}
    	Let $ y = 0: $
    	
    	\begin{TAB}(@, 1mm, 5mm)[2mm]{ll}{ccccc}
    		$ -3y - 9 = 0 $  &  \\
    		
    		\pause $ -3(0) - 9 = 0 $  &  \pause  Substitution \\
    		
    		\pause $ 0 - 9 = 0 $  &  \pause Simplification \\
    		    		
    		\pause $ -9 \neq 0 $  &  \pause Simplification \\
    		& \\    		
    	\end{TAB}
    	
    	$ \therefore $ the x-intercept $ a $ is undefined.
    \end{frame}
    
    \begin{frame}
    	\frametitle{How to Describe a Graph Using the Intercepts when the Equation is Given?}
    	\begin{enumerate}  
    		\item Let $ y = 0 $ and solve for $ x $ to get the x-intercept $ a $.
    		\item Let $ x = 0 $ and solve for $ y $ to get the y-intercept $ b $.
    	\end{enumerate}  
    \end{frame}
    
    \begin{frame}
    	\frametitle{Example 3}
    	Let $ x = 0: $
    	
    	\begin{TAB}(@, 1mm, 5mm)[2mm]{ll}{ccccc}
    		$ -3y - 9 = 0 $  &  \\
    		
    		\pause $ -3y = 9 $  & \pause Addition Property \\
    		
    		\pause $ \dfrac{-3y}{-3} = \dfrac{9}{-3} $  & \pause Division Property \\
    		
    		\pause $ y = -3 $  & \pause Simplification \\
    		& \\
    	\end{TAB} 
    	
    	$ \therefore $ the y-intercept $ b $ is $ -3 $.
    \end{frame}
    
    \begin{frame}
    	\frametitle{How to Describe a Graph Using the Intercepts when the Equation is Given?}
    	\begin{enumerate}  
    		\item Let $ y = 0 $ and solve for $ x $ to get the x-intercept $ a $.
    		\item Let $ x = 0 $ and solve for $ y $ to get the y-intercept $ b $.
    		\item Describe the graph using the intercepts.
    	\end{enumerate}  
    \end{frame}
    
    \begin{frame}
    	\frametitle{Example 3}
    	
    	$ -3y - 9 = 0 $ 
    	
    	The x-intercept $ a $ is\\ undefined and \\ the y-intercept $ b $ is $ -3 $.
    	
    	\vone
    	
    	\pause Since the x-intercept is \\ undefined and the \\y-intercept is a \\real number,\\ the graph is a\\ horizontal line.
    	
    	\pause \plotoverlay{9.7}{5}{graph7}{0.4}{6}
    \end{frame}

    \begin{frame}
    	\frametitle{Example 4}
    	Determine the intercepts of the linear equation $ x = 4 $ and describe the graph.
    \end{frame}
    
    \begin{frame}
    	\frametitle{How to Describe a Graph Using the Intercepts when the Equation is Given?}
    	\begin{enumerate}  
    		\item Let $ y = 0 $ and solve for $ x $ to get the x-intercept $ a $.
    	\end{enumerate}  
    \end{frame}
    
    \begin{frame}
    	\frametitle{Example 4}
    	Let $ y = 0: $
    	
    	\begin{TAB}(@, 1mm, 5mm)[2mm]{ll}{cc}
    		$ x = 4 $  &  \\
    			& \\    		
    	\end{TAB}
    	
    	$ \therefore $ the x-intercept $ a $ is $ 4 $.
    \end{frame}
    
    \begin{frame}
    	\frametitle{How to Describe a Graph Using the Intercepts when the Equation is Given?}
    	\begin{enumerate}  
    		\item Let $ y = 0 $ and solve for $ x $ to get the x-intercept $ a $.
    		\item Let $ x = 0 $ and solve for $ y $ to get the y-intercept $ b $.
    	\end{enumerate}  
    \end{frame}
    
    \begin{frame}
    	\frametitle{Example 4}
    	Let $ x = 0: $
    	
    	\begin{TAB}(@, 1mm, 5mm)[2mm]{ll}{ccc}
    		$ x = 4 $  &  \\
    		
    		\pause $ 0 \neq 4 $  & \pause Substitution \\
    		& \\
    	\end{TAB} 
    	
    	$ \therefore $ the y-intercept $ b $ is undefined.
    \end{frame}
    
    \begin{frame}
    	\frametitle{How to Describe a Graph Using the Intercepts when the Equation is Given?}
    	\begin{enumerate}  
    		\item Let $ y = 0 $ and solve for $ x $ to get the x-intercept $ a $.
    		\item Let $ x = 0 $ and solve for $ y $ to get the y-intercept $ b $.
    		\item Describe the graph using the intercepts.
    	\end{enumerate}  
    \end{frame}
    
    \begin{frame}
    	\frametitle{Example 4}
    	
    	$ x = 4 $ 
    	
    	The x-intercept $ a $ is $ 4 $ \\ and the y-intercept $ b $ \\ is undefined.
    	
    	\vone
    	
    	\pause Since the x-intercept is \\ a real number and the \\y-intercept is undefined,\\ the graph is a\\ vertical line.
    	
    	\pause \plotoverlay{9.7}{5}{graph8}{0.4}{6}
    \end{frame}

    \begin{frame}
    	\frametitle{How to Describe a Graph Using the Slope?}
    	\begin{center}
    		\begin{TAB}(@, 1mm, 5mm)[4mm]{cc}{cccccc}
    			\textbf{Value/Sign of $ m $} & \textbf{Trend of Graph} \\
    			
    			\pause Positive & \pause Rises from left to right\\
    			
    			\pause  Negative & \pause Falls from left to right \\
    			
    			\pause  Zero & \pause  Horizontal line \\
    			
    			\pause  Undefined & \pause Vertical line \\
    			& \\
    		\end{TAB}
    	\end{center}
    \end{frame}
    
    \begin{frame}
    	\frametitle{How to Describe a Graph Using the Intercepts?}
    	\begin{TAB}(@, 1mm, 5mm)[1mm]{cc}{cccccc}
    		\textbf{Signs/Values of $ a $ and $ b $} & \textbf{Trend of Graph} \\
    		
    		\pause  Same signs & \pause Falls from left to right \\
    		
    		\pause  Different signs & \pause Rises from left to right \\
    		
    		\pause  $ a = \text{undefined}, b \in \mathbb{R} $ & \pause  Horizontal line \\
    		
    		\pause  $ a \in \mathbb{R}, b = \text{undefined} $ & \pause Vertical line \\
    		& \\
    	\end{TAB}
    \end{frame}
    
    \begin{frame}
    	\begin{center}
    		\textbf{\LARGE Thank you for watching.}
    	\end{center}
    \end{frame}
	
\end{document}