\documentclass[14pt]{beamer}
\usepackage[utf8]{inputenc}
\usepackage{bookman} %font style
\usepackage{multicol}
\usepackage{cancel}
\usepackage{tikz}
\usepackage{xcolor} % for font color
\usetikzlibrary{calc}
\usetikzlibrary{positioning}
\usetikzlibrary{arrows.meta}
\usetikzlibrary{angles,quotes}
%\usetikzlibrary{decorations.pathreplacing}
\usepackage{wasysym} %for checked symbol 
\usepackage{tabularx} 
\usepackage{color, colortbl} %coloring table cells
\usepackage{gensymb} %degree symbol
\usepackage{amsfonts} %integer symbol
\usepackage{pgfplots} %graphs
\usepackage{easytable} % TAB
\usepackage{siunitx} % celsius symbol
\usepackage{stackengine} %to define \pesos 
\usepackage{systeme} % system of equations
\usepackage{pifont} % cross mark

\newcommand\pesos{\stackengine{-1.1ex}{P}{\stackengine{-1ex}{$-$}{$-$}{O}{c}{F}{F}{S}}{O}{c}{F}{T}{S}} 

\definecolor{Lgray}{gray}{0.9}
\definecolor{Dgray}{gray}{0.6}

\newcolumntype{Y}{>{\centering\arraybackslash}X} %for tabularx

\makeatletter
\newlength\beamerleftmargin
\setlength\beamerleftmargin{\Gm@lmargin}
\makeatother

\newcommand{\void}{\text{\hspace{2em}}}

\newcommand{\minivoid}{\text{\hspace{1em}}}

\newcommand{\sol}[3]{\begin{tikzpicture}[remember picture, overlay]
		\color{red}
        \node at (current page.south west) (sw) {};
		\node[anchor=south west, inner sep=0pt] at ($(sw) + (#1, #2) $)   (a) {
			\begin{minipage}[t]{0.75\textwidth}
				#3
			\end{minipage}};
    \end{tikzpicture}}


\newcommand \redcheck {{\color{red}\checkmark}}

\newcommand \redcross {{\color{red}\ding{56}}}

\newcommand{\vone}{\vspace{1em}}

\newcommand{\vhalf}{\vspace{0.5em}}

\newcommand{\mygrid}[2]{\begin{center}
		\begin{tikzpicture}[scale=#1]		
			\draw [help lines] (-10, -10) grid (10, 10);
			\draw[line width=0.5mm, <->, >={Latex[round]}] (-10, 0) -- (10, 0);
			\draw[line width=0.5mm, <->, >={Latex[round]}] (0, -10) -- (0, 10);	
			\input{#2}
		\end{tikzpicture}  
	\end{center}} 

\newcommand{\arrowcomment}[9]{\begin{tikzpicture}[remember picture, overlay]
		\node at (current page.south west) (sw) {};
		\node[anchor=south west, inner sep=0pt] at ($(sw) + (0, 0) $)   (a) {
			\begin{tikzpicture}[->,>=stealth, thick, main node/.style={rectangle,font=\sffamily\bfseries}, remember picture, overlay]

				\node (1)  at (#1, #2) {};

				\node[main node] (2)  at (#3, #4) {#5};

				\draw [->, red] (1.#6) to [out=#8,in=#9] (2.#7);

			\end{tikzpicture}
		};
\end{tikzpicture}} 

\newcommand{\plotit}[3]{\begin{center}
		\begin{tikzpicture}[scale=#2, main node/.style={rectangle,font=\sffamily\bfseries}]		
			\draw [help lines] (-#3, -#3) grid (#3, #3);
			\draw[line width=0.5mm, <->, >={Latex[round]}] (-#3, 0) -- (#3, 0);
			\draw[line width=0.5mm, <->, >={Latex[round]}] (0, -#3) -- (0, #3);	
			\input{#1}
		\end{tikzpicture}  
\end{center}} 

\newcommand{\plotpoint}[7]{\begin{center}
		\begin{tikzpicture}[scale=#7]		
			\coordinate (a) at (#1, #2);
			\draw [help lines] (-#5, -#5) grid (#5, #5);
			\draw[line width=#6 mm, <->, >={Latex[round]}] (-#5, 0) -- (#5, 0);
			\draw[line width=#6 mm, <->, >={Latex[round]}] (0, -#5) -- (0, #5);	
			\fill [fill=black] (a) circle (#3 pt);
			\node[anchor=#4, inner sep=2pt, rotate=0] (a-label) at (a) {$(#1, #2)$};
			\end{tikzpicture}  
	\end{center}} 

\newcommand{\plotoverlay}[5]{\begin{tikzpicture}[remember picture, overlay]
		\node at (current page.south west) (sw) {};
		\node[anchor=south west, inner sep=0pt] at ($(sw) + (#1, #2) $)   (a) {
			\begin{minipage}[t]{0.75\textwidth}
				 \begin{tikzpicture}[scale=#4, remember picture, overlay]		

						\draw [help lines] (-#5, -#5) grid (#5, #5);

						\draw[line width=0.5mm, <->, >={Latex[round]}] (-#5, 0) -- (#5, 0);

						\draw[line width=0.5mm, <->, >={Latex[round]}] (0, -#5) -- (0, #5);	

						\input{#3}

				\end{tikzpicture}
		\end{minipage}};
\end{tikzpicture}}

\newcommand{\lcmthreebythree}[9]{\begin{tikzpicture}[remember picture, overlay]
		\color{red}
		\node at (current page.south west) (sw) {};
		\node[anchor=south west, inner sep=0pt] at ($(sw) + (#1, #2) $)   (a) {
			\begin{minipage}[t]{0.75\textwidth}
				\pause Find the LCM: \\

				\begin{tabular}{rclll}

					$ \pause #3 $ & $=$ & $ \pause #6 $ & & \\

					$ \pause #4 $ & $=$ &  & $ \pause #7$ & \\

					$ \pause #5  $ & $=$ &  & & $ \pause #8 $ \\

					\hline

					\pause LCM & $=$ & \pause $ (#6) $ & \pause $ (#7) $ & \pause $ (#8) \pause = #9 $  \\

				\end{tabular}
		\end{minipage}};
\end{tikzpicture}}


\newcommand{\lcmtwobytwolineone}[3]{
	\def \lcmtwobytwolineonenumber {#1}
	\def \lcmtwobytwolineonefactorone {#2}
	\def \lcmtwobytwolineonefactortwo {#3}}

\newcommand{\lcmtwobytwolinetwo}[3]{
	\def \lcmtwobytwolinetwonumber {#1}
	\def \lcmtwobytwolinetwofactorone {#2}
	\def \lcmtwobytwolinetwofactortwo {#3}
%	\lcmtwobytwolineone
}

\newcommand{\lcmtwobytwolinethree}[3]{
	\def \lcmtwobytwolinethreelcm {#3}
	\def \lcmtwobytwolinethreefactorone {#1}
	\def \lcmtwobytwolinethreefactortwo {#2}
%	\lcmtwobytwolinetwo
}

\newcommand{\lcmtwobytwo}[2]{\begin{tikzpicture}[remember picture, overlay]
%		\lcmtwobytwolinethree
		\color{red}
		\node at (current page.south west) (sw) {};
		\node[anchor=south west, inner sep=0pt] at ($(sw) + (#1, #2) $)   (a) {
			\begin{minipage}[t]{0.75\textwidth}
				\pause Find the LCM: 
				
				\begin{tabular}{rccccc}

					$ \pause \lcmtwobytwolineonenumber $ & $=$ & $ \pause \lcmtwobytwolineonefactorone $ & $ \pause \lcmtwobytwolineonefactortwo $ & & \\

					$ \pause \lcmtwobytwolinetwonumber $ & $=$ & $ \pause \lcmtwobytwolinetwofactorone $ & $ \pause \lcmtwobytwolinetwofactortwo $ & & \\

					\hline

					\pause LCM & $=$ & \pause $ (\lcmtwobytwolinethreefactorone) $ & \pause $ (\lcmtwobytwolinethreefactortwo) $ & $=$ & \pause $ \lcmtwobytwolinethreelcm $  \\

				\end{tabular}
		\end{minipage}};
\end{tikzpicture}}


\newcommand{\lcmtwobythreelineone}[4]{
	\def \lcmtwobythreelineonenumber {#1}
	\def \lcmtwobythreelineonefactorone {#2}
	\def \lcmtwobythreelineonefactortwo {#3}
	\def \lcmtwobythreelineonefactorthree {#4}}

\newcommand{\lcmtwobythreelinetwo}[4]{
	\def \lcmtwobythreelinetwonumber {#1}
	\def \lcmtwobythreelinetwofactorone {#2}
	\def \lcmtwobythreelinetwofactortwo {#3}
	\def \lcmtwobythreelinetwofactorthree {#4}
}

\newcommand{\lcmtwobythreelinethree}[4]{
	\def \lcmtwobythreelinethreefactorone {#1}
	\def \lcmtwobythreelinethreefactortwo {#2}
	\def \lcmtwobythreelinethreefactorthree {#3}
	\def \lcmtwobythreelinethreelcm {#4}
}

\newcommand{\lcmtwobythree}[2]{\begin{tikzpicture}[remember picture, overlay]
		\color{red}
		\node at (current page.south west) (sw) {};
		\node[anchor=south west, inner sep=0pt] at ($(sw) + (#1, #2) $)   (a) {
			\begin{minipage}[t]{0.75\textwidth}
				\pause Find the LCM: 
				
				\begin{tabular}{rcccccc}
					
					$ \pause \lcmtwobythreelineonenumber $ & $=$ & $ \pause \lcmtwobythreelineonefactorone $ & $ \pause \lcmtwobythreelineonefactortwo $ & $ \pause \lcmtwobythreelineonefactorthree $ & & \\
					
					$ \pause \lcmtwobythreelinetwonumber $ & $=$ & $ \pause \lcmtwobythreelinetwofactorone $ & $ \pause \lcmtwobythreelinetwofactortwo $ & $ \pause \lcmtwobythreelinetwofactorthree $ & & \\
					
					\hline
					
					\pause LCM & $=$ & \pause $ (\lcmtwobythreelinethreefactorone) $ & \pause $ (\lcmtwobythreelinethreefactortwo) $ & \pause $ (\lcmtwobythreelinethreefactorthree) $ & $=$ & \pause $ \lcmtwobythreelinethreelcm $  \\
					
				\end{tabular}
		\end{minipage}};
\end{tikzpicture}}

\newcommand{\plotsystvars}[8]{
	\def \eqone {#1}
	\def \eqtwo {#2}
	\def \solx {#3}
	\def \soly {#4}
	\def \solanchor {#5}
	\def \labelxshift {#6}
	\def \labelyshift {#7}
	\def \solmarksize {#8}
}

\newcommand{\plotsyst}[7]{
\begin{tikzpicture}[scale=#1]
	
	\begin{axis} 
		[
		xticklabels={}, 
		yticklabels={}, 
		ymin=-#2, ymax=#2,
		xmin=-#2, xmax=#2,
		axis lines = center, 
		inner axis line style={Latex-Latex,very thick}, 
		grid=both,
		minor tick num=#7, 
		tick align=inside,
		after end axis/.code={
			\path (axis cs: \solx,\soly) 
			node [anchor=\solanchor, xshift=\labelxshift pt, yshift=\labelyshift pt] {$ (\solx, \soly) $}; } 
		] 
		
		\addplot[<->, >={Latex[round]},  ultra thick, domain=#3:#4, samples=200]{\eqone}node[]{};
		
		\addplot[<->, >={Latex[round]},  ultra thick, domain=#5:#6, samples=200]{\eqtwo}node[]{};
			
		\pause \addplot[only marks, mark=*, mark size=\solmarksize pt, color=black,] coordinates {(\solx, \soly)};
	\end{axis} 

\end{tikzpicture} 
}
\usetheme{default}
\usecolortheme{seahorse}

\title[] {Systems of Linear Equations in Two Variables}
\author{Jonathan R. Bacolod}
\institute[SHS]{Sauyo High School}
\date{}

\begin{document}
	\frame{\titlepage}
		
	\begin{frame}
		\frametitle{What is a System of Linear Equations?}

		\begin{itemize}
			\item It consists of two or more linear equations with the same variables considered together for which a common solution is desired.
			\item \pause It is also called Simultaneous Equations.
		\end{itemize}
	\end{frame}

    \begin{frame}
    	\frametitle{Example}
    	\[
\systeme{2x+y=10, x+y=6, x+3y=13}
\]
    \end{frame}

    \begin{frame}
    	\frametitle{What is a Solution Set of a System of Linear Equations?}
    	A solution set is an ordered pair of real numbers that satisfies both equations of
    	the system.
    \end{frame}

   \begin{frame}
   	\frametitle{Example 1}
 		\plotit{sample-system}{0.4}{8}
   \end{frame}

    \begin{frame}
    	\frametitle{How to Check Whether an Ordered Pair is a Solution to a Linear System?}
    	\begin{enumerate}
    		\item Replace $ x $ and $ y $ with the given values in both equations.
    		\item \pause Simplify. Check if the ordered pair satisfies both equations.
    	\end{enumerate}
    \end{frame}

     \begin{frame}
    	\frametitle{Example 2}
    	Is the ordered pair $ (2, 1) $ a solution to the system $ \systeme{x - y = 1, x + y = 3} $?
    \end{frame}

    \begin{frame}
    	\frametitle{Example 2 \hspace{0.2em}
    		\only<1-4>{\begin{minipage}[m]{.7\textwidth}
    				\small Step 1: Replace $ x $ and $ y $ with the given values in both equations.
    		\end{minipage}}
    		\only<5->{\begin{minipage}[m]{.7\textwidth}
    				\small Step 2: Simplify. Check if the ordered pair satisfies both equations.
    		\end{minipage}}}
    	    	
    \begin{TAB}(@, 1mm, 5mm)[2mm]{ll}{ccccc}
    	Given: $ x = 2 $,  & \pause $ y = 1 $  \\%
    	
    	\pause $ x - y = 1 $  &  \\%
    	
    	\pause $ 2 - 1 = 1 $ & \pause Substitution Property \\%
    	
    	\pause $ 1 = 1 $ & \pause Simplification \\%
    	& \\
    \end{TAB}
    
    	$ \therefore $ the ordered pair $ (2, 1) $ satisfies the equation $ x - y = 1 $.
    \end{frame}

    \begin{frame}
    	\frametitle{Example 2}
    	Is the ordered pair $ (2, 1) $ a solution to the system $ \systeme{x - y = 1, x + y = 3} $?
    \end{frame}
    
    \begin{frame}
    	\frametitle{Example 2 \hspace{0.2em}
    		\only<1-4>{\begin{minipage}[m]{.7\textwidth}
    				\small Step 1: Replace $ x $ and $ y $ with the given values in both equations.
    		\end{minipage}}
    		\only<5->{\begin{minipage}[m]{.7\textwidth}
    				\small Step 2: Simplify. Check if the ordered pair satisfies both equations.
    	\end{minipage}}}
    	
    	\begin{TAB}(@, 1mm, 5mm)[2mm]{ll}{ccccc}
    		Given: $ x = 2 $,  & \pause $ y = 1 $  \\%
    		
    		\pause $ x + y = 3 $  &  \\%
    		
    		\pause $ 2 + 1 = 3 $ & \pause Substitution Property \\%
    		
    		\pause $ 3 = 1 $ & \pause Simplification \\%
    		& \\
    	\end{TAB}
    	
    	$ \therefore $ the ordered pair $ (2, 1) $ satisfies the equation $ x + y = 3 $.
    \end{frame}
    
   \begin{frame}
   	\frametitle{Example 2}
   	 $ \therefore $ since the ordered pair $ (2, 1) $ satisfies both the equations $ x - y = 1 $ and $ x + y = 3 $, it is a solution to the system $ \systeme{x - y = 1, x + y = 3} $.
   \end{frame}

    \begin{frame}
    	\frametitle{What are the Kinds of Systems of Linear Equations?}
    	\begin{enumerate}
    		\item Consistent-dependent: a system of equations that can be rewritten as identical equations and have an infinite solution
    		\item \pause Consistent-independent: a system of equations that can not be rewritten as contradicting equations nor identical equations; they stay
    		different and have one solution
    		\item \pause Inconsistent: a system of equations that can be rewritten as
    		contradicting equations and has no solution
    	\end{enumerate}
    \end{frame}

    \begin{frame}
    	\frametitle{Example 1}
    	Determine whether the following system of linear equations is consistent-dependent, consistent-independent, or inconsistent. Then state the number of solution/s it has.
    	\[ \systeme{2x + y = 4, 4x + 2y = 8} \] 
    \end{frame}

    \begin{frame}
    	\frametitle{Example 1}
    	
    	\begin{TAB}(@, 1mm, 5mm)[2mm]{ll}{cccc}
    		$ \systeme{2x + y = 4, 4x + 2y = 8} $  & \\%
    		
    		\pause $ \systeme{2(2x + y) = 2(4), 4x + 2y = 8} $  & Multiplication Property \\%
    		
    		\pause $ \systeme{4x + 2y = 8, 4x + 2y = 8} $ & \pause Distributive Property \\%
    			& \\
    	\end{TAB}
    	
    	$ \therefore $ since the equations are identical, it is a \textbf{Consistent-dependent} system and has infinitely many solutions.
    \end{frame}

    \begin{frame}
    	\frametitle{Example 2}
    	Determine whether the following system of linear equations is consistent-dependent, consistent-independent, or inconsistent. Then state the number of solution/s it has.
    	\[ \systeme{2x + 3y = 4, 6x + 9y = 8} \] 
    \end{frame}

    \begin{frame}
    	\frametitle{Example 2}
    	
    	\begin{TAB}(@, 1mm, 5mm)[2mm]{ll}{cccc}
    		$ \systeme{2x + 3y = 4, 6x + 9y = 8} $  & \\%
    		
    		\pause $ \systeme{3(2x + 3y) = 3(4), 6x + 9y = 8} $  & Multiplication Property \\%
    		
    		\pause $ \systeme{6x + 9y = 12, 6x + 9y = 8} $ & \pause Distributive Property \\%
    		& \\
    	\end{TAB}
    	
    	$ \therefore $ since the equations are contradicting, it is an \textbf{Inconsistent} system and has no solution.
    \end{frame}
    
    \begin{frame}
    	\frametitle{Example 3}
    	Determine whether the following system of linear equations is consistent-dependent, consistent-independent, or inconsistent. Then state the number of solution/s it has.
    	\[ \systeme{x - 2y = 1, 2x + 4y = 3} \] 
    \end{frame}
    
    \begin{frame}
    	\frametitle{Example 3}
    	
    	\begin{TAB}(@, 1mm, 5mm)[2mm]{ll}{ccc}
    		$ \systeme{x - 2y = 1, 2x + 4y = 3} $  & \\%
    		
    		\pause $ \systeme{2(x - 2y) = 2(1), 2x + 4y = 3} $  & Multiplication Property \\%
    		
    		\pause $ \systeme{2x - 4y = 2, 2x + 4y = 3} $ &  Distributive Property \\%
%    		& \\
    	\end{TAB}
    	
    	$ \therefore $ since the equations are not contradicting and not identical, it is a \textbf{Consistent-independent} system and has one solution.
    \end{frame}
    
    
    \begin{frame}
    	\begin{center}
    		\textbf{\LARGE Thank you for watching.}
    	\end{center}
    \end{frame}
	
\end{document}